\section{Formule e formule inverse}

In fisica, lo scopo finale è comprendere come le grandezze fisiche  sono legate tra di loro. Approfondiremo il concetto di grandezza più avanti. Per il momento, pensate che una grandezza può essere una proprietà di un oggetto (ad esempio la sua massa) espressa da un numero ed indicata da una lettera. La parte della matematica che studia le formule (dette più correttamente \textbf{equazioni}) si chiama algebra. L'algebra, al pari dell'aritmetica, tratta di numeri, solo che questi sono rappresentati da lettere. In fisica, le lettere rappresentano grandezze e sono costituite da un numero che esprime una misura fatta rispetto ad una certa unità. Ad esempio:
\[
m=\SI{30}{kg}
\]

indica il valore di una massa. In questo paragrafo, vedremo come si estrae una varibile incognita da una formula (le cosiddette formule inverse). Non ci interessa chiarire il significato dei simboli ma piuttosto mostrare come si ricava una incognita usando due principi matematici (chiamati principi di equivalenza). In quanto segue, darò per scontata la conoscenza delle frazioni e di questi principi. Quì voglio solo mostrarvi con un serie di esempi, come si ricava una formula inversa. Partiamo:

	
	
	\[F= ma, \,\,  a = \text{?} \rightarrow \frac{F}{m} =\frac{\cancel{m} a}{\cancel{m}}\rightarrow a=\frac{F}{m}
	\]

 Abbiamo diviso entrambi i lati dell'equazione (chiamati membri) per uno stesso numero (la massa $m$) in modo da lasciare la nostra incognita (la lettera $a$) isolata al numeratore. La formula quindi, ci permette di calcolare $a$ conoscendo $F$ ed $m$.

\[
d=\frac{m}{V} \,\,\text{,}\,\,\,  m=\text{?} \rightarrow  V\cdot  d = \frac{m}{\cancel{V}}\cdot \cancel{V}.\rightarrow m = d V.
\]

Abbiamo ricavato la formula per la massa, noti densità e volume.

\[
F=k\cdot \frac{q_1 q_2}{r^2} \,\,\text{,}\,\,\,  r=\text{?} \rightarrow r^2 = k\cdot \frac{q_1 q_2}{F}\rightarrow r=\sqrt{k\cdot \frac{q_1 q_2}{F}}
\]
In questo caso abbiamo applicato la regola secondo cui, spostando una lettera che è moltiplicata da un lato all'altro dell'uguaglianza, se è denominatore, diventa numeratore e viceversa ($r^2$ è diventato numeratore ed $F$ denominatore). In fine, per eliminare il quadrato, abbiamo estratto la radice quadrata, poichè è l'operazione inversa del quadrato. Provate a calcolare la radice quadrata di $2^2$ con la calcolatrice e vedrete che otterrete 2!.
\[
s= s_0 + v t \,\,\text{,}\,\,\,  t=\text{?} \rightarrow s-s_0 = s_0 -s_0 + v t\rightarrow \frac{s-s_0}{v} = \frac{\cancel{v} t }{\cancel{v}}\rightarrow t=\frac{s-s_0}{v}
\]
In questo caso, per ricavare $t$, abbiamo aggiunto ai due membri il termine $-s_0$ in modo da lasciare il termine $v t$ e solo dopo abbiamo diviso per $v$.


\[
V=\pi r^2 h \,\,\text{,}\,\,\,  h=\text{?} \rightarrow \frac{V}{\pi r^2} = \frac{\cancel{ \pi r^2} h}{\cancel{\pi r^2}} \rightarrow h= \frac{V}{\pi r^2 }
\]
Abbiamo ricavato l'altezza del cilindro conoscendo raggio e volume. Nella prossima equazione, vogliamo calcolare il diametro note altezza e volume.
\[
V=\pi r^2 h =\pi \left(\frac{D}{2} \right)^2 h= \pi \frac{D^2}{4}  h\,\,\text{,}\,\,\,  D=\text{?} 
\]
Procediamo in analogia ai casi precedenti:
\[
\frac{4 V}{\pi h} = \frac{ \cancel{ 4 \pi h }D^2 }{ \cancel{4 \pi h}} \rightarrow D=\sqrt{\frac{4 V}{\pi h}}
\]




