\section{Approssimazione della Gaussiana}
In un esperimento, se ripetiamo molte volte una misura e calcoliamo la media e la deviazione standard, al crescere del numero di misure, l'istogramma dei dati approssima sempre meglio la distribuzione gaussiana. Questo è dovuto al teorema centrale del limite, il quale afferma tra l'altro che le formule per media e deviazione standard della media, se applicate a campioni di dati molto grandi, ci restituiscono proprio i valori teorici di queste grandezze che restano comunque un concetto teorico. Quando una persona fà un esperimento di misura, a quella persona e quell'apparato, corrisponde una media teorica, ossia i valori delle grandezze si sparpaglieranno in un certo modo perché quella persona e quell'apparato hanno una sorta di sensibilità: se qualcun altro fà la misura, questa si \textit{sparpaglierà} diversamente. Ad esempio, se Marco lascia cadere un pallina mille volte (povero Marco... ) e misura la media e la deviazione standard dei tempi di caduta, magari otterrà una deviazione standard di 0,8 s e una media di 0,2 s. Se facciamo cadere la pallina usando una fotocellula per misurare il tempo, le misure si sparpaglieranno di meno e avremo una deviazione standard ad esempio di 0,01 s con una media di 0,16 s. Come si vede, la grandezza da misurare (il tempo di caduta) è la stessa ma uno dei sue sistemi è più preciso (quello con la fotocellula). Se andassimo a costruire gli istogrammi sperimentali, quello della misura del ragazzo, avrebbe una forma a campana più larga perché la deviazione standard misura tra l'altro quanto è largo l'istogramma. 

