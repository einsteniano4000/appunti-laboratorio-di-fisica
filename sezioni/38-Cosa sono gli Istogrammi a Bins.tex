\section{Cosa sono gli Istogrammi a Bins}
Un istogramma è uno strumento grafico utilizzato per rappresentare la distribuzione di un insieme di dati. Esso divide i dati in intervalli, chiamati \textit{bins}, e conta il numero di eventi (o frequenze) che ricadono in ciascun intervallo.

L'area di ciascun rettangolo nell'istogramma rappresenta la frequenza degli eventi nell'intervallo considerato rispetto al totale degli eventi. Questo significa che l'altezza del rettangolo (quando i bins hanno larghezze uguali) è proporzionale al numero di eventi in quel bin. Se i bins hanno larghezze diverse, l'altezza del rettangolo è proporzionale alla densità di frequenza, in modo che l'area rimanga rappresentativa della frequenza relativa.

