\section{Proprietà degli strumenti}
Gli strumenti di misura appartengono a due grandi categorie: gli strumenti analogici (dotati di una scala e a volte di un indice mobile, come negli orologi o i cronometri a lancetta) e digitali. Uno strumento digitale ha un display elettronico sul quale leggiamo direttamente la misura. In laboratorio di elettronica userete principalmente strumenti digitali i quali hanno grandi vantaggi rispetto a quelli analogici. Tuttavia è bene osservare che spesso tali strumenti, danno una certa sicurezza in quanto risparmiano la fatica di leggere una scala e  forniscono immediatamente il risultato. Comunque, come tutti gli strumenti di misura, anch'essi hanno dei limiti di funzionamento, possono cioè danneggiarsi se usati male, e, inoltre, avendo molte funzioni, possono avere una curva di apprendimento molto ripida.  Entrambi i tipi di strumenti hanno quattro importanti caratteristiche: \textbf{sensibilità}, \textbf{portata massima (e minima)}, \textbf{prontezza}, \textbf{precisione}. 
\begin{description}
\item[Sensibilità] Si tratta, come abbiamo già detto, della minima \textit{variazione} della grandezza che può essere rilevata dallo strumento. Essa, nelle misure dirette, rappresenta anche l'incertezza di misura (o errore assoluto). Per determinare la sensibilità di uno strumento analogico si osserva la sua scala. Nel caso di un righello o di un cilindro graduato, basta leggere due valori scritti e contare le tacche fra di essi. La sensibilità è data dalla differenza tra i due valori, divisa per il numero di tacche. Per strumenti più sofisticati, tipo un calibro o una bilancia elettronica, di solito la sensibilità è indicata sullo strumento stesso. In laboratorio abbiamo in dotazione vari tipi di calibro con sensibilità di $\SI{0,05}{mm}$, $\SI{0,1}{mm}$ e persino $\SI{0,01}{mm}$ (calibro Palmer).
\item[Portata massima] Si tratta del massimo valore della grandezza che uno strumento può rilevare. Essa è particolarmente importante ai fini del corretto uso dello strumento. Se si tenta di leggere un valore più grande della portata, lo strumento rischia di rompersi. A volte si parla anche di \textit{campo} di misura. Il campo è, più in generale, l'intervallo di valori che uno strumento può leggere. Se abbiamo un termometro che legge da $\SI{-20}{\celsius}$ a $\SI{100}{\celsius}$, allora il campo è $\SI{-20}{\celsius}\div\SI{100}{\celsius}$. 


\item[Precisione] La precisione è definita come il rapporto tra sensibilità e portata. In quanto tale non ha unità di misura. Sui testi si trova una diversa definizione di precisione. Si dice che uno strumento è più preciso di un altro se, ripetendo la misura di una grandezza costante, si ottengono sempre gli stessi valori oppure valori più vicini. Noi preferiamo definire questa come l'affidabilità di uno strumento.
\item[Prontezza] Si tratta della velocità con cui lo strumento ci restituisce il valore misurato. Se uno strumento è più rapido di un altro nel fornire una lettura, si dice che è più \textit{pronto}.

E' importante osservare che, potendo scegliere, è sempre bene preferire lo strumento con sensibilità piccola ma non sempre è possibile. Se il nostro strumento ha una portata non adatta, potremmo romperlo. Si pensi al caso di una bilancia di precisione. Queste bilance, a volte hanno portate di poco più di un kilo e potrebbero rompersi se ci caricassimo più peso (oppure segnerebbero errore (se digitali)). Riprenderemo questo concetto più avanti dopo una lunga parentesi matematica..
\end{description}


