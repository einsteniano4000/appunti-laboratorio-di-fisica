\section{Introduzione}
Python è un linguaggio di programmazione versatile e potente, ideale per iniziare a programmare e per l'analisi dati. In questa guida, esploreremo la sintassi base, l'ambiente di sviluppo Google Colab, e alcuni elementi chiave di Python, NumPy e Matplotlib. In queste pagine sono inseriti i listati degli script python che uderemo nell'analisi dei dati. Si tratta di comandi del linguaggio  evidenziati da un bordo grigio. I listati non si possono copiare da questi appunti, sono disponibili in una cartella sul drive. Alcuni caratteri, richiedono l'uso combinato di più tasti (ad esempio, il carattere ''\textasciicircum`` sulla mia tastiera, si ottiene premendo contemporaneamente shift (una freccia spessa verso l'alto che si trova sulla destra della tastiera) e il tasto ì accentato (in alto a destra)). Per questi simboli, occorre imparare le opportune combinazioni che cambiano da tastiera a tastiera. Siccome si tratta solo di caratteri estetici, potete sostituirli come vi pare negli script (questi lavorano solo con numeri e lettere, i simboli non sono importanti ma dovendo inserire le unità di misura bisogna un pò ingegnarsi).
 
