\section{Esercizi}
\begin{esercizio}
E' stato misurato il volume di un oggetto ottenendo il risultato
\[
V=\left( 90,0 \pm 0,4\right)\si{mL}.
\]
Determina il risultato corretto nel sistema SI.\\
\hspace*{\fill}$\left[V=\left( 90,0 \pm 0,4\right)\times10^{-6}\,\si{m^3}\right]$ 
\end{esercizio}

\begin{esercizio}
In laboratorio è stata misurata la durata del periodo di un pendolo ottenendo il valore $T=\SI{1,874}{s}$ con una semidispersione di $\Delta T =\SI{0,0291}{s}$. Scrivi il risultato in maniera corretta e calcola l'errore percentuale.\\
\hspace*{\fill} $\left[T=\left(1,87 \pm 0,03\right)\si{s}\text{;} E_{\%}=1,6\%\right]$
\end{esercizio}

\begin{esercizio}
Un gruppo di studenti ha misurato il periodo di oscillazione di un pendolo semplice per 20 volte. I dati raccolti sono riportati nella tabella \ref{tab:pend}. Calcola la media delle misure del periodo (\(\overline{T}\)) e l'incertezza $\Delta T$ e scrivi il risultato in forma corretta.
\hspace*{\fill} $\left[T=\left(2,16 \pm 0,02 \right) \,\si{s} \right]$.
\begin{table}[h!]
\centering
\caption{Misure del periodo di oscillazione di un pendolo}
\label{tab:pend}
\begin{tabular}{cc}
\toprule
\textbf{Numero di occorrenze} & \textbf{Misura del periodo (s)} \\
\midrule
5  & 2.14 \\
4  & 2.15 \\
5  & 2.16 \\
3  & 2.17 \\
3  & 2.18 \\
\bottomrule
\end{tabular}
\end{table}


\end{esercizio}


\begin{esercizio}
E' stato misurato l'intervallo d'incertezza per la massa di un oggetto e si è ottenuto l'intervallo compreso tra $\SI{459,7}{g}$ e $\SI{460,3}{g}$. Determina la massa e il suo errore e scrivi il risultato in forma corretta.\\
 \hspace*{\fill}  $\left[M=\left(460,0 \pm 0,3\right)\si{g}\right]$
\end{esercizio}


\begin{esercizio}
Spiega la differenza tra incertezza assoluta e relativa.
\end{esercizio}

\begin{esercizio}
Elenca le seguenti misure per ordine crescente di precisione.
\begin{multicols}{3}
\begin{elenco}
 \item[a)] $L=\left(20,2 \pm 0,1\right)\si{cm}$
 \item[b)] $M=\left(4,01 \pm 0,01\right)\si{kg}$
 \item[c)] $t=\left(1,22 \pm 0,02\right)\si{s}$ \\
 \hspace*{\fill}  $\left[\text{b, a, c}\right]$
\end{elenco}
\end{multicols}

\end{esercizio}


