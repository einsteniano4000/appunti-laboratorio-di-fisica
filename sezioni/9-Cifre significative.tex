\section{Cifre significative}
	Diamo una importante definizione:
\begin{csf}
	Le cifre significative di un numero, sono tutte le cifre certe e la prima incerta.
\end{csf}
Dunque, per conoscere le cifre significative di un numero, occorre conoscere l'incertezza su tale numero. L'incertezza è un concetto nuovo per voi ma abbiamo già visto un esempio di incertezza (la sensibilità nelle misure dirette). La misura $L = \SI{10,234671}{\milli\meter}$,con un errore di $\SI{1}{\milli\meter}$, avrebbe come cifre significative solo le prime 2 (1 e 0) perché 1 è certa e 0 è la prima cifra incerta, dunque non ha senso scrivere il 2, il 3 e le altre cifre più a destra, per cui scriveremo:
\[
\SI{10,234671}{\milli\meter} \approx \SI{10}{\milli\meter}
\]
Se la misura fosse stata $\SI{10,634671}{\milli\meter}$, poiché 6 è maggiore di 5, avremmo approssimato come segue:
\[
\SI{10,634671}{\milli\meter} \approx \SI{11}{\milli\meter}
\]
\subsection{Valutazione a vista delle cifre significative}
Se in un testo non ci viene data l'incertezza ma solo la misura di una grandezza, ci sono convenzioni per valutare le cifre significative. Si assume che tutti gli zeri prima della prima cifra diversa da zero, siano non significativi. Ad esempio, nel numero 0,00034040, i primi quattro zeri non sono significativi. Lo zero tra i due 4 è significativo e lo è anche lo zero finale, dunque il numero 0,00034040 ha 5 cifre significative. Se volessimo scrivere il numero con 4 cifre significative, dovremmeo prima arrotondare e poi togliere l'ultima cifra, nel nostro caso quindi, otterremmo 0,0003404 perché la prima cifra tolta è uno zero. Se una misura è scritta in notazione scientifica, si contano solo le cifre significative della parte decimale, quindi il numero $\SI{1,206e+3}{\milli\meter}$ ha 4 cifre significative. Se volessimo approssimarlo a 3 cifre, scriveremmo: $\SI{1,206e+3}{\milli\meter} \approx \SI{1,21e+3}{\milli\meter}$ perché abbiamo eliminato una cifra maggiore di 5. Se la cifra che elimiamo è 5, non c'è una regola condivisa (noi però approssimeremo sempre per difetto).



\subsection{Cifre significative nelle operazioni}

Una delle domande più frequenti degli studenti durante le verifiche, riguarda le cifre decimali nei risultati: quante cifre bisogna tenere nei calcoli? La calcolatrice scientifica, comunemente ha un display che mostra 12 cifre ma non è sempre necessario tenerle tutte. Tutto dipende dalle cifre significative nei dati. Le regole, che non dimostreremo (ma useremo), sono due:
\begin{csp}
Moltiplicando o dividendo due grandezze, il risultato và fornito con il numero di cifre significative di quella che ne ha di meno.
\end{csp}
Se ad esempio, abbiamo v=$\SI{12,6}{\meter\per\second}$ e t=$\SI{0,10}{\second}$, allora scriveremo:
\[
s=v\cdot t =  \left(\SI{12,6}{\meter\per\second} \right)\cdot\left( \SI{0,10}{\second}\right) = \SI{1,26}{\meter\per\second} \approx \SI{1,3}{\meter}. 
\]
\begin{css}
Sommando o sottraendo due grandezze, si tengono nel risultato, solo le cifre decimali ottenute sommando cifre significative.
\end{css}
Se ad esempio, abbiamo $s_1=\SI{45,26}{\meter}$ e $s_2=\SI{17,4}{\meter}$, abbiamo: $s_1 +s_2 = \SI{62,66}{\meter} \approx \SI{62,7}{\meter}$. Da notare che in questa definizione si usano le cifre decimali, mentre nella precedente le cifre significative. Queste due regole sono regole approssimate, se vogliamo una valutazione corretta delle cifre significative nelle operazioni, dobbiamo applicare le regole del prossimo capitolo.

