\section{Densità} Dato un corpo (solido, liquido o gassoso) di massa $M$ e volume $V$, la densità è la grandezza (derivata):
\[
d=\frac{M}{V}
\]
Guardando la formula, ricaviamo che l'unità di misura è quella di una massa diviso un volume. Nel sistema internazionale, l'unità base è $\si{\kilo\gram\per\cubic\meter}$ ma esistono ovviamente altre unità. Ad esempio $\si{\gram\per\cubic\centi\meter}$.

\begin{testexample}[\thetcbcounter \,Calcoli con la densità]
$\SI{7860}{\kilo\gram\per\cubic\meter} = \cdots  \si{\kilo\gram\per\cubic\centi\meter} $. Dobbiamo trasformare il denominatore con le regole che conosciamo (ricordiamo che $\SI{1}{\cubic\meter} = \SI{e+6}{\cubic\centi\meter}$):
\[
\SI{7860}{\kilo\gram\per\cubic\meter} =\frac{\SI{7860}{\kilogram}}{\SI{e+6}{\cubic\centi\meter}}  =\SI{7,860e-3}{\kilo\gram\per\cubic\centi\meter}
\]

$\SI{1000}{\kilo\gram\per\cubic\meter} = \cdots \si{\gram\per\milli\liter}$. Ricordiamo che $\SI{1}{\cubic\centi\meter} = \SI{1}{\milli\liter}$:



\[
\SI{1000}{\kilo\gram\per\cubic\meter} = \frac{1000\cdot \SI{e+3}{\gram}}{\SI{e6}{\milli\liter}} = \SI{1}{\gram\per\milli\liter}
\]

Questa equivalenza è molto importante nella pratica dove si usano spesso i millilitri, dunque la mettiamo in evidenza:
\[
\colorboxed{ocre}{
\SI{e3}{\kilo\gram\per\cubic\meter} = \SI{1}{\gram\per\milli\liter}
}
\]
\end{testexample}

\begin{testexample}[\thetcbcounter \,Formule inverse e calcolo dimensionale.]

Introduciamo il calcolo delle unità di misura (calcolo dimensionale). Per ricavare l' unità di misura di una grandezza G (rappresentata dalla scrittura [G]) basta inserire nella formula, al posto di ogni grandezza, la sua unità di misura. Facciamo un esempio.
Notiamo che la formula del volume è corretta dal punto di vista delle unità di misura. Infatti, $[m]=\text{kg}\;\, [d]=\frac{\text{kg}}{{\text{m}^3}}$ e quindi:
        \[
        [V]=\frac{\text{kg}}{\frac{\text{kg}}{{\text{m}^3}}}=\cancel{\text{kg}}\cdot \frac{\text{m}^3}{\cancel{\text{kg}}   }=\text{m}^3.
      \]
\end{testexample}



\begin{testexample}[\thetcbcounter \,Esercizio sulla densità]

Un'ampolla di vetro ha una massa di $\SI{85,382}{\gram}$ e un volume di $\SI{80}{\milli\liter}$. L'ampolla viene riempita con gas xenon e la sua massa diventa $\SI{85,852}{\gram}$.
\begin{itemize}
\item Calcola la densità $d_{X_e}$ dello xenon.
\item In seguito, la lampada viene di nuovo svuotata e riempita di idrogeno, fino ad ottenere una densità di $0,153\, d_{X_e}$. Determina la  massa totale della lampada riempita di idrogeno.
\end{itemize}
\end{testexample}

Approfittiamo per poter ripassare le formule inverse e le cifre significative. Calcoliamo anzitutto la massa dello xeon per differenza:
\[
M_{X_e} = \SI{85,852}{\gram} -\SI{85,382}{\gram}  = \SI{0,470}{\gram} =\SI{4,70e-4}{\kilo\gram}
\]
Il testo vuole la densità in $\si{\kilo\gram\per\cubic\meter}$, quindi trasformiamo il volume in $\si{\cubic\meter}$ :  $\SI{80}{\milli\liter} =\SI{8,0e-5}{\cubic\meter}$.
Usando la formula per la densità, abbiamo:
\[
d_{X_e}=\frac{M}{V} = \frac{\SI{4,70e-4}{\kilo\gram}}{\SI{8,0e-5}{\cubic\meter}}=\SI{5,875}{\kilo\gram\per\cubic\meter}      \approx  \SI{5,9}{\kilo\gram\per\cubic\meter}.  
\]
Per determinare la massa totale della lampada riempita di idrogeno, dobbiamo sommare la massa dell'ampolla e la massa dell'idrogeno, massa che possiamo calcolare con la formula: $M=d\cdot V$.
Dai dati, vediamo che la densità dell'idrogeno, vale: 
\[
d_{H} = 0,153 \cdot\left( \SI{5,9}{\kilo\gram\per\cubic\meter}\right) = \SI{0,9027}{\kilo\gram\per\cubic\meter } \approx \SI{0,90}{\kilo\gram\per\cubic\meter }.
\]
La massa totale dell'ampolla e dell'idrogeno è dunque:
\[
\begin{aligned}
    M &= \SI{0,085382}{\kilo\gram} + d \cdot V \\
      &= \SI{0,085382}{\kilo\gram} + \left(\SI{0,90}{\kilo\gram\per\cubic\meter}\right) \times \left(\SI{8,0e-5}{\cubic\meter}\right) \\
      &= \SI{0,085382}{\kilo\gram}  + \SI{0,00045}{\kilo\gram} = \SI{0,085454}{\kilo\gram} = \SI{85,454}{\gram}.
\end{aligned}
\]

Notiamo che abbiamo dovuto fare un passaggio in più per poter decidere nella somma quante cifre tenere, togliendo la notazione scientifica. Questa operazione è spesso lunga per cui capiterà in calcoli molto lunghi di non usare la regola per le cifre significative della somma e tenere nel risultato finale, tante cifre significative quante presenti nei dati meno precisi.


\begin{testexample}[\thetcbcounter \,Esercizio sulla densità e il volume]
Un cilindro di metallo ha un'altezza di \(h = \SI{50,0}{\centi\metre}\) e un diametro di base \(d = \SI{20,0}{\centi\metre}\). La massa del cilindro è \(M = \SI{15,0}{\kilogram}\).

\begin{enumerate}
    \item \textbf{Calcola il volume del cilindro in litri.}
    \item \textbf{Calcola la densità del cilindro in \si{\kilogram\per\litre}.}
\end{enumerate}
\end{testexample}

Le formule sono:

\begin{itemize}
    \item Il volume \(V\) di un cilindro si calcola con la formula:
    \[
    V = \pi \left(\frac{d}{2}\right)^2 h = \frac{\pi d^2 h}{4}
    \]
    dove \(d\) è il diametro della base e \(h\) è l'altezza.

    \item La densità \(\rho\) di un oggetto si calcola con la formula:
    \[
    \rho = \frac{M}{V}
    \]
    dove \(M\) è la massa e \(V\) è il volume.
\end{itemize}

\textit{Svolgimento}\\

\begin{enumerate}
    \item \textbf{Calcolo del volume del cilindro:}

    \begin{itemize}
        \item Il diametro \(d\) e l'altezza \(h\) sono dati in centimetri:
        \[
        d = \SI{20,0}{\centi\metre}, \quad h = \SI{50,0}{\centi\metre}
        \]
        \item Il volume in centimetri cubi è:
        \[
        V = \frac{\pi d^2 h}{4} = \frac{\pi (\SI{20}{\centi\metre})^2 \times \SI{50}{\centi\metre}}{4} =  \SI{15708}{\centi\metre\cubed} \approx\SI{1,57e+4}{\centi\meter\cubed}
        \]
        \item Convertiamo il volume in litri (1 litro = 1000 centimetri cubi):
        \[
        V = \SI{1,57e+4}{\centi\meter\cubed} = \SI{15,7}{\liter}
        \]
    \end{itemize}

    \item \textbf{Calcolo della densità del cilindro:}

    \begin{itemize}
        \item La massa \(M\) del cilindro è già data in chilogrammi:
        \[
        M = \SI{15}{\kilogram}
        \]
        \item Utilizziamo la formula della densità:
        \[
        \rho = \frac{M}{V} = \frac{\SI{15}{\kilogram}}{\SI{15.7}{\litre}} \approx \SI{0.955}{\kilogram\per\litre}
        \]
    \end{itemize}
\end{enumerate}









\chapter{Errori di misura}
