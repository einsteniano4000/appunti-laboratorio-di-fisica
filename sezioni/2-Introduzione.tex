\section{Introduzione}
La fisica è una scienza sperimentale di tipo \textbf{quantitativo}, ossia una scienza che studia la natura attraverso l'uso della matematica. Le sue affermazioni sono sempre traducibili in equazioni. Cosa sia un'equazione lo imparerete presto. Da un punto di vista matematico, si tratta di una relazione (spesso algebrica, ma ce ne sono di più complesse) che lega tra loro i valori misurati di una o più grandezze fisiche. La fisica è un insieme di \textbf{teorie}. Una teoria, detta in modo semplice, è un modello della realtà. Un modello è una sorta di riproduzione semplificata della realtà. Per capire tutto ciò però, dobbiamo anzitutto cominciare ad allenare il cervello sui concetti matematici fondamentali.
