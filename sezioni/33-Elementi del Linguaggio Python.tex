\section{Elementi del Linguaggio Python}

\subsection{Variabili e Oggetti}
Le variabili in Python non richiedono dichiarazioni esplicite del tipo. Esempio:
\begin{lstlisting}[language=Python]
x = 10
y = "Ciao"
z = 3.14
\end{lstlisting}

\subsection{Stringhe}
Le stringhe sono delimitate da virgolette singole o doppie. Esempio:
\begin{lstlisting}[language=Python]
stringa = "Questa è una stringa"
\end{lstlisting}

\subsubsection{F-String}
Le f-string sono una funzionalità di Python che permette di inserire variabili all'interno delle stringhe in modo semplice e leggibile. Si utilizzano anteponendo una `f` alla stringa e racchiudendo le variabili tra parentesi graffe `{}`.

Esempio:
\begin{lstlisting}[language=Python]
nome = "Alice"
eta = 25
stringa = f"Il nome è {nome} e l'età è {eta} anni."
print(stringa)
\end{lstlisting}

In questo esempio, le variabili `nome` e `eta` sono incluse direttamente all'interno della stringa senza la necessità di concatenare manualmente o utilizzare metodi di formattazione complessi.

Esempio con espressioni:
\begin{lstlisting}[language=Python]
a = 10
b = 5
stringa = f"La somma di {a} e {b} è {a + b}."
print(stringa)
\end{lstlisting}

In questo caso, anche le espressioni possono essere valutate direttamente all'interno delle parentesi graffe.

\subsection{Costrutto di Selezione}
Il costrutto di selezione in Python usa le parole chiave \textit{if}, \textit{elif} e \textit{else}. Esempio:
\begin{lstlisting}[language=Python]
x = 5
if x > 0:
    print("x è positivo")
elif x == 0:
    print("x è zero")
else:
    print("x è negativo")
\end{lstlisting}

\subsection{Controllo di Flusso}
I controlli di flusso includono strutture come cicli e dichiarazioni condizionali.

\subsubsection{Indentazione}
Python utilizza l'indentazione per definire blocchi di codice. Ogni blocco deve essere indentato con lo stesso numero di spazi o tabulazioni. Esempio:
\begin{lstlisting}[language=Python]
x = 10
if x > 0:
    print("x è positivo")
    if x > 5:
        print("x è maggiore di 5")
\end{lstlisting}

\subsubsection{Cicli \textit{for}}
Il ciclo \textit{for} in Python è estremamente versatile e può essere utilizzato per iterare su una varietà di sequenze e strutture dati.

\paragraph{Iterare su una Lista}
Il ciclo \textit{for} può iterare attraverso ogni elemento di una lista. Esempio:
\begin{lstlisting}[language=Python]
frutti = ['mela', 'banana', 'ciliegia']
for frutto in frutti:
    print(frutto)
\end{lstlisting}

\paragraph{Iterare su un Range di Numeri}
Il ciclo \textit{for} può iterare attraverso un range di numeri generato dalla funzione \textit{range}. Esempio:
\begin{lstlisting}[language=Python]
for i in range(5):
    print(i)
\end{lstlisting}

\paragraph{Iterare su un Insieme}
Gli insiemi (\textit{sets}) sono collezioni non ordinate di elementi unici. Esempio:
\begin{lstlisting}[language=Python]
insieme = {1, 2, 3, 4}
for numero in insieme:
    print(numero)
\end{lstlisting}

\paragraph{Iterare su un Dizionario}
I dizionari (\textit{dict}) sono collezioni di coppie chiave-valore. Esempio di iterazione su chiavi e valori:
\begin{lstlisting}[language=Python]
dizionario = {'nome': 'Alice', 'età': 25, 'città': 'Roma'}

# Iterare sulle chiavi
for chiave in dizionario:
    print(chiave, dizionario[chiave])

# Iterare su chiavi e valori
for chiave, valore in dizionario.items():
    print(chiave, valore)
\end{lstlisting}

\paragraph{Enumerare gli Elementi di una Sequenza}
La funzione \textit{enumerate} restituisce una coppia (indice, valore) per ogni elemento di una sequenza. Esempio:
\begin{lstlisting}[language=Python]
frutti = ['mela', 'banana', 'ciliegia']
for indice, frutto in enumerate(frutti):
    print(indice, frutto)
\end{lstlisting}

\paragraph{Uso di \textit{zip} per Iterare su Più Sequenze}
La funzione \textit{zip} può essere utilizzata per iterare su più sequenze contemporaneamente. Esempio:
\begin{lstlisting}[language=Python]
nomi = ['Alice', 'Bob', 'Charlie']
eta = [25, 30, 35]

for nome, eta_persona in zip(nomi, eta):
    print(nome, eta_persona)
\end{lstlisting}

\subsubsection{Cicli \textit{while}}
Il ciclo \textit{while} itera finché una condizione è vera. È utile quando non si conosce in anticipo il numero di iterazioni. Ecco alcuni esempi:

\paragraph{Esempio Base}
Un ciclo `while` di base per contare da 0 a 4:
\begin{lstlisting}[language=Python]
count = 0
while count < 5:
    print(count)
    count += 1
\end{lstlisting}

\paragraph{Esempio con Lista}
In questo esempio, gli elementi della lista vengono rimossi uno alla volta fino a che la lista è vuota:
\begin{lstlisting}[language=Python]
lista = [1, 2, 3, 4, 5]
while lista:
    elemento = lista.pop(0)
    print(f"Eliminato: {elemento}, Lista rimanente: {lista}")
\end{lstlisting}

