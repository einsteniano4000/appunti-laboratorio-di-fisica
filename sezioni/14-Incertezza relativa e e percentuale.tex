\section{Incertezza relativa e e percentuale}
Data una misura $x=\left( \overline{x} \pm \Delta x \right)$ si definisce incertezza relativa il rapporto:
\[  \colorboxed{ocre}{
E_r =\frac{\Delta x}{\overline{x}}}
\]
L'incertezza relativa è un numero senza unità di misura, generalmente piccolo. Infatti, in un esperimento ben progettato, gli errori sono piccoli, rispetto alla grandezza, e dunque il rapporto è minore di uno. Strettamente legata all'incertezza, c'è l'incertezza precentuale, definita come il prodotto della prima per cento. L'incertezza relativa, consente di confrontare due misure, anche di grandezze diverse. Date due grandezze misurate, \textbf{quella più precisa ha l'incertezza relativa minore}. A titolo di esempio, calcoliamo l'incertezza relativa e percentuale della misura di periodo fatta sopra:
\[
Er=\frac{0,5}{1,2}=0,417
\]
mentre l'errore percentuale vale:
\[
E_{\%} = E_r \cdot 100 = 41,7\, \%
\]
Volendo confrontare la precisione di questa misura con questa misura di massa:
\[
m=\left(0,10 \pm 0,01 \right) \si{g}
\]
devo calcolare l'errore relativo della massa:
\[
E_r = \frac{0,01}{0,10}= 0,1
\]
quindi questa misura è più precisa.
