\documentclass{standalone}
\usepackage{tikz}

\begin{document}
\begin{tikzpicture}[x=1cm, y=1cm]

% Tacche principali e etichette
\foreach \x in {360,370} {
    \draw[thick] (\x-360,0) -- (\x-360,1); % Tacche lunghe
    \node[above] at (\x-360,1) {\pgfmathparse{int(\x/10)}\pgfmathresult|0}; % Etichette principali
}

% Tacche intermedie e etichette
\foreach \i in {1,...,9} {
    \draw (\i,0) -- (\i,0.5); % Tacche corte
    \node[above] at (\i,0.5) {\i}; % Etichette intermedie

    % Suddivisioni piccolissime
    \foreach \j in {1,...,4} {
        \draw (\i + \j*0.2,0) -- (\i + \j*0.2,0.25); % Tacche piccolissime
    }
}

% Suddivisioni tra la tacca 9 e la tacca principale a 10
\foreach \j in {1,...,4} {
    \draw (9 + \j*0.2,0) -- (9 + \j*0.2,0.25); % Tacche piccolissime
}

% Aggiungi ulteriori tacche piccole tra 0 e 1
\foreach \j in {1,...,4} {
    \draw (\j*0.2,0) -- (\j*0.2,0.25); % Tacche piccolissime
}

% Aggiungi freccia rossa verticale che indica la terza tacca dopo 360 cm
\draw[red, thick, ->] (0.6, -0.5) -- (0.6, 0.05); % Freccia rossa con punta alla y del terzo segmento

\end{tikzpicture}
\end{document}
