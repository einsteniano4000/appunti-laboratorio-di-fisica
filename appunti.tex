\documentclass[12pt,a4paper,oneside]{book}
\usepackage{multirow}


\usepackage[T1]{fontenc} 
\usepackage{textcomp} 	
%\usepackage[utf8x]{inputenc}
\usepackage[italian]{babel}
\renewcommand{\rmdefault}{ppl}
\usepackage{mathpazo}
\usepackage[scaled=.95]{helvet}
\usepackage{eulervm}
\usepackage{microtype}
\usepackage{cancel}
\usepackage{minted}
%=============================TIPOGRAFIA====================================
\usepackage[bindingoffset=6mm]{geometry}
\usepackage{microtype}
%===========================================================================
\usepackage{amsmath}
\usepackage{siunitx}
\usepackage{float}
\usepackage{amssymb}
\usepackage{pgfplots}
\usepackage{mdframed} % Per le scatole colorate
\sisetup{output-decimal-marker = {,}}
\sisetup{
  per-mode=fraction, % Usa la modalità frazione per le unità per
  fraction-function=\tfrac % Usa \tfrac per le frazioni
}
\usepackage{amsthm}
\usepackage{array}
\usepackage{booktabs}
\newtheorem{grf}{Grandezza fisica}
\newtheorem{udm}{Unità di misura}
\newtheorem{sdm}{Strumento di misura}
\newtheorem{nsc}{Notazione scientifica}
\newtheorem{fco}{Forma corretta}
\newtheorem{sen}{Sensibilità}
\newtheorem{csf}{Cifre significative}
\newtheorem{csp}{Cifre significative in un prodotto}
\newtheorem{css}{Cifre significative in una somma}
\usepackage{graphicx}
\usepackage[font=scriptsize,labelfont=bf,hypcap=false]{caption}
\usepackage{pgf,tikz}
\newcommand{\eserciziop}[1]{\noindent\textbf{\textit{Esercizio:}} #1}
\newcommand{\soluzione}[1]{\noindent\textcolor{blue}{\textbf{Soluzione:}} #1}
\usepackage{stmaryrd}

\usetikzlibrary{arrows,patterns,shadows}
 

\newcounter{testexample} %serve per l'esempio
\usepackage{cancel}
\usepackage{hyperref}
\usepackage{sectsty} %serve per colorare section
\usepackage{multicol}
\usepackage{enumitem} %serve per "elenco"
\usepackage[table]{xcolor}
 % Required for specifying colors by name
\definecolor{remarkpink}{RGB}{255,200,200}
\definecolor{remarkpinkdark}{RGB}{255,100,100}
\colorlet{ocre}{red!55!black}
\usepackage[margin=10pt,font=small,labelfont={bf,color=ocre}
,labelsep=endash]{caption}
\usepackage{titlesec}
\usepackage[most]{tcolorbox}
\usepackage{fontawesome5}
%aggiunta per cambiare colore alle sezioni



\newtcolorbox{remark}[1][]{
  enhanced,
  colback=pink!20,
  colframe=pink!60!black,
  boxrule=0.5pt,
  arc=4pt,
  left=2mm,
  right=2mm,
  top=2mm,
  bottom=2mm,
  overlay={
    \node[anchor=east, inner sep=0pt, outer sep=0pt] 
    at ([xshift=-2mm]frame.west) 
    {\textcolor{pink!60!black}{\faSmile[regular]}};
  },
  enlarge left by=5mm,
  drop shadow={
    shadow xshift=0.5mm,
    shadow yshift=-0.5mm,
    shadow scale=1.02,
    opacity=0.3,
  },
  #1
}






\newtcolorbox{definizione}{
  colback=yellow!20,
  colframe=yellow!60!black,
  title=Definizione,
  fonttitle=\bfseries
}


\newcommand*{\colorboxed}{}
\def\colorboxed#1#{%
  \colorboxedAux{#1}%
}
\newcommand*{\colorboxedAux}[3]{%
  % #1: optional argument for color model
  % #2: color specification
  % #3: formula
  \begingroup
    \colorlet{cb@saved}{.}%
    \color#1{#2}%
    \boxed{%
      \color{cb@saved}%
      #3%
    }%
  \endgroup
}
\usepackage{titletoc} 
% Chapter text styling
\titlecontents{chapter}[1.25cm] % Indentation
{\addvspace{12pt}\large\sffamily\bfseries} % Spacing and font options for chapters
{\color{ocre!60}\contentslabel[\Large\thecontentslabel]{1.25cm}\color{ocre}} % Chapter number
{\color{ocre}}  
{\color{ocre!60}\normalsize\;\titlerule*[.5pc]{.}\;\thecontentspage} % Page number

% Section text styling
\titlecontents{section}[1.25cm] % Indentation
{\addvspace{3pt}\sffamily\bfseries} % Spacing and font options for sections
{\contentslabel[\thecontentslabel]{1.25cm}} % Section number
{}
{\hfill\color{black}\thecontentspage} % Page number
[]

% Subsection text styling
\titlecontents{subsection}[1.25cm] % Indentation
{\addvspace{1pt}\sffamily\small} % Spacing and font options for subsections
{\contentslabel[\thecontentslabel]{1.25cm}} % Subsection number
{}
{\ \titlerule*[.5pc]{.}\;\thecontentspage} % Page number
[]


% Chapter text styling
\titlecontents{lchapter}[0em] % Indenting
{\addvspace{15pt}\large\sffamily\bfseries} % Spacing and font options for chapters
{\color{ocre}\contentslabel[\Large\thecontentslabel]{1.25cm}\color{ocre}} % Chapter number
{}  
{\color{ocre}\normalsize\sffamily\bfseries\;\titlerule*[.5pc]{.}\;\thecontentspage} % Page number

% Section text styling
\titlecontents{lsection}[0em] % Indenting
{\sffamily\small} % Spacing and font options for sections
{\contentslabel[\thecontentslabel]{1.25cm}} % Section number
{}
{}

%c Subsection text styling
\titlecontents{lsubsection}[.5em] % Indentation
{\normalfont\footnotesize\sffamily} % Font settings
{}
{}
{}

%----------------------------------------------------------------------------------------
%	PAGE HEADERS
%----------------------------------------------------------------------------------------

\usepackage{fancyhdr} % Required for header and footer configuration



\pagestyle{fancy}
\renewcommand{\chaptermark}[1]{\markboth{\sffamily\normalsize\bfseries\chaptername\ \thechapter.\ #1}{}} % Chapter text font settings
\renewcommand{\sectionmark}[1]{\markright{\sffamily\normalsize\thesection\hspace{5pt}#1}{}} % Section text font settings
%\fancyhf{} \fancyhead[LE,RO]{\sffamily\normalsize\thepage} % Font setting for the page number in the header
\fancyhead[LO]{\rightmark} % Print the nearest section name on the left side of odd pages
%\fancyhead[RE]{\leftmark} % Print the current chapter name on the right side of even pages
\renewcommand{\headrulewidth}{0.5pt} % Width of the rule under the header
\addtolength{\headheight}{2.5pt} % Increase the spacing around the header slightly
\renewcommand{\footrulewidth}{0pt} % Removes the rule in the footer
\fancypagestyle{plain}{\fancyhead{}\renewcommand{\headrulewidth}{0pt}} % Style for when a plain pagestyle is specified

% Removes the header from odd empty pages at the end of chapters
\makeatletter
\renewcommand{\cleardoublepage}{
\clearpage\ifodd\c@page\else
\hbox{}
\vspace*{\fill}
\thispagestyle{empty}
\newpage
\fi}

%----------------------------------------------------------------------------------------
%	SECTION NUMBERING IN THE MARGIN
%----------------------------------------------------------------------------------------

\makeatletter
\renewcommand{\@seccntformat}[1]{\llap{\textcolor{ocre}{\csname the#1\endcsname}\hspace{1em}}}                    

\renewcommand{\section}{\@startsection{section}{1}{\z@}
{-4ex \@plus -1ex \@minus -.4ex}
{1ex \@plus.2ex }
{\normalfont\large\sffamily\bfseries}}
\renewcommand{\subsection}{\@startsection {subsection}{2}{\z@}
{-3ex \@plus -0.1ex \@minus -.4ex}
{0.5ex \@plus.2ex }
{\color[rgb]{0.141,0.596,0.749}\normalfont\sffamily\bfseries}}
\renewcommand{\subsubsection}{\@startsection {subsubsection}{3}{\z@}
{-2ex \@plus -0.1ex \@minus -.2ex}
{.2ex \@plus.2ex }
{\color[rgb]{0.141,0.596,0.749}\normalfont\small\sffamily\bfseries}}                        
\renewcommand\paragraph{\@startsection{paragraph}{4}{\z@}
{-2ex \@plus-.2ex \@minus .2ex}
{.1ex}
{\normalfont\small\sffamily\bfseries}}


\newtheoremstyle{esercizio}
{}                % Space above
{}                % Space below
{\upshape\small}        % Theorem body font % (default is "\upshape")
{}                % Indent amount
{\bfseries}       % Theorem head font % (default is \mdseries)
{.}               % Punctuation after theorem head % default: no punctuation
{ }               % Space after theorem head
{}    
\theoremstyle{esercizio}
\newtheorem{esercizio}{}

\newsavebox{\boxa}
\savebox{\boxa}[12pt][c]{\fbox{A}}
\newcommand{\boxA}{\usebox{\boxa}}

\newsavebox{\boxb}
\savebox{\boxb}[12pt][c]{\fbox{B}}
\newcommand{\boxB}{\usebox{\boxb}}

\newsavebox{\boxc}
\savebox{\boxc}[12pt][c]{\fbox{C}}
\newcommand{\boxC}{\usebox{\boxc}}

\newsavebox{\boxd}
\savebox{\boxd}[12pt][c]{\fbox{D}}
\newcommand{\boxD}{\usebox{\boxd}}


\newenvironment{elenco}{\begin{enumerate}[label=\bfseries\alph*.]}{\end{enumerate}}

%\chapterfont{\color{red!55!black}}
%\sectionfont{\color{red!55!black}}
\chapterfont{\color[rgb]{0.141,0.596,0.749}
}
\sectionfont{\color[rgb]{0.141,0.596,0.749}
}





%%AMBIENTE ESEMBIO BEGIN

\def\exampletext{Esempio}

\NewDocumentEnvironment{testexample}{ O{} }
{
	\colorlet{colexam}{rgb:red,0.141;green,0.596;blue,0.749}
	\newtcolorbox[use counter=testexample]{testexamplebox}{%
		% Example Frame Start
		empty,% Empty previously set parameters
		title={\exampletext: #1},
		attach boxed title to top left,
		minipage boxed title,
		boxed title style={empty,size=minimal,toprule=0pt,top=4pt,left=3mm,overlay={}},
		coltitle=colexam,fonttitle=\bfseries,fontupper=\sffamily\small,
		before=\par\medskip\noindent,parbox=false,boxsep=0pt,left=3mm,right=0mm,top=2pt,breakable,pad at break=0mm,
		before upper=\csname @totalleftmargin\endcsname0pt, % Use instead of parbox=true. This ensures parskip is inherited by box.
		% Handles box when it exists on one page only
		overlay unbroken={\draw[colexam,line width=2.5pt] ([xshift=-0pt]title.north west) -- ([xshift=-0pt]frame.south west); },
		% Handles multipage box: first page
		overlay first={\draw[colexam,line width=2.5pt] ([xshift=-0pt]title.north west) -- ([xshift=-0pt]frame.south west); },
		% Handles multipage box: middle page
		overlay middle={\draw[colexam,line width=2.5pt] ([xshift=-0pt]frame.north west) -- ([xshift=-0pt]frame.south west); },
		% Handles multipage box: last page
		overlay last={\draw[colexam,line width=2.5pt] ([xshift=-0pt]frame.north west) -- ([xshift=-0pt]frame.south west); },%
	}
	\begin{testexamplebox}}
	{\end{testexamplebox}\endlist}

%%AMBIENTE ESEMPIO END











\title{Laboratorio di fisica}
\author{Prof. Romano}

% Definizione dei colori per i listati
\definecolor{mybackgroundcolor}{rgb}{0.95, 0.95, 0.95}
\definecolor{mykeywordcolor}{rgb}{0.9, 0.0, 0.0}

% Configurazione dell'ambiente lstlisting
\lstset{ 
    backgroundcolor=\color{mybackgroundcolor}, % Colore di sfondo
    basicstyle=\footnotesize\ttfamily, % Dimensione del font e tipo
    keywordstyle=\color{mykeywordcolor}, % Colore delle parole chiave
    commentstyle=\color{gray}, % Colore dei commenti
    stringstyle=\color{red}, % Colore delle stringhe
    frame=single, % Bordo attorno al listato
    breaklines=true, % Ritorno a capo automatico
    captionpos=b % Posizione della didascalia
}




\begin{document}
\maketitle
\tableofcontents
\chapter*{Prefazione}
In queste brevi note ho riassunto I principali concetti sulle grandezze fisiche, argomento propodeutico a tutto il corso. Lo stile usato è discorsivo, per approfondire i concetti fare riferimento al libro poichè in esso si trova una trattazione più sistematica (ma anche ridondante a mio parere). Buona lettura!
\chapter{Misura di grandezze}
\section{Introduzione}
La fisica è una scienza sperimentale di tipo \textbf{quantitativo}, ossia una scienza che studia la natura attraverso l'uso della matematica. Le sue affermazioni sono sempre traducibili in equazioni. Cosa sia un'equazione lo imparerete presto. Da un punto di vista matematico, si tratta di una relazione (spesso algebrica, ma ce ne sono di più complesse) che lega tra loro i valori misurati di una o più grandezze fisiche. La fisica è un insieme di \textbf{teorie}. Una teoria, detta in modo semplice, è un modello della realtà. Un modello è una sorta di riproduzione semplificata della realtà. Per capire tutto ciò però, dobbiamo anzitutto cominciare ad allenare il cervello sui concetti matematici fondamentali.
\section{Formule e formule inverse}

In fisica, lo scopo finale è comprendere come le grandezze fisiche  sono legate tra di loro. Approfondiremo il concetto di grandezza più avanti. Per il momento, pensate che una grandezza può essere una proprietà di un oggetto (ad esempio la sua massa) espressa da un numero ed indicata da una lettera. La parte della matematica che studia le formule (dette più correttamente \textbf{equazioni}) si chiama algebra. L'algebra, al pari dell'aritmetica, tratta di numeri, solo che questi sono rappresentati da lettere. In fisica, le lettere rappresentano grandezze e sono costituite da un numero che esprime una misura fatta rispetto ad una certa unità. Ad esempio:
\[
m=\SI{30}{kg}
\]

indica il valore di una massa. In questo paragrafo, vedremo come si estrae una varibile incognita da una formula (le cosiddette formule inverse). Non ci interessa chiarire il significato dei simboli ma piuttosto mostrare come si ricava una incognita usando due principi matematici (chiamati principi di equivalenza). In quanto segue, darò per scontata la conoscenza delle frazioni e di questi principi. Quì voglio solo mostrarvi con un serie di esempi, come si ricava una formula inversa. Partiamo:

	
	
	\[F= ma, \,\,  a = \text{?} \rightarrow \frac{F}{m} =\frac{\cancel{m} a}{\cancel{m}}\rightarrow a=\frac{F}{m}
	\]

 Abbiamo diviso entrambi i lati dell'equazione (chiamati membri) per uno stesso numero (la massa $m$) in modo da lasciare la nostra incognita (la lettera $a$) isolata al numeratore. La formula quindi, ci permette di calcolare $a$ conoscendo $F$ ed $m$.

\[
d=\frac{m}{V} \,\,\text{,}\,\,\,  m=\text{?} \rightarrow  V\cdot  d = \frac{m}{\cancel{V}}\cdot \cancel{V}.\rightarrow m = d V.
\]

Abbiamo ricavato la formula per la massa, noti densità e volume.

\[
F=k\cdot \frac{q_1 q_2}{r^2} \,\,\text{,}\,\,\,  r=\text{?} \rightarrow r^2 = k\cdot \frac{q_1 q_2}{F}\rightarrow r=\sqrt{k\cdot \frac{q_1 q_2}{F}}
\]
In questo caso abbiamo applicato la regola secondo cui, spostando una lettera che è moltiplicata da un lato all'altro dell'uguaglianza, se è denominatore, diventa numeratore e viceversa ($r^2$ è diventato numeratore ed $F$ denominatore). In fine, per eliminare il quadrato, abbiamo estratto la radice quadrata, poichè è l'operazione inversa del quadrato. Provate a calcolare la radice quadrata di $2^2$ con la calcolatrice e vedrete che otterrete 2!.
\[
s= s_0 + v t \,\,\text{,}\,\,\,  t=\text{?} \rightarrow s-s_0 = s_0 -s_0 + v t\rightarrow \frac{s-s_0}{v} = \frac{\cancel{v} t }{\cancel{v}}\rightarrow t=\frac{s-s_0}{v}
\]
In questo caso, per ricavare $t$, abbiamo aggiunto ai due membri il termine $-s_0$ in modo da lasciare il termine $v t$ e solo dopo abbiamo diviso per $v$.


\[
V=\pi r^2 h \,\,\text{,}\,\,\,  h=\text{?} \rightarrow \frac{V}{\pi r^2} = \frac{\cancel{ \pi r^2} h}{\cancel{\pi r^2}} \rightarrow h= \frac{V}{\pi r^2 }
\]
Abbiamo ricavato l'altezza del cilindro conoscendo raggio e volume. Nella prossima equazione, vogliamo calcolare il diametro note altezza e volume.
\[
V=\pi r^2 h =\pi \left(\frac{D}{2} \right)^2 h= \pi \frac{D^2}{4}  h\,\,\text{,}\,\,\,  D=\text{?} 
\]
Procediamo in analogia ai casi precedenti:
\[
\frac{4 V}{\pi h} = \frac{ \cancel{ 4 \pi h }D^2 }{ \cancel{4 \pi h}} \rightarrow D=\sqrt{\frac{4 V}{\pi h}}
\]




\section{Sistema internazionale S.I.}

Diamo la seguente definizione:
\begin{grf}
	Una grandezza fisica è una proprietà di un corpo che si può misurare oggettivamente (nel senso che chiunque la misuri deve ottenere lo stesso risultato).
\end{grf}

Per misurare una grandezza occorre confrontare la grandezza con l'unità di misura:
\begin{udm}
    L'unità di misura di una grandezza è un campione di riferimento dello stesso tipo della grandezza.
\end{udm}
Se vogliamo misure una massa ad esempio, dovremo usare come unità di misura il chilogrammo, se dobbiamo misurare
una superficie, useremo il metro quadro, e così via. Abbiamo citato gli strumenti di misura e quindi li definiamo:
\begin{sdm}
Uno strumento di misura è un oggetto su cui è riportata un'unità di misura.
\end{sdm}

 Consideriamo la prima grandezza che incontreremo nello studio della meccanica, la lunghezza. Per definire una grandezza fisica, occorre darne una definizione operativa, ossia definire \textit{come si misura}, fornendo un procedimento e una unità di misura. Nel caso della lunghezza, si utilizza un righello (o un metro da falegname) che è un oggetto rettilineo sul quale sono indicate delle tacche. Le tacche riportano l'unità di misura ripetura tante volte. Ad intervalli regolari, la scala così costruita, contiene tacche più grandi utili per calcolare la cosiddetta sensibilità dello strumento ossia, \textbf{la più piccola variazione di una grandezza che uno strumento può rilevare}. Nel caso della  figura \ref{fig:lun} la misura della lunghezza del lato del tavolo, si scrivedrà:
 \[
 L=\SI{0,763}{\meter} =\SI{763}{\milli\meter}
 \]
 
    \begin{figure}[h!]
    \centering
    \includegraphics[width=0.5\linewidth]{path_to_image/lun.jpg} 
    \caption{Misura di lunghezza col metro da sarto}
    \label{fig:lun}
\end{figure}  
 
Questa misura, indica chiaramente che abbiamo un ``errore'' di un millimetro (vedremo più avanti cosa sono gli errori) ossia, siamo certi che la nostra misura è compresa nel seguente intervallo:
\[
\SI{762}{\milli\meter} \le L \le \SI{764}{\milli\meter}
\]


Uno strumento di misura è uno strumento tarato, ossia uno strumento che ci fornisce direttamente  il valore di  una misura rispetto ad una certa unità di misura. Ad esempio, lo strumento che abbiamo usato per misurare il lato del tavolo, si chiama "metro da falegname" ed ha tacche distanti tra loro 1 mm. Questo strumento misura la lunghezza, ma esistono altre grandezze fisiche e quindi altri strumenti. Nella tabella \ref{tab:si_units} elenchiamo le grandezze fondamentali del cosiddetto sistema internazionale (abbreviato S.I.) un insieme di 7 grandezze che un comitato tecnico scientifico ha approvato in modo da semplificare ad esempio i commerci e la comunicazione scientifica poiché al mondo, in ambiti e paesi diversi, erano e sono tutt'ora in uso, unità diverse (nei paesi anglosassoni ad esempio, sopravvivono unità come pollice, piede, oncia, miglia, gallone etc.). Per questo corso ci serviranno il primo anno solo le prime quattro (lunghezza, massa, tempo, temperatura). Per quanto riguarda la temperatura, useremo quest'anno il grado centigrado, simbolo °C. 

\begin{table}[h!]
	\centering
	\begin{tabular}{|c|c|c|c|}
		\hline
		\textbf{Grandezza} & \textbf{Simb.} & \textbf{Unità (nome)} & \textbf{Strumento} \\
		\hline
		Lunghezza & l & m (metro) & Metro, righello \\
		\hline
		Massa & m & kg (chilogrammo) & Bilancia \\
		\hline
		Tempo & t & s (secondo) & Cronometro \\
		\hline
		Corr. elettrica & I & A (ampere) & Amperometro \\
		\hline
		Temp. termodin. & T & K (kelvin) & Termometro \\
		\hline
		Quant. sostanza & n & mol (mole) & Cont. particelle \\
		\hline
		Int. luminosa & $I_v$ & cd (candela) & Fotometro \\
		\hline
	\end{tabular}
	\caption{Grandezze fondamentali del SI.}
	\label{tab:si_units}
\end{table}

Come avrete notato, il valore di una grandezza si scrive usando una lettera, nel nostro caso la L. Non bisogna confondere le lettere usate per la misura con le lettere usate per l'unità di misura. Ad esempio nella scrittura $m=\SI{4}{\kilo\gram}$, emme è il simbolo della grandezza. Nella scrittura $L=\SI{10}{\meter}$, la emme significa "metro" ed è l'unità di misura.

\section{Sensibilità e scrittura formale}

Ricordando l'esempio del tavolo, possiamo riassumere l'importante concetto di sensibilità di uno strumento di misura:

\begin{sen}
	La sensibilità di uno strumento di misura, è la più piccola variazione della grandezza fisica che lo strumento può misurare.
\end{sen}

Si badi alla parola ``variazione''. Se ho un termometro le cui divisioni hanno una differenza tra loro di un grado ma il termometro ha una temperatura minima di -20 gradi, la sensibilità sarà di un grado.
Una volta definite le grandezze fisiche, bisogna prendere coscienza del fatto che la misura di una grandezza va sempre indicata come segue:

\[
x=\left( \overline{x} \pm \Delta x\right)
\]
dove :
\begin{itemize}
\item $x$ rappresenta il nome della grandezza;
\item $\overline{x}$ rappresenta il valore misurato (direttamente o indirettamente, come vedremo). Talvolta esso viene chiamato il miglior valore. Questa espressione non significa che si tratti di uno dei tanti misurati ma si tratta di un valore che rappresenta meglio di altri il risultato della misura. Tale valore può essere una misura unica, una media o il risultato di calcoli;
\item $\Delta x$ rappresenta l'incertezza di misura o \textbf{errore assoluto}. Questa incertezza può avere origine dallo strumento (in tal caso si usa la sensibilità)oppure, se la misura è indiretta, essere causata da un calcolo. In tal caso si parla di incertezza \textit{propagata}. Faremo solo alcuni cenni all'incertezza propagata. In generale, per noi l'incertezza $\Delta x$ sarà quel numero positivo tale che il risultato della misura, $\overline{x}$, sia compreso tra $\overline{x} -\Delta x$ e  $\overline{x} +\Delta x$
\end{itemize}

Misurando la lunghezza dell'altro lato del tavolo, avremmo potuto ottenere:
\[
L=\left(1,200 \pm 0,001\right)\si{\meter}  \,\, \text{abbiamo usato il metro come unità di misura.}
\]

Volendo usare la scrittura $L=\left(1,2 \pm 0,001\right)\si{\meter}$, sbaglieremmo, ed anche se scrivessimo $L=\left(1,2 \pm \SI{1}{\milli\meter}\right)$ o, ancora, $L=\left(1,200 \pm 1\right)\si{\meter}$ perché vorrebbe dire che la sensibilità (ossia la distanza tra due tacche) è un metro!

Ricordiamo quindi che \textbf{la misura e l'errore assoluto, devono avere lo stesso numero di cifre decimali} (da non confondere con la cifre significative su cui ci soffermeremo più avanti).
\section{Proprietà degli strumenti}
Gli strumenti di misura appartengono a due grandi categorie: gli strumenti analogici (dotati di una scala e a volte di un indice mobile, come negli orologi o i cronometri a lancetta) e digitali. Uno strumento digitale ha un display elettronico sul quale leggiamo direttamente la misura. In laboratorio di elettronica userete principalmente strumenti digitali i quali hanno grandi vantaggi rispetto a quelli analogici. Tuttavia è bene osservare che spesso tali strumenti, danno una certa sicurezza in quanto risparmiano la fatica di leggere una scala e  forniscono immediatamente il risultato. Comunque, come tutti gli strumenti di misura, anch'essi hanno dei limiti di funzionamento, possono cioè danneggiarsi se usati male, e, inoltre, avendo molte funzioni, possono avere una curva di apprendimento molto ripida.  Entrambi i tipi di strumenti hanno quattro importanti caratteristiche: \textbf{sensibilità}, \textbf{portata massima (e minima)}, \textbf{prontezza}, \textbf{precisione}. 
\begin{description}
\item[Sensibilità] Si tratta, come abbiamo già detto, della minima \textit{variazione} della grandezza che può essere rilevata dallo strumento. Essa, nelle misure dirette, rappresenta anche l'incertezza di misura (o errore assoluto). Per determinare la sensibilità di uno strumento analogico si osserva la sua scala. Nel caso di un righello o di un cilindro graduato, basta leggere due valori scritti e contare le tacche fra di essi. La sensibilità è data dalla differenza tra i due valori, divisa per il numero di tacche. Per strumenti più sofisticati, tipo un calibro o una bilancia elettronica, di solito la sensibilità è indicata sullo strumento stesso. In laboratorio abbiamo in dotazione vari tipi di calibro con sensibilità di $\SI{0,05}{mm}$, $\SI{0,1}{mm}$ e persino $\SI{0,01}{mm}$ (calibro Palmer).
\item[Portata massima] Si tratta del massimo valore della grandezza che uno strumento può rilevare. Essa è particolarmente importante ai fini del corretto uso dello strumento. Se si tenta di leggere un valore più grande della portata, lo strumento rischia di rompersi. A volte si parla anche di \textit{campo} di misura. Il campo è, più in generale, l'intervallo di valori che uno strumento può leggere. Se abbiamo un termometro che legge da $\SI{-20}{\celsius}$ a $\SI{100}{\celsius}$, allora il campo è $\SI{-20}{\celsius}\div\SI{100}{\celsius}$. 


\item[Precisione] La precisione è definita come il rapporto tra sensibilità e portata. In quanto tale non ha unità di misura. Sui testi si trova una diversa definizione di precisione. Si dice che uno strumento è più preciso di un altro se, ripetendo la misura di una grandezza costante, si ottengono sempre gli stessi valori oppure valori più vicini. Noi preferiamo definire questa come l'affidabilità di uno strumento.
\item[Prontezza] Si tratta della velocità con cui lo strumento ci restituisce il valore misurato. Se uno strumento è più rapido di un altro nel fornire una lettura, si dice che è più \textit{pronto}.

E' importante osservare che, potendo scegliere, è sempre bene preferire lo strumento con sensibilità piccola ma non sempre è possibile. Se il nostro strumento ha una portata non adatta, potremmo romperlo. Si pensi al caso di una bilancia di precisione. Queste bilance, a volte hanno portate di poco più di un kilo e potrebbero rompersi se ci caricassimo più peso (oppure segnerebbero errore (se digitali)). Riprenderemo questo concetto più avanti dopo una lunga parentesi matematica..
\end{description}


\section{Equivalenze e notazione scientifica}

Misurare, vuol dire eseguire un confronto. Se scrivo:
\[
L=\SI{10}{\meter}
\]
significa che $\frac{L}{m} = 10$, ossia l'unità di misura è contenuta dieci volte nella misura della grandezza. Dobbiamo quindi conoscere le frazioni. Nella pratica, le grandezze fondamentali a volte sono scomode da usare. In luogo del metro ad esempio, nella meccanica di precisione si usa il millimetro, scritto mm ( 1 m = 1000 mm, si legge ``un metro è uguale a mille millimetri''). Si dice che il millimetro è un sottomultiplo del metro. Analogamente, il kilometro (simbolo km) è un multiplo del metro (1 km = 1000 m). Nella tabella \ref{tab:prefixes}indichiamo i prefissi del sistema metrico decimale. 

\begin{table}[h!]
	\centering
	\begin{tabular}{|c|c|c|}
		\hline
		\textbf{Nome prefisso} & \textbf{Simbolo prefisso} & \textbf{Significato} \\
		\hline
		Giga & G & \( \times 10^9 \) \\
		\hline
		Mega & M & \( \times 10^6 \) \\
		\hline
		Kilo & k & \( \times 10^3 \) \\
		\hline
		Etto & h & \( \times 10^2 \) \\
		\hline
		Deca & da & \( \times 10^1 \) \\
		\hline
		Deci & d & \( \frac{1}{10} \) \\
		\hline
		Centi & c & \( \frac{1}{100} \) \\
		\hline
		Milli & m & \( \frac{1}{1000} \) \\
		\hline
		Micro & \(\mu\) & \( \frac{1}{10^6} \) \\
		\hline
		Nano & n & \( \frac{1}{10^9} \) \\
		\hline
		Pico & p & \( \frac{1}{10^{12}} \) \\
		\hline
	\end{tabular}
	\caption{Prefissi da Giga a Pico con simboli e significato}
	\label{tab:prefixes}
\end{table}

\begin{testexample}[Qualche equivalenza]
$\SI{0,48}{\meter} = \cdots \si{\centi\meter}$. Dobbiamo scrivere $\SI{1}{\meter} = \SI{100}{\centi\meter}$. Ok, e poi? $m=\SI{100}{\meter} \rightarrow \SI{0,48}{\meter} = 0,48\times\left(\SI{100}{\centi\meter}\right) = \SI{48}{\centi\meter}$ (abbiamo spostato la virgola). Come abbiamo fatto a sapere che  $\SI{1}{\meter} = \SI{100}{\centi\meter}$? Semplice, abbiamo usato una formula inversa! Dalla tabella vediamo che 
\[
\SI{1}{\centi\meter} = \frac{1}{100}\si{\meter}
\]
Vogliamo ricavare $m$:
\[
\textcolor{red}{100}\cdot \SI{1}{\centi\meter} = \frac{1}{\cancel{100}}\cdot m \cdot \textcolor{red}{\cancel{100}} \rightarrow \SI{100}{\centi\meter}  = \SI{1}{\meter}.
\]
Magico, vero? Facciamo un altro esempio.\\

$\SI{27}{\milli\meter} = \cdots \si{\deca\meter}$. Questa è più difficile perché so che $ \SI{1}{\milli\meter}  = \frac{1}{1000} \si{\meter}$. Però, ricordiamo che $ \SI{1}{\meter}  = \frac
{\SI{1}{\deca\meter}}{10}$ (provatelo), dunque $ \SI{27}{\milli\meter}  = 27\times \frac
{1}{1000}\times\frac{\SI{1}{\deca\meter}}{10} =\frac{27}{10000}\si{\deca\meter} =\SI{0,027}{\deca\meter}$. 

Questi numeri, numeri con molti zeri, sono scomodi da scrivere. Per sperimentare una notazione (ossia un modo di scrittura) più comodo per tali numeri, dobbiamo ripassare alcuni concetti sulle potenze.
\end{testexample}

\subsection{Potenze di 10}
Ricordiamo che $10^5=10\times 10 \times 10\times 10\times10$. Esistono anche le potenze negative, ad esempio $10^{-2}$. Ma non possiamo moltiplicare un numero per ``meno due volte'' per sè stesso. In realtà, questa è una frazione che si scrive in forma decimale (sappiamo passare dai numeri alle frazioni, vero??):
\[
10^{-2}=\frac{1}{100} = 0,01
\]
In matematica imparerete anche a calcolare espressioni del tipo $\left(\frac{1}{3}\right)^{-2} =\left(\frac{3}{1}\right)^{2} = 9$ ma a noi non interessa così tanto l'algebra. poiché calcoleremo tutto usando la calcolatrice. Dobbiamo però capire il significato di certe scritture. Anzitutto dobbiamo capire che $10^{-2}$ \underline{non è negativo}, è solo un numero positivo minore di uno. Ovviamente, $10^{-2}=$ \underline{è} negativo e vale $-0,01$. -$2$ si chiama esponente  e $10$ la base. Le potenze di 10 godono di alcune proprietà che è bene conoscere. Moltiplicando due potenze \textbf{della stessa base} si ottiene una sola potenza che ha per esponente, \textbf{la somma degli esponenti}:
\[
10^2\times 10^3 = 10^5
\]
Vale anche con potenze negative:
\[
10^3\times10^{-5} = 10^{-2} = 0,01.
\]
Si possono ovviamente calcolare espressioni senza grande fatica, ad esempio:
\[
\left( 2 \cdot 10^2 \right) \times \left( 3 \cdot 10^3 \right) = \left( 3 \cdot 2 \right) \cdot 10^{(2+3)} = 6 \cdot 10^5
\]
A noi comunque, interessa il modo con cui è scritto il risultato, ossia  in notazione scientifica. Ne parleremo nella prossima sezione. Per concludere questi cenni, ricordiamo alcune altre proprietà delle potenze che ritroveremo. Il rapporto tra due potenze, si calcola usando la differenza di esponenti
\[
\frac{10^4}{10^2}=10^{4-2} = 10^2
\]
Vale anche se la seconda potenza è negativa:
\[
\frac{10^2}{10^{-5}}=10^{2-(-5)} = 10^{2+5} = 10^7.
\]
Ricordiamo ancora le identià (si assumono, non si dimostrano): $$a^0 =1 $$ (qualunque a diverso da zero)  ; $$a^1 = a$$ (qualunqiue sia il numero a).  
\subsection{Notazione scientifica}
Il numero $6 \cdot 10^5$ è scritto in notazione scientifica. Definiamo questo modo di scrivere i numeri:
\begin{nsc}
	Un numero è scritto in notazione scientifica se è scritto come il prodotto di un numero decimale compreso tra 1 e 9 (esclusi 0 e 10) e una potenza di 10 (positiva) o negativa.
\end{nsc}
Secondo questa definizione, $24 \cdot 10^5$ non è scritto in notazione scientifica perché 24 è maggiore di 9. Nemmeno $0,2\times10^3$ è scritto in notazione scientifica perché 0,2 è minore di 1. Ogni numero si può scrivere in notazione scientifica usando le proprietà delle potenze e la rappresentazione decimale. Ad esempio: $24 \cdot 10^5 = (2,4\cdot10)\cdot10^5 = 2,4\times 10^{1+5}=2,4\cdot 10^6$. Abbiamo usato il fatto che $10^1 = 10$. Ricordiamo inoltre che $10^0 = 1$  ma che non ha senso calcolare $0^0$. Una proprietà che si rivelerà utile quando eseguiremo le equivalenze coi metri quadri, è la potenza di potenza. Vediamo come si applica:
\[
\left(10^2\right)^3 = 10^6
\]
ossia,  \textbf{si moltiplicano gli esponenti}. In fine, un esempio con potenze con segno negativo:
\[
\left(10^2\right)^{-3} = 10^{-6}
\]
Quando si lavora con le potenze, bisogna stare molto attenti a quello che si fa. Nel caso di questi due esempi, sarebbe sbagliato sommare gli esponenti, perché questo si fa solo quando si moltiplicano due potenze (quando si moltiplica ad esempio $10^2$ per $10^3$,  il risultato fa $10^5$ e non $10^6$). Notare poi come tutto torna, ad esempio $10^0\cdot 10^1 = 10^{0+1} =10^1 = 10$ il che è ovvio perché $10^0$ è uno e qualunque numero moltiplicato uno fa sempre uno. Queste considerazioni, valgono anche per espressioni contenenti frazioni, come questa:
\[
\left(\frac{1}{2}\right)^2\times \left(\frac{1}{2}\right)^3 \times\left(\frac{1}{2}\right)^{-6}= \left(\frac{1}{2}\right)^{2+3-6} = \left(\frac{1}{2}\right)^{-1} =2.
\]
ma ci fermiamo quì.


\begin{testexample}[ \thetcbcounter \, Scrittura corretta]
Supponiamo di avere misurato la quantità
\[
L=\left(1240\pm1\right)\si{\milli\meter}
\]
Voglio esprimere questa misura in metri (unità del sistema internazionale) in modo compatto. Potrei dividere tutto per mille e scrivere:
\[
L=\left(1,240\pm 0,001\right)\si{\meter}
\]

Tuttavia, usando le potenze di 10 abbiamo:
\[
L=\left(1240\pm 1\right)\cdot 10^{-3}\si{\meter}
\]
In questo modo, non dobbiamo cambiare la quantità in parentesi. Notiamo inoltre, che anche l'errore è moltiplicato per la potenza. La scrittura, è un modo breve di scrivere in realtà.

\[
L= 1240\cdot 10^{-3}\si{\meter} \pm 1\cdot 10^{-3}\si{\meter}
\]
Questa informazione è importante quando dobbiamo trasformare (non ci capiterà spesso) una misura di temperatura da Celsius a Kelvin e viceversa, ma questo lo vedremo dopo aver fatto qualche osservazione in più sugli strumenti di misura.

Proviamo a scrivere nel sistema internazionale, la misura di massa:
\[
m=\left(240,0\pm 0,1\right)\si{\gram}
\]
Dobbiamo trasformare tutto (misura ed errore) in kilogrammi, in quanto questi rappresentano l'unità di massa (base) nel sistema internazionale. Sappiamo che $\SI{1}{\kilo\gram}=\SI{e3}{\gram}$. Ricaviamo la formula inversa:
\[
\frac{\SI{1}{\kilo\gram}}{10^3} = \frac{\textcolor{red}{\cancel{10^3}}\,\si{\gram}}{\textcolor{red}{\cancel{10^3}}} \rightarrow \SI{1}{\gram} = \frac{\SI{1}{\kilo\gram}}{10^3} = \SI{1}{\kilo\gram}\cdot 10^{-3}. 
\]
dunque,
\[
m=\left(240,0\pm 0,1\right)\,\si{\gram} = \left(240,0\pm 0,1\right)\cdot 10^{-3}\,\si{\kilo\gram}
\]
Comode le potenze negative!


\end{testexample}

\section{Alcuni esempi sulla sensibilità e la portata}
Quando dobbiamo valutare la sensibilità di uno strumento digitale (con display per intenderci) possiamo cercare questa informazione che dovrebbe essere riportata da qualche parte, insieme alla portata, sullo strumento stesso. Nelle nostre bilance di laboratorio è proprio così (figura~\ref{fig:bilancia}). Se non è scritto, la sensibilità è una unità sull'ultima posizione decimale che comprare sul display. Nel caso della bilancia in figura, vediamo che quando è scarica, sul display c'è scritto 0,0 g e capiamo che la sensibilità è di 0,1 g. Per la portata è più complicato in certi casi. 

\begin{testexample}[\thetcbcounter \, Bilancia digitale]
	
\begin{minipage}{\linewidth}
	\centering
	\includegraphics[scale=0.14]{bilancia}
	\captionof{figure}{Bilancia digitale}
	\label{fig:bilancia}
\end{minipage}
	
	
\end{testexample}
\begin{testexample}[\thetcbcounter \,Cronometro analogico]
Guardando il cronometro analogico in figura \ref{fig:cron}, vediamo che il valore massimo del tempo, si ottiene quando la lancetta interna (che và in 30 secondi dal bianco al rosso) arriva su 15 minuti, dunque la sensibilità è di 0,1 s e la portata da massima di 15 min = 900 s. La misura che vediamo in foto, corrisponde ad un tempo di 3 minuti e 5,6 secondi, ossia $3\times 60 \,\si{\second} +\SI{5,6}{\second}= \SI{185,6}{\second}$ quindi la scrittura formale (o corretta) della misura sarà:
\[
t=\left(185,6\pm 0,1\right)\,\si{\second}
\]

\begin{minipage}{\linewidth}
	\centering
	\includegraphics[scale=0.2]{cronometro}
	\captionof{figure}{Cronometro analogico}
	\label{fig:cron}
\end{minipage}

\end{testexample}

\begin{testexample}[\thetcbcounter \,Cilindro graduato]
In chimica e in fisica, spesso si misurano volumi di liquidi usando un cilindro graduato. Guardando la figura \ref{fig:cil}, possiamo calcolare la sensibilità del cilindro, infatti, tra 70 mL e 80 mL ci sono 10 tacche, quindi la sensibilità vale:
\[
s = \frac{(80 - 70),\si{\milli\liter}}{10,\text{Div}} = \SI{1}{\milli\liter} 
\]

Per quanto riguarda invece il campo di misura, i valori sono: minimo $\SI{10}{\milli\liter}$ e massimo $\SI{100}{\milli\liter}$. In fine, la misura del liquido contenuto è:

\[
V=\left(60 \pm 1\right)\,\si{\milli\liter}
\]

E' bene comunque notare che non è sempre così semplice. In laboratorio, abbiamo cilindri con varie sensibilità. Se la sensibilità è ad esempio 5 mL e il liquido sale due tacche sopra i 200 mL, allora la nostra misura sarà: $$V=\left(210 \pm 5 \right) \, \si{\milli\liter}$$.


\begin{minipage}{\linewidth}
	\centering
    \includegraphics[scale=0.35]{cil.jpg}
   \captionof{figure}{Cilindro graduato}
	\label{fig:cil}
\end{minipage}
\end{testexample}


\begin{testexample}[\thetcbcounter \,Cordella metrica]
	Lo strumento nella figura \ref{fig:cordella1} è una cordella metrica, usata per misurare lunghezze di alcuni metri. Guardando la figura, possiamo cercare di valutarne la sensibilità anche se sulla scala ci sono pochi numeri. Vediamo scritto 10 ma ovviamente non può trattarsi di millimetri, altrimenti 10 mm sarebbero un cm! Dunque sono 10 cm e, siccome tra 0 e 10 ci sono 10 divisioni, la sensibilità è di 1 cm. La portata è scritta sulla cordella ed è di 20 m (dovremmo usare la sensibilità come unità di misura anche per indicare la portata e quindi la portata si dovrebbe scrivere P= $20\times 100\,\si{\centi\meter = 2000\,\si{\centi\meter}}$)
		
		\begin{minipage}{\linewidth}
			\centering
		\includegraphics[scale=0.2]{cordella1}
		\captionof{figure}{Cordella metrica da 1 cm di sensibilità}
		\label{fig:cordella1}
	\end{minipage}
Nella figura  \ref{fig:cordella2}, vediamo invece una cordella metrica diversa.	Sulla scala leggiamo 1|0 che evidentemente indica 10 cm (infatti 2|0 dista grosso modo 10 cm da tale tacca). Tra 10 e 11 ci sono 5 tacche, pertanto la sensibilità, sarà:
\[
s=\frac{\left(11-10\right)\,\si{\centi\meter}}{5\, \text{Div}} = \SI{0,2}{\centi\meter}.
\]
	
			\begin{minipage}{\linewidth}
		\centering
	\includegraphics[scale=0.2]{cordella2}
		\captionof{figure}{Cordella metrica da 0,2 cm di sensibilità}
		\label{fig:cordella2}
	\end{minipage}

mentre la portata è di	30 m.
	
\end{testexample}

Quale strumento usare quando se ne hanno vari a disposizione? Il metro da falegname è più sensibile delle cordelle metriche col suo millimetro di sensibilità ma non è indicato per misurare lunghezze maggiori della sua portata di 2 metri. Se devo pesare una persona, non potrò usare la bilancia di laboratorio con la sua portata di 1200 g!


\begin{testexample}[\thetcbcounter \,Una semplice misura di lunghezza.]
Supponiamo di dover misurare la lunghezza di un muro e di usare la cordella metrica con sensibilità di 0,2 cm. Poiché la sensibilità è di 0,2 cm, il risultato dovrà essere indicato in cm. Guardando la figura, vediamo che la tacca più vicina al bordo è la terza dopo 360 cm, quindi $L = \SI{360}{\centi\meter} + 3\times (\SI{0,2}{\centi\meter}) = \SI{360,6}{\centi\meter}$, dunque la misura corretta si scriverà:
\[
l=\left(360,6 \pm 0,2\right)\si{\centi\meter}
\]	
% DISEGNAQUI e fai la foto in laboratorio	
	\begin{tikzpicture}[x=1cm, y=1cm]

% Tacche principali e etichette
\foreach \x in {360,370} {
    \draw[thick] (\x-360,0) -- (\x-360,1); % Tacche lunghe
    \node[above] at (\x-360,1) {\pgfmathparse{int(\x/10)}\pgfmathresult|0}; % Etichette principali
}

% Tacche intermedie e etichette
\foreach \i in {1,...,9} {
    \draw (\i,0) -- (\i,0.5); % Tacche corte
    \node[above] at (\i,0.5) {\i}; % Etichette intermedie

    % Suddivisioni piccolissime
    \foreach \j in {1,...,4} {
        \draw (\i + \j*0.2,0) -- (\i + \j*0.2,0.25); % Tacche piccolissime
    }
}

% Suddivisioni tra la tacca 9 e la tacca principale a 10
\foreach \j in {1,...,4} {
    \draw (9 + \j*0.2,0) -- (9 + \j*0.2,0.25); % Tacche piccolissime
}

% Aggiungi ulteriori tacche piccole tra 0 e 1
\foreach \j in {1,...,4} {
    \draw (\j*0.2,0) -- (\j*0.2,0.25); % Tacche piccolissime
}

% Aggiungi freccia rossa verticale che indica la terza tacca dopo 360 cm
\draw[red, thick, ->] (0.6, -0.5) -- (0.6, 0.05); % Freccia rossa con punta alla y del terzo segmento

\end{tikzpicture}
\end{testexample}

\section{Cifre significative}
	Diamo una importante definizione:
\begin{csf}
	Le cifre significative di un numero, sono tutte le cifre certe e la prima incerta.
\end{csf}
Dunque, per conoscere le cifre significative di un numero, occorre conoscere l'incertezza su tale numero. L'incertezza è un concetto nuovo per voi ma abbiamo già visto un esempio di incertezza (la sensibilità nelle misure dirette). La misura $L = \SI{10,234671}{\milli\meter}$,con un errore di $\SI{1}{\milli\meter}$, avrebbe come cifre significative solo le prime 2 (1 e 0) perché 1 è certa e 0 è la prima cifra incerta, dunque non ha senso scrivere il 2, il 3 e le altre cifre più a destra, per cui scriveremo:
\[
\SI{10,234671}{\milli\meter} \approx \SI{10}{\milli\meter}
\]
Se la misura fosse stata $\SI{10,634671}{\milli\meter}$, poiché 6 è maggiore di 5, avremmo approssimato come segue:
\[
\SI{10,634671}{\milli\meter} \approx \SI{11}{\milli\meter}
\]
\subsection{Valutazione a vista delle cifre significative}
Se in un testo non ci viene data l'incertezza ma solo la misura di una grandezza, ci sono convenzioni per valutare le cifre significative. Si assume che tutti gli zeri prima della prima cifra diversa da zero, siano non significativi. Ad esempio, nel numero 0,00034040, i primi quattro zeri non sono significativi. Lo zero tra i due 4 è significativo e lo è anche lo zero finale, dunque il numero 0,00034040 ha 5 cifre significative. Se volessimo scrivere il numero con 4 cifre significative, dovremmeo prima arrotondare e poi togliere l'ultima cifra, nel nostro caso quindi, otterremmo 0,0003404 perché la prima cifra tolta è uno zero. Se una misura è scritta in notazione scientifica, si contano solo le cifre significative della parte decimale, quindi il numero $\SI{1,206e+3}{\milli\meter}$ ha 4 cifre significative. Se volessimo approssimarlo a 3 cifre, scriveremmo: $\SI{1,206e+3}{\milli\meter} \approx \SI{1,21e+3}{\milli\meter}$ perché abbiamo eliminato una cifra maggiore di 5. Se la cifra che elimiamo è 5, non c'è una regola condivisa (noi però approssimeremo sempre per difetto).



\subsection{Cifre significative nelle operazioni}

Una delle domande più frequenti degli studenti durante le verifiche, riguarda le cifre decimali nei risultati: quante cifre bisogna tenere nei calcoli? La calcolatrice scientifica, comunemente ha un display che mostra 12 cifre ma non è sempre necessario tenerle tutte. Tutto dipende dalle cifre significative nei dati. Le regole, che non dimostreremo (ma useremo), sono due:
\begin{csp}
Moltiplicando o dividendo due grandezze, il risultato và fornito con il numero di cifre significative di quella che ne ha di meno.
\end{csp}
Se ad esempio, abbiamo v=$\SI{12,6}{\meter\per\second}$ e t=$\SI{0,10}{\second}$, allora scriveremo:
\[
s=v\cdot t =  \left(\SI{12,6}{\meter\per\second} \right)\cdot\left( \SI{0,10}{\second}\right) = \SI{1,26}{\meter\per\second} \approx \SI{1,3}{\meter}. 
\]
\begin{css}
Sommando o sottraendo due grandezze, si tengono nel risultato, solo le cifre decimali ottenute sommando cifre significative.
\end{css}
Se ad esempio, abbiamo $s_1=\SI{45,26}{\meter}$ e $s_2=\SI{17,4}{\meter}$, abbiamo: $s_1 +s_2 = \SI{62,66}{\meter} \approx \SI{62,7}{\meter}$. Da notare che in questa definizione si usano le cifre decimali, mentre nella precedente le cifre significative. Queste due regole sono regole approssimate, se vogliamo una valutazione corretta delle cifre significative nelle operazioni, dobbiamo applicare le regole del prossimo capitolo.

\section{Area e volume}
L'area è definita nel sistema internazionale come la misura della superficie rispetto all'unità di riferimento (il metro quadro, simbolo $\si{\square\meter}$, figura  \ref{fig:quad}), ossia un quadrato di lato 1 m. Il volume invece è definito come la misura dello spazio occupato da un corpo rispetto all'unità di misura base, un cubo di lato un metro, il metro cubo (simbolo $\si{\cubic\meter}$, figura \ref{fig:cub})\\

\begin{minipage}{\linewidth}
	\centering
	\includegraphics[scale=0.3]{quad}
	\captionof{figure}{Il metro quadro}
	\label{fig:quad}
\end{minipage}
\begin{minipage}{\linewidth}
	\centering
	\includegraphics[scale=0.3]{cub}
	\captionof{figure}{Il metro cubo}
	\label{fig:cub}
\end{minipage}


Poichè queste due grandezze non fanno parte delle 7 grandezze fondamentali, si chiamano \textit{derivate}. I multipli di \(\si{\square\meter}\) sono quadrati con lati di lunghezza \(\SI{1}{\deca\meter}\) (\(\si{\square\deca\meter}\), decametro quadro), \(\SI{1}{\hecto\meter}\) (\(\si{\square\hecto\meter}\), ettometro quadro), \(\SI{1}{\kilo\meter}\) (\(\si{\square\kilo\meter}\), chilometro quadro), mentre i sottomultipli sono quadrati di lato \(\SI{1}{\deci\meter}\) (\(\si{\square\deci\meter}\), decimetro quadro), \(\SI{1}{\centi\meter}\) (\(\si{\square\centi\meter}\), centimetro quadro), \(\SI{1}{\milli\meter}\)\ (\(\si{\square\milli\meter}\), millimetro quadro). Analogamente per il volume: \(\si{\cubic\deca\meter}\) (decametro cubo), \(\si{\cubic\deci\meter}\) (decimetro cubo), ecc.


Le  equivalenze con questa grandezze si riducono a quelle lineari (basta contare i quadrati o i cubi contenuti) per andare dal multiplo a sottomultiplo). Per andare da sottomultipli a  multipli, basta dividere per un'opportuna potenza di 10. Vediamo un esempio: \,$\SI{1,5}{\cubic\meter} = \cdots \si{\cubic\centi\meter}$. Un cubo di lato $\SI{1}{\cubic\meter}$ contiene su ogni lato, 100 cm (oppure 10 dm, vedi  figura \ref{fig:da-m3-a-cm3}), dunque $\SI{1}{\cubic\meter} = 100\times100\times100 = \SI{e+6}{\cubic\centi\meter}$, per cui:
\[
\SI{1,5}{\cubic\meter} = \SI{1,5e+6}{\cubic\centi\meter}
\]

\begin{minipage}{\linewidth}
	\centering
	\includegraphics[scale=0.5]{da-m3-a-cm3}
	\captionof{figure}{Suddivisione di $\SI{1}{\cubic\meter}$ in cubi di lato 1 dm e 1 cm}
	\label{fig:da-m3-a-cm3}
\end{minipage}






Altro esempio: $\SI{1200}{\cubic\deci\meter} = \cdots \si{\cubic\meter}$.

Poichè 1 metro cubo contiene  $10\times10\times10 = \SI{e3}{\cubic\deci\meter}$,   e poiché   nella nostra equivalenza andiamo da sottomultiplo a multiplo, dobbiamo dividere per $10^3$:
\[
\SI{1200}{\cubic\deci\meter} = \frac{1200}{10^3}\,\si{\cubic\meter} = \SI{1,2}{\cubic\meter}.
\]

Nella pratica dei laboratori, si incontra il litro, simbolo L, definito come un decimetro cubo. Proviamo a risolvere l'equivalenza: $\SI{1}{\cubic\centi\meter} =\cdots \si{\milli\liter}$. In questi casi, conviene trasformare in litri e poi da litri a millilitri:
\[
\SI{1}{\cubic\centi\meter} = \SI{e-3}{\cubic\deci\meter}= \SI{e-3}{\liter}=10^{-3}\times\SI{e+3}{\milli\liter} =\SI{1}{\milli\liter}.
\]
Abbiamo usato la regola seguente: se tra due multipli lineari c'è una potenza di 10 $``N''$, tra i cubi c'è una potenza $3 N$. Poiché tra centimetri e decimetri c'è solo una potenza $10^1$, allora tra i cubi c'è $10^3$. Si guardi a tal proposito la figura \ref{fig:equiv2}
\vspace{0.5cm}
\begin{minipage}{\linewidth}
	\centering
	\includegraphics[scale=0.3]{equiv2}
	\captionof{figure}{Equivalenze tra volumi}
	\label{fig:equiv2}
\end{minipage}


Questa equivalenza è particolarmente importante per cui la mettiamo in edidenza:
\[
\colorboxed{ocre}{
\SI{1}{\cubic\centi\meter} = \SI{1}{\milli\liter}
}
\]
Risolviamo in fine una equivalenza con un'area: $\SI{2,34e+4}{\square\deci\meter} = \cdots \si{\square\deca\meter}$. In questo caso vale la regola del ''2'' ossia le potenze vanno moltiplicate per due:
\[
\SI{2,34e+4}{\square\deci\meter} = 2,34\times10^{4}\times 10^{-4}\,\si{\square\deca\meter} 
\]

\section{Volumi di solidi regolari}
\subsection{Cubo}
Il cubo è un solido con 6 facce quadrate congruenti.

\begin{figure}[!htbp] 
\centering
\begin{tikzpicture}[scale=0.8]
    \draw (0,0,0) -- (2,0,0) -- (2,2,0) -- (0,2,0) -- cycle;
    \draw (0,0,0) -- (0,0,2) -- (0,2,2) -- (0,2,0);
    \draw (2,0,0) -- (2,0,2) -- (2,2,2) -- (2,2,0);
    \draw (0,0,2) -- (2,0,2) -- (2,2,2) -- (0,2,2) -- cycle;
    \draw[<->] (0,-0.5,0) -- node[below] {$l$} (2,-0.5,0);
\end{tikzpicture}
\caption{Cubo con lato $l$}
\end{figure}

\textbf{Formula del volume:} $V = l^3$

\subsection{Parallelepipedo}
Il parallelepipedo è un solido con 6 facce rettangolari.

\begin{figure}[!htbp] 
\centering
\begin{tikzpicture}[scale=0.6]
    \draw (0,0,0) -- (3,0,0) -- (3,2,0) -- (0,2,0) -- cycle;
    \draw (0,0,0) -- (0,0,4) -- (0,2,4) -- (0,2,0);
    \draw (3,0,0) -- (3,0,4) -- (3,2,4) -- (3,2,0);
    \draw (0,0,4) -- (3,0,4) -- (3,2,4) -- (0,2,4) -- cycle;
    \draw[<->] (0,-0.5,0) -- node[below] {$l$} (3,-0.5,0);
    \draw[<->] (3.5,0,0) -- node[right] {$h$} (3.5,2,0);
    \draw[<->] (3,0,-0.5) -- node[below] {$p$} (3,0,3.5);
\end{tikzpicture}
\caption{Parallelepipedo con dimensioni $l$, $h$, $p$}
\end{figure}

\textbf{Formula del volume:} $V = l \cdot h \cdot p$

\subsection{Cilindro}
Il cilindro è un solido generato dalla rotazione di un rettangolo attorno a uno dei suoi lati.

\begin{figure}[!htbp] 
\centering
\begin{tikzpicture}[scale=0.7]
    % Disegna le ellissi
    \draw (0,0) ellipse (1.5cm and 0.5cm);
    \draw (0,4) ellipse (1.5cm and 0.5cm);
    % Disegna le linee verticali
    \draw (-1.5,0) -- (-1.5,4);
    \draw (1.5,0) -- (1.5,4);
    % Disegna le linee laterali
    \draw (0,0) -- (0,4);
    % Disegna le linee di misurazione
    \draw[<->] (-1.5,0) -- node[below] {$d$} (1.5,0);
    \draw[<->] (2,0) -- node[right] {$h$} (2,4);
\end{tikzpicture}
\caption{Cilindro con diametro $d$ e altezza $h$}
\end{figure}

\textbf{Formula del volume:} $V = \frac{\pi d^2}{4} \cdot h$

\subsection{Cono a base retta}
Il cono a base retta è un solido generato dalla rotazione di un triangolo rettangolo attorno a uno dei suoi cateti.

\begin{figure}[!htbp] 
\centering
\begin{tikzpicture}[scale=0.7]
    \draw (0,0) ellipse (2cm and 0.5cm);
    \draw (0,0) -- (0,4);
    \draw (2,0) -- (0,4);
    \draw (-2,0) -- (0,4);
    \draw[dashed] (0,0) -- (2,0);
\draw[<->] (-2,0) -- node[below] {$d$} (2,0);
    \draw[<->] (2.5,0) -- node[right] {$h$} (2.5,4);
\end{tikzpicture}
\caption{Cono con diametro $d$ e altezza $h$}
\end{figure}

\textbf{Formula del volume:} $V = \frac{\pi d^2}{12} \cdot h$

\subsection{Sfera}
La sfera è un solido generato dalla rotazione di un semicerchio attorno al suo diametro.

\begin{figure}[!htbp] 
\centering
\begin{tikzpicture}[scale=1]
    % Disegna la sfera con sfumatura
    \shade[ball color=blue!30] (0,0) circle (2cm);
    % Disegna l'equatore con linea continua
    \draw[dashed, thick] (0,0) ellipse (2cm and 0.6cm);

    % Disegna il contorno della sfera
    \draw[thick] (0,0) circle (2cm);

    % Disegna il diametro orizzontale tratteggiato
    \draw[dashed] (-2,0) -- (2,0);

    % Etichetta del diametro
    \node[below] at (0,0) {$d$};
\end{tikzpicture}
\caption{Sfera.}
\end{figure}

\textbf{Formula del volume:} $V = \frac{\pi d^3}{6}$

\subsection{Esercizi svolti}

\eserciziop{Calcola il volume di un cubo con lato \SI{5,00}{\centi\metre}.}

\soluzione{
$V = l^3 = (\SI{5}{\centi\metre})^3 = \SI{125}{\centi\metre\cubed}$
}

\eserciziop{Calcola il volume di un parallelepipedo con lunghezza \SI{6,0}{\centi\metre}, larghezza \SI{4,0}{\centi\metre} e altezza \SI{3,0}{\centi\metre}.}

\soluzione{
$V = l \cdot h \cdot p = \SI{6}{\centi\metre} \cdot \SI{4}{\centi\metre} \cdot \SI{3}{\centi\metre} = \SI{72}{\centi\metre\cubed}$
}

\eserciziop{Calcola il volume di un cilindro con diametro \SI{8,00000}{\centi\metre} e altezza \SI{10,000}{\centi\metre}.}

\soluzione{
$V = \frac{\pi d^2}{4} \cdot h = \frac{\pi \cdot (\SI{8}{\centi\metre})^2}{4} \cdot \SI{10}{\centi\metre} \approx \SI{502,65}{\centi\metre\cubed}$
}

\eserciziop{Calcola il volume di un cono a base retta con diametro \SI{6,000}{\centi\metre} e altezza \SI{9,000}{\centi\metre}.}

\soluzione{
$V = \frac{\pi d^2}{12} \cdot h = \frac{\pi \cdot (\SI{6}{\centi\metre})^2}{12} \cdot \SI{9}{\centi\metre} \approx \SI{84,82}{\centi\metre\cubed}$
}

\eserciziop{Calcola il volume di una sfera con diametro \SI{10,000}{\centi\metre}.}

\soluzione{
$V = \frac{\pi d^3}{6} = \frac{\pi \cdot (\SI{10}{\centi\metre})^3}{6} \approx \SI{523,60}{\centi\metre\cubed}$
}


\section{Densità} Dato un corpo (solido, liquido o gassoso) di massa $M$ e volume $V$, la densità è la grandezza (derivata):
\[
d=\frac{M}{V}
\]
Guardando la formula, ricaviamo che l'unità di misura è quella di una massa diviso un volume. Nel sistema internazionale, l'unità base è $\si{\kilo\gram\per\cubic\meter}$ ma esistono ovviamente altre unità. Ad esempio $\si{\gram\per\cubic\centi\meter}$.

\begin{testexample}[\thetcbcounter \,Calcoli con la densità]
$\SI{7860}{\kilo\gram\per\cubic\meter} = \cdots  \si{\kilo\gram\per\cubic\centi\meter} $. Dobbiamo trasformare il denominatore con le regole che conosciamo (ricordiamo che $\SI{1}{\cubic\meter} = \SI{e+6}{\cubic\centi\meter}$):
\[
\SI{7860}{\kilo\gram\per\cubic\meter} =\frac{\SI{7860}{\kilogram}}{\SI{e+6}{\cubic\centi\meter}}  =\SI{7,860e-3}{\kilo\gram\per\cubic\centi\meter}
\]

$\SI{1000}{\kilo\gram\per\cubic\meter} = \cdots \si{\gram\per\milli\liter}$. Ricordiamo che $\SI{1}{\cubic\centi\meter} = \SI{1}{\milli\liter}$:



\[
\SI{1000}{\kilo\gram\per\cubic\meter} = \frac{1000\cdot \SI{e+3}{\gram}}{\SI{e6}{\milli\liter}} = \SI{1}{\gram\per\milli\liter}
\]

Questa equivalenza è molto importante nella pratica dove si usano spesso i millilitri, dunque la mettiamo in evidenza:
\[
\colorboxed{ocre}{
\SI{e3}{\kilo\gram\per\cubic\meter} = \SI{1}{\gram\per\milli\liter}
}
\]
\end{testexample}

\begin{testexample}[\thetcbcounter \,Formule inverse e calcolo dimensionale.]

Introduciamo il calcolo delle unità di misura (calcolo dimensionale). Per ricavare l' unità di misura di una grandezza G (rappresentata dalla scrittura [G]) basta inserire nella formula, al posto di ogni grandezza, la sua unità di misura. Facciamo un esempio.
Notiamo che la formula del volume è corretta dal punto di vista delle unità di misura. Infatti, $[m]=\text{kg}\;\, [d]=\frac{\text{kg}}{{\text{m}^3}}$ e quindi:
        \[
        [V]=\frac{\text{kg}}{\frac{\text{kg}}{{\text{m}^3}}}=\cancel{\text{kg}}\cdot \frac{\text{m}^3}{\cancel{\text{kg}}   }=\text{m}^3.
      \]
\end{testexample}



\begin{testexample}[\thetcbcounter \,Esercizio sulla densità]

Un'ampolla di vetro ha una massa di $\SI{85,382}{\gram}$ e un volume di $\SI{80}{\milli\liter}$. L'ampolla viene riempita con gas xenon e la sua massa diventa $\SI{85,852}{\gram}$.
\begin{itemize}
\item Calcola la densità $d_{X_e}$ dello xenon.
\item In seguito, la lampada viene di nuovo svuotata e riempita di idrogeno, fino ad ottenere una densità di $0,153\, d_{X_e}$. Determina la  massa totale della lampada riempita di idrogeno.
\end{itemize}
\end{testexample}

Approfittiamo per poter ripassare le formule inverse e le cifre significative. Calcoliamo anzitutto la massa dello xeon per differenza:
\[
M_{X_e} = \SI{85,852}{\gram} -\SI{85,382}{\gram}  = \SI{0,470}{\gram} =\SI{4,70e-4}{\kilo\gram}
\]
Il testo vuole la densità in $\si{\kilo\gram\per\cubic\meter}$, quindi trasformiamo il volume in $\si{\cubic\meter}$ :  $\SI{80}{\milli\liter} =\SI{8,0e-5}{\cubic\meter}$.
Usando la formula per la densità, abbiamo:
\[
d_{X_e}=\frac{M}{V} = \frac{\SI{4,70e-4}{\kilo\gram}}{\SI{8,0e-5}{\cubic\meter}}=\SI{5,875}{\kilo\gram\per\cubic\meter}      \approx  \SI{5,9}{\kilo\gram\per\cubic\meter}.  
\]
Per determinare la massa totale della lampada riempita di idrogeno, dobbiamo sommare la massa dell'ampolla e la massa dell'idrogeno, massa che possiamo calcolare con la formula: $M=d\cdot V$.
Dai dati, vediamo che la densità dell'idrogeno, vale: 
\[
d_{H} = 0,153 \cdot\left( \SI{5,9}{\kilo\gram\per\cubic\meter}\right) = \SI{0,9027}{\kilo\gram\per\cubic\meter } \approx \SI{0,90}{\kilo\gram\per\cubic\meter }.
\]
La massa totale dell'ampolla e dell'idrogeno è dunque:
\[
\begin{aligned}
    M &= \SI{0,085382}{\kilo\gram} + d \cdot V \\
      &= \SI{0,085382}{\kilo\gram} + \left(\SI{0,90}{\kilo\gram\per\cubic\meter}\right) \times \left(\SI{8,0e-5}{\cubic\meter}\right) \\
      &= \SI{0,085382}{\kilo\gram}  + \SI{0,00045}{\kilo\gram} = \SI{0,085454}{\kilo\gram} = \SI{85,454}{\gram}.
\end{aligned}
\]

Notiamo che abbiamo dovuto fare un passaggio in più per poter decidere nella somma quante cifre tenere, togliendo la notazione scientifica. Questa operazione è spesso lunga per cui capiterà in calcoli molto lunghi di non usare la regola per le cifre significative della somma e tenere nel risultato finale, tante cifre significative quante presenti nei dati meno precisi.


\begin{testexample}[\thetcbcounter \,Esercizio sulla densità e il volume]
Un cilindro di metallo ha un'altezza di \(h = \SI{50,0}{\centi\metre}\) e un diametro di base \(d = \SI{20,0}{\centi\metre}\). La massa del cilindro è \(M = \SI{15,0}{\kilogram}\).

\begin{enumerate}
    \item \textbf{Calcola il volume del cilindro in litri.}
    \item \textbf{Calcola la densità del cilindro in \si{\kilogram\per\litre}.}
\end{enumerate}
\end{testexample}

Le formule sono:

\begin{itemize}
    \item Il volume \(V\) di un cilindro si calcola con la formula:
    \[
    V = \pi \left(\frac{d}{2}\right)^2 h = \frac{\pi d^2 h}{4}
    \]
    dove \(d\) è il diametro della base e \(h\) è l'altezza.

    \item La densità \(\rho\) di un oggetto si calcola con la formula:
    \[
    \rho = \frac{M}{V}
    \]
    dove \(M\) è la massa e \(V\) è il volume.
\end{itemize}

\textit{Svolgimento}\\

\begin{enumerate}
    \item \textbf{Calcolo del volume del cilindro:}

    \begin{itemize}
        \item Il diametro \(d\) e l'altezza \(h\) sono dati in centimetri:
        \[
        d = \SI{20,0}{\centi\metre}, \quad h = \SI{50,0}{\centi\metre}
        \]
        \item Il volume in centimetri cubi è:
        \[
        V = \frac{\pi d^2 h}{4} = \frac{\pi (\SI{20}{\centi\metre})^2 \times \SI{50}{\centi\metre}}{4} =  \SI{15708}{\centi\metre\cubed} \approx\SI{1,57e+4}{\centi\meter\cubed}
        \]
        \item Convertiamo il volume in litri (1 litro = 1000 centimetri cubi):
        \[
        V = \SI{1,57e+4}{\centi\meter\cubed} = \SI{15,7}{\liter}
        \]
    \end{itemize}

    \item \textbf{Calcolo della densità del cilindro:}

    \begin{itemize}
        \item La massa \(M\) del cilindro è già data in chilogrammi:
        \[
        M = \SI{15}{\kilogram}
        \]
        \item Utilizziamo la formula della densità:
        \[
        \rho = \frac{M}{V} = \frac{\SI{15}{\kilogram}}{\SI{15.7}{\litre}} \approx \SI{0.955}{\kilogram\per\litre}
        \]
    \end{itemize}
\end{enumerate}









\chapter{Errori di misura}
\section{Incertezza nelle misure ripetute }
Supponiamo di voler misurare il periodo di oscillazione di un pendolo. Un pendolo, in fisica, è costituito da una sferetta sospesa ad un filo. Quando la pallina viene sollevata e lasciata andare, compie un moto periodico. Il tempo che impiega la pallina ad andare e venire, si chiama periodo e si misura in secondi. Si veda la figura \ref{fig:pendolo}. I due punti più in alto si chiamano \textit{punti morti superiori} e il punto più basso, \textit{punto morto inferiore}. Nei punti alti superiori il pendolo rimane per un attimo vermo (prima di invertire il senso del moto) mentre nel punto morto inferiore, raggiunge la massima velocità.

   \begin{figure}[h!]
    \centering
    \includegraphics[width=0.3\linewidth]{path_to_image/pendolo.png} 
    \caption{Un pendolo oscillante.}
    \label{fig:pendolo}
\end{figure}  


Se si cronometrano le oscillazioni, si vede che, ripetendo la misura, non si ottiene lo stesso tempo. La ragione è che questo tipo di misura è soggetto ad incertezze casuali (ad esempio dovute al tempo di reazione di chi misura, tempo che cambia ogni volta che si ripete la misura stessa). Per ovviare a questo problema, è stata sviluppata una teoria statistica che consente di valutare il miglior valore e l'errore da associare alla misura nel limite in cui si fanno tante misure con strumenti di alta sensibilità. Senza entrare nel merito, ricordiamo semplicemente che il miglior valore risulta la media dei valori :

\[
\overline{T} = \frac{1}{N} \sum_{i=1}^{N} T_i
\]

\noindent
Questo significa che sommiamo tutti i valori $T_1, T_2, \ldots, T_N$ e poi dividiamo per il numero totale di valori $N$. In altre parole:
\[
\overline{T} = \frac{1}{N} (T_1 + T_2 + \cdots + T_N)
\]
Lo sparpagliamento dei dati  è legato alla cosiddetta deviazione standard:

\[
\sigma = \sqrt{\frac{1}{N} \sum_{i=1}^{N} (T_i - \overline{T})^2}
\]

\noindent
Questo significa che per ogni valore $T_i$, calcoliamo la differenza tra $T_i$ e la media $\overline{T}$, la eleviamo al quadrato, sommiamo questi quadrati per tutti i valori da $1$ a $N$, dividiamo per $N$, e infine prendiamo la radice quadrata. In altre parole:
\[
\sigma = \sqrt{\frac{1}{N} \left[ (T_1 - \overline{T})^2 + (T_2 - \overline{T})^2 + \cdots + (T_N - \overline{T})^2 \right]}
\]

Si assume invece, come incertezza (o errore assoluto), il rapporto tra la deviazione standard e la radice quadrata del numero di dati:
\[
\sigma_{\overline{T}}=\frac{\sigma}{\sqrt{N}}
\] 
Notiamo che questa quantità, diminuisce all'aumentare dei dati perché cresce il denominatore. Su questo punto, torneremo nell'ultimo capitolo di questi appunti.
Dopo aver calcolato l'incertezza, dobbiamo sempre ricordarci di scriverla con una sola cifra significativa. Vedremo più avanti vari esempi su come farlo. Le formule sono abbastanza complesse ma è semplice usare un foglio di calcolo dove esistono funzioni apposite per questo tipo di calcoli, oppure si può usare un linguaggio di programmazione con librerie matematiche (ad esempio numpy di Python). \\

Se non possiamo ripetere la misura molte volte (diciamo almeno 30) allora la statistica (ossia le tre formule scritte prima) non ci fornisce risultati attendibili. In tal caso, si segue una strada diversa. L'errore si calcola come semidispersione massima ed è dato dalla formula:
\[
\Delta X = \frac{X_{max}-X_{min}}{2}
\]
dove $X_{max}$  e$X_{min}$ sono il valore massimo e  il valore minimo dei nostri dati. Se questo valore viene più piccolo della sensibilità comunque, si assume come errore assoluto la sensibilità. Infatti,ricordando l'esempio del pendolo, se usiamo un cronometro con sensibilità di 0,1 s, è impossibile distinguere due valori che abbiano una differenza minore di questa quantità. 

Supponiamo di aver misurato il periodo $T$ di oscillazione di un pendolo 5 volte e di aver calcolato la media e la semidispersione, ottenendo $T=\SI{1,21}{s}$ e $\Delta x = \SI{0,51}{s}$. Anzitutto attotondiamo l'errore ad \underline{una} cifra significativa, ossia $\Delta x \approx \SI{0,5}{s}$, poi arrotondiamo il periodo alla seconda cifra decimale: $T\approx \SI{1,2}{s}$ e , in fine, scriviamo il risultato corretto:
\[
T=\left(1,2 \pm 0,5 \right)\si{\second}
\]  
\section{Incertezza relativa e e percentuale}
Data una misura $x=\left( \overline{x} \pm \Delta x \right)$ si definisce incertezza relativa il rapporto:
\[  \colorboxed{ocre}{
E_r =\frac{\Delta x}{\overline{x}}}
\]
L'incertezza relativa è un numero senza unità di misura, generalmente piccolo. Infatti, in un esperimento ben progettato, gli errori sono piccoli, rispetto alla grandezza, e dunque il rapporto è minore di uno. Strettamente legata all'incertezza, c'è l'incertezza precentuale, definita come il prodotto della prima per cento. L'incertezza relativa, consente di confrontare due misure, anche di grandezze diverse. Date due grandezze misurate, \textbf{quella più precisa ha l'incertezza relativa minore}. A titolo di esempio, calcoliamo l'incertezza relativa e percentuale della misura di periodo fatta sopra:
\[
Er=\frac{0,5}{1,2}=0,417
\]
mentre l'errore percentuale vale:
\[
E_{\%} = E_r \cdot 100 = 41,7\, \%
\]
Volendo confrontare la precisione di questa misura con questa misura di massa:
\[
m=\left(0,10 \pm 0,01 \right) \si{g}
\]
devo calcolare l'errore relativo della massa:
\[
E_r = \frac{0,01}{0,10}= 0,1
\]
quindi questa misura è più precisa.
\section{Esercizi}
\begin{esercizio}
E' stato misurato il volume di un oggetto ottenendo il risultato
\[
V=\left( 90,0 \pm 0,4\right)\si{mL}.
\]
Determina il risultato corretto nel sistema SI.\\
\hspace*{\fill}$\left[V=\left( 90,0 \pm 0,4\right)\times10^{-6}\,\si{m^3}\right]$ 
\end{esercizio}

\begin{esercizio}
In laboratorio è stata misurata la durata del periodo di un pendolo ottenendo il valore $T=\SI{1,874}{s}$ con una semidispersione di $\Delta T =\SI{0,0291}{s}$. Scrivi il risultato in maniera corretta e calcola l'errore percentuale.\\
\hspace*{\fill} $\left[T=\left(1,87 \pm 0,03\right)\si{s}\text{;} E_{\%}=1,6\%\right]$
\end{esercizio}

\begin{esercizio}
Un gruppo di studenti ha misurato il periodo di oscillazione di un pendolo semplice per 20 volte. I dati raccolti sono riportati nella tabella \ref{tab:pend}. Calcola la media delle misure del periodo (\(\overline{T}\)) e l'incertezza $\Delta T$ e scrivi il risultato in forma corretta.
\hspace*{\fill} $\left[T=\left(2,16 \pm 0,02 \right) \,\si{s} \right]$.
\begin{table}[h!]
\centering
\caption{Misure del periodo di oscillazione di un pendolo}
\label{tab:pend}
\begin{tabular}{cc}
\toprule
\textbf{Numero di occorrenze} & \textbf{Misura del periodo (s)} \\
\midrule
5  & 2.14 \\
4  & 2.15 \\
5  & 2.16 \\
3  & 2.17 \\
3  & 2.18 \\
\bottomrule
\end{tabular}
\end{table}


\end{esercizio}


\begin{esercizio}
E' stato misurato l'intervallo d'incertezza per la massa di un oggetto e si è ottenuto l'intervallo compreso tra $\SI{459,7}{g}$ e $\SI{460,3}{g}$. Determina la massa e il suo errore e scrivi il risultato in forma corretta.\\
 \hspace*{\fill}  $\left[M=\left(460,0 \pm 0,3\right)\si{g}\right]$
\end{esercizio}


\begin{esercizio}
Spiega la differenza tra incertezza assoluta e relativa.
\end{esercizio}

\begin{esercizio}
Elenca le seguenti misure per ordine crescente di precisione.
\begin{multicols}{3}
\begin{elenco}
 \item[a)] $L=\left(20,2 \pm 0,1\right)\si{cm}$
 \item[b)] $M=\left(4,01 \pm 0,01\right)\si{kg}$
 \item[c)] $t=\left(1,22 \pm 0,02\right)\si{s}$ \\
 \hspace*{\fill}  $\left[\text{b, a, c}\right]$
\end{elenco}
\end{multicols}

\end{esercizio}


\section{Confronto tra misure}
Supponiamo che due gruppi di studenti abbiano misurato la densità di un oggetto ottenendo i valori $d_1=\left(6,7 \pm 0,2\right)\si{g/cm^3}$ e $d_2=\left(6,9 \pm 0,2\right)\si{g/cm^3}$. Ci chiediamo: questi due valori sono compatibili? All'apparenza no, perchè sono diversi. Siamo portati a pensare che i materiali siano leggermente diversi, oppure uno dei gruppi abbia sbagliato la presa dati. In realtà, per rispondere correttamente, dobbiamo ricordare che le misure sono incerte e il risultato è un intervallo di valori. Nel primo caso l'intervallo và da un minimo di $\SI{6,5}{g/	cm^3}$ ad un massimo di $\SI{6,9}{g/cm^3}$. Per il secondo gruppo, l'intervallo va da $\SI{6,7}{g/cm^3}$ a $\SI{7,1}{g/cm^3}$. La situazione è rappresentata in figura:

\definecolor{zzttqq}{rgb}{0.6,0.2,0}
\definecolor{uququq}{rgb}{0.25,0.25,0.25}
\definecolor{qqqqzz}{rgb}{0,0,0.6}
\definecolor{zzqqtt}{rgb}{0.6,0,0.2}
\definecolor{qqqqff}{rgb}{0,0,1}
\begin{tikzpicture}[line cap=round,line join=round,>=triangle 45,x=1.0cm,y=1.0cm]
\clip(-0.92,-2.28) rectangle (10.44,3.34);
\fill[pattern color=zzttqq,fill=zzttqq,pattern=north east lines] (2.98,1.02) -- (2.96,0) -- (6,0) -- (6.02,1.02) -- cycle;
\draw [line width=3.6pt,color=zzqqtt] (0,0)-- (6,0);
\draw [line width=3.6pt,color=qqqqzz] (2.98,1.02)-- (8.98,1.02);
\draw (-0.12,-0.14) node[anchor=north west] {$6,5 $};
\draw (6.24,0.4) node[anchor=north west] {$6,9$};
\draw (2.1,1.42) node[anchor=north west] {$6,7$};
\draw (9.1,1.44) node[anchor=north west] {$7,1$};
\draw [line width=2pt,dash pattern=on 2pt off 2pt] (2.96,1.94)-- (2.96,-1.06);
\draw [dash pattern=on 2pt off 2pt] (6.02,1.94)-- (6.02,-1.06);
\draw [color=zzttqq] (2.98,1.02)-- (2.96,0);
\draw [color=zzttqq] (2.96,0)-- (6,0);
\draw [color=zzttqq] (6,0)-- (6.02,1.02);
\draw [color=zzttqq] (6.02,1.02)-- (2.98,1.02);
\begin{scriptsize}
\fill [color=qqqqff] (0,0) circle (1.5pt);
\fill [color=qqqqff] (6,0) circle (1.5pt);
\fill [color=qqqqff] (2.98,1.02) circle (1.5pt);
\fill [color=qqqqff] (8.98,1.02) circle (1.5pt);
\fill [color=uququq] (2.96,0) circle (1.5pt);
\fill [color=uququq] (6.02,1.02) circle (1.5pt);
\end{scriptsize}
\end{tikzpicture}

In questo grafico abbiamo riportato i due intervalli di misura in rosso e blu. La zona tratteggiata indica che i due intervalli hanno una zona in comune, ossia la nostra grandezza potrebbe avere come valore vero un valore compreso tra $6,7$ e $6,9$ ma non sappiamo dire di più. Si dice allora che le due misure sono compatibili. Se i due intervalli non si sovrappongono, allora le misure si dicono incompatibili. Nel seguente grafico vediamo proprio un caso simile.

\begin{tikzpicture}[line cap=round,line join=round,>=triangle 45,x=1.0cm,y=1.0cm]
\clip(-0.92,-2.28) rectangle (14.94,3.34);
\draw [line width=3.6pt,color=zzqqtt] (0,0)-- (5.02,0);
\draw [line width=3.6pt,color=qqqqzz] (6.5,1.02)-- (8.98,1.02);
\draw (-0.32,-0.04) node[anchor=north west] {$6,5 $};
\draw (4.52,0.04) node[anchor=north west] {$6,9$};
\draw (5.40,1.44) node[anchor=north west] {$6,10$};
\draw (9.00,1.46) node[anchor=north west] {$6,11$};
\begin{scriptsize}
\fill [color=qqqqff] (0,0) circle (1.5pt);
\fill [color=qqqqff] (5.02,0) circle (1.5pt);
\fill [color=qqqqff] (6.5,1.02) circle (1.5pt);
\fill [color=qqqqff] (8.98,1.02) circle (1.5pt);
\end{scriptsize}
\end{tikzpicture}
Come si vede il valore massimo della prima misura, $6,9$, è minore del valore minimo della seconda ($6,10$) dunque non ci sono valori comuni ai due intervalli d'incertezza, per cui le misure sono incompatibili. Questo si può generalizzare al caso di più misure. Se anche un solo intervallo non si sovrappone agli altri, allora le misure sono incompatibli. Ovviamente, se tutte le altre sono vicine tra loro e una sola misura è parecchio diversa, probabilmente gli sperimentatori hanno commesso qualche errore di distrazione in quel momento e la misura andrebbe
 ripetuta o semplicemente ignorata.
 
 \section{Esercizi}
 
\begin{esercizio}
Determina se le seguenti misure sono compatibili ed eventualmente indica quella non compatibile con le altre.
\begin{multicols}{3}
\begin{elenco}
 \item[a)] $\left(20,2 \pm 0,2\right)\si{cm}$\\
 \item[b)] $\left(20,0 \pm 0,1\right)\si{cm}$\\
 \item[c)] $\left(19,0 \pm 0,5\right)\si{cm}$ \\
 \hspace*{\fill}$\left[\text{No}\right]$
\end{elenco}
\end{multicols}

\end{esercizio} 
 
\begin{esercizio}
Determina se le seguenti misure sono compatibili:
\begin{multicols}{3}
\begin{elenco}
 \item[a)] $\left(210 \pm 5\right)\si{mm}$
 \item[b)] $\left(20,0 \pm 0,6\right)\si{cm}$
 \item[c)] $\left(20.5 \pm 0,1\right)\si{cm}$ \\
 \hspace*{\fill}  $\left[\text{Si}\right]$
\end{elenco}
\end{multicols}

\end{esercizio}   
 
 
 


 
\section{Misure indirette: propagazione degli errori}
Quando si misura una grandezza attraverso un calcolo, l'incertezza presente nei dati si ``propaga'' e la ritroviamo nella misura finale. Il modo con cui le incertezze si propagano viene studiato col metodo della propagazione degli errori. Alla base di questo metodo, ci sono due formule, usate nel caso la formula contenga prodotti/quozienti oppure somme/differenze. Tali formule, è bene dirlo, hanno una bene precisa giusitificazione matematica, ma noi non la forniremo, accontentandoci dei risultati.
\subsection{Somme/differenze}
Supponiamo che 

\[
c=a+b
\]

dove $a$ e $b$ sono grandezze di cui conosciamo l'errore. Quanto valgono il miglior valore di $c$  e il suo errore assoluto $\Delta c$? Ovviamente il miglior valore di $c$ è $\overline{c}=\overline{a}+\overline{b}$ (ad esempio se $a=1$ e $b=3$ allora $c= 4$). Per conoscere invece l'errore assoluto $\Delta c$ dovremmo conoscere l'intervallo di incertezza di $c$, ma $c$ non è stato misurato con uno strumento, ma con un calcolo. Non disponiamo di una riga, una bilancia o qualunque altro strumento per valutare la sensibilità. Dunque non è così ovvio come procedere. I risultati della teoria prevedono che $c$ abbia un valore che è compreso tra $\overline{c} -\Delta c$ e $\overline{c} +\Delta c$ dove l'errore assoluto è semplicemente la somma degli errori:
\[  \colorboxed{ocre}{
\Delta c = \Delta a +\Delta b}
\]

Questa formula quando si applica? Molto semplice. Supponiamo di voler misurare la lunghezza di un tavolo di circa $\SI{1100}{\milli\meter}$ e di disporre di una riga di $\SI{600}{\milli\meter}$ di portata. Allora dovremmo usare \textit{due volte} la riga, una volta per intero   e una volta per una parte in modo che la nostra lunghezza diventi:
\[
\overline{l}= \SI{600}{\milli\meter} +\SI{500}{\milli\meter}=\SI{1100}{\milli\meter}
\]
Dunque, abbiamo fatto un calcolo. Accostando la riga cioè, abbiamo mentalmente fatto una somma perché ci sembra ovvio che, mettendo in fila la riga, le lunghezze si sommino. La teoria ci dice che l'errore sulla lunghezza vale:
\[
\Delta l = \SI{1}{\milli\meter}+\SI{1}{\milli\meter}=\SI{2}{\milli\meter}
\]
in quanto l'errore di lettura sulla riga è di $\SI{1}{\milli\meter}$.

Lo stesso discorso si può fare in altri casi. Supponiamo ad esempio di disporre di una bilancia di portata $\SI{4000}{\gram}$. Se dobbiamo pesare due pacchi di peso $\SI{3500}{\gram}$ e $\SI{2000}{\gram}$, non possiamo metterli entrambi sulla bilancia perchè si romperebbe (lo so l'esempio vi sembra artificioso e vi starete chiedendo ``come fai a sapere quanto pesano senza pesarli?'', ma  si può  valutare  ad occhio la massa.) Poi è sempre meglio essere prudenti in queste cose e se una volta pesati separatamente, vediamo che la somma è minore di $\SI{4000}{\gram}$, allora significa che siamo stati troppo prudenti!). Anche  in questo caso quindi, ricorreremo ad una \underline{somma} e, supponendo che la singola pesata abbia un errore di $\SI{1}{g}$, il nostro risultato è :
\[
M =\left(5500 \pm 2 \right)\si{\gram}
\]
Notiamo che abbiamo preso come errore la somma degli errori, come nel caso precedente.

Questo discorso vale anche per le differenze. La formula analoga alla precedente, nel caso $d=a-b$, è 


\[  \colorboxed{ocre}{
\Delta d = \Delta a + \Delta b}
\]
ossia facciamo \textbf{la somma degli errori}, non la differenza. Un esempio è la misura della'area di una figura cava.
\begin{testexample}[ \thetcbcounter \, Area di un trapezio forato]
 Guardando la figura in basso, si chiede di calcolare l'area della parte in grigio conoscendo l'area esterna e l'area del cerchio. Supponiamo che l'area esterna (trapezio) valga:
\[
A_{tot}=\left (120 \pm 2\right)\si{cm^2}
\]
mentre quella del cerchio
\[
A_{cerc}=\left( 65 \pm 1\right)\si{cm^2}
\]
\definecolor{ffffff}{rgb}{1,1,1}
\begin{tikzpicture}[line cap=round,line join=round,>=triangle 45,x=1.0cm,y=1.0cm]
\clip(-2.86,-0.18) rectangle (10.16,4.14);
\fill[fill=black,fill opacity=0.65] (0,0) -- (4,4) -- (10,4) -- (10,0) -- cycle;
\draw [line width=0.4pt,fill=black,fill opacity=1.0] (6.42,2) circle (1.4cm);
\draw (0,0)-- (4,4);
\draw (4,4)-- (10,4);
\draw (10,4)-- (10,0);
\draw (10,0)-- (0,0);
\draw [color=ffffff](6,2.18) node[anchor=north west] {\textbf{$A_{cerc}$}};
\draw [color=ffffff](2.52,0.92) node[anchor=north west] {\textbf{$A_{est}$}};
\end{tikzpicture}

allora l'area grigia vale:
\[
A_{est}=\left ( 55 \pm  3\right)\si{cm^2}
\]
\end{testexample}
\subsection{Moltiplicazioni/divisioni}
La regola per gli errori sui prodotti è più complessa. Essa afferma che se una grandezza è prodotto o rapporto tra due grandezze, il suo errore relativo si può calcolare direttamente senza calcolare l'errore assoluto. Quindi con i prodotti e quozienti noi calcoliamo subito l'errore relativo. Traduciamo in formule questa frase. Se $c=a \cdot b$, allora:
\[  \colorboxed{ocre}{
\frac{\Delta c}{\overline{c}} =\frac{\Delta b}{\overline{b}}+\frac{\Delta a}{\overline{b}}
}
\]
Questa formula lega l'errore relativo su $c$ a quelli di $a$ e $b$. Facciamo un esempio. Supponiamo di aver misurato i lati $a$ e $b$ di un rettangolo coi relativi errori ottenendo: $a=\left(10,25  \pm 0,05\right)\si{mm}$ e  $b=\left(15,00  \pm 0,05\right)\si{mm}$. L'errore relativo su $c$ è pertanto:
\[
\frac{\Delta c}{\overline{c}}=\frac{0,05}{10,25} +\frac{0,05}{15,00}=0,008211382
\]
Essendo questo un errore relativo, possiamo lasciare qualche cifra significativa in più. Ora calcoliamo l'area :
\[
\overline{A}=\overline{a} \cdot \overline{b} = \SI{10,25}{mm} \cdot \SI{15,00}{mm}  = \SI{153.75}{mm^2} 
\]
Ora ci chiediamo come calcolare l'errore assoluto. Ma sappiamo che l'errore relativo e l'errore assoluto sono legati, e l'errore assoluto, si può calcolare da una formula inversa: $\Delta A = E_r \cdot \overline{A}=0,008211382 \cdot \SI{153.75}{mm^2}=\SI{1,26}{mm^2}	\approx \SI{1}{mm^2}$. In definitiva abbiamo:
\[
A=\left(154 \pm 1\right)\si{mm^2}
\]
Notiamo che abbiamo approssimato l'area per avere lo stesso numero di cifre decimali dell'errore.

\textbf{Osservazione.} In questo esercizio abbiamo calcolato prima l'errore relativo e poi quello assoluto. Se in un problema è richiesto solo l'errore relativo, questo approccio è il più conveniente. In laboratorio di solito, si richiede di calcolare direttamente l'errore assoluto. In tal caso, possiamo verificare che la formula da applicare è:
\[  \colorboxed{ocre}{
\Delta c = \overline{c} \cdot \left(\frac{\Delta a}{\overline{a}} + \frac{\Delta b}{\overline{b}}\right)}
\]
Notiamo che l'unità di misura dell'errore è la stessa di $c$ perché la quantità in parentesi è una somma di errori relativi e quindi non ha unità di misura. Per questo, quando si fanno gli esercizi. è sempre bene effettuare i calcoli portandosi dietro le unità di misura. Notiamo ancora che, in parentesi, c'è una somma di errori relativi. Se uno dei due è molto più piccolo dell'altro, allora possiamo trascurarlo. Questo significa che l'errore relativo finale è l'errore relativo della misura \underline{meno precisa}. Per questo, quando in laboratorio facciamo misure, dobbiamo evitare che una misura rozza ``rovini'' il risultato. Questa formula vale anche se vogliamo calcolare l'errore in un rapporto, ovvero, se $c=\frac{a}{b}$, allora: 

\[
\Delta c = \overline{c} \cdot \left(\frac{\Delta a}{\overline{a}} + \frac{\Delta b}{\overline{b}}\right)
\] 


Facciamo un altro esempio.
\vspace{0.5 cm}
\begin{testexample}[\thetcbcounter \, Densità di un liquido]
Supponiamo di aver misurato la massa e il volume di un liquido ottenendo i risultati: $M=\left(320,0  \pm 0,1 \right)\si{g} $ e $V=\left(407 \pm 1 \right)\si{mL}$. Calcoliamo la densità senza approssimarla:
\[
\overline{d}=\frac{\overline{M}}{\overline{V}}=\frac{\SI{320,0}{g}}{\SI{407}{mL}}=\SI{7,400491}{\frac{g}{mL}}
\]
Ora calcoliamo l'errore:
\[
\Delta d = \overline{d}\cdot \left(\frac{\Delta M}{\overline{M}} +\frac{\Delta V}{\overline{V}}  \right)   = \SI{7,400491}{\frac{g}{mL}}\cdot\left(\frac{0,1}{320,0} +\frac{1}{407}  \right) =\SI{0,02}{\frac{g}{mL}}
\]
In definitiva la nostra densità vale:
\[
d=\left(7,40 \pm 0.02 \right)\si{\frac{g}{mL}}
\]
\end{testexample}


\begin{testexample}[\thetcbcounter \, Velocità media]
Un gruppo di studenti ha misurato il tempo $t=\left(2,3450 \pm 0,0001\right)\,\si{\second}$ impiegato da un carrello, per percorrere  lo spazio percorso $s=\left(1822 \pm 1\right)\,\si{\milli\meter}$.  Determinare la velocità media e il suo errore assoluto e scrivere il risultato in forma corretta nel sistema internazionale: $v = \left( \overline{v} \pm \Delta v\right)\,\si{\meter\per\second}$.


La velocità media $\overline{v}$ si ottiene dividendo lo spazio percorso $s$ per il tempo impiegato $t$:

$$\overline{v} = \frac{\overline{s}}{\overline{t}} = \frac{1822\,\si{\milli\meter}}{2,345\,\si{\second}} = \SI{776,97}{\milli\meter\per\second} =  \SI{0,77697}{\meter\per\second}  $$
Calcolo dell'errore assoluto:
\[
\Delta v = \overline{v}\cdot \left(\frac{\Delta s}{\overline{s}} +\frac{\Delta t}{\overline{t}}  \right) =  \left( \SI{0,77697}{\meter\per\second}\right)\cdot \left(\frac{1}{1822} +\frac{0,0001}{2,3450}  \right) = \SI{0,000459571}{\meter\per\second} \approx \SI{0,0005}{\meter\per\second}.   
\]
In definitiva:
\[
v=\left(0,7770 \pm 0,0005\right) \, \si{\meter\per\second}
\]
Osserviamo che nel calcolo dell'errore relativo, non occorre trasformare i dati nel sistema S.I. perché eventuali fattori di conversione, si semplificherebbero nella frazione. Inoltre, prima di inserire la velocità nella scrittura corretta, l'abbiamo approssimata alla quarta cifra decimale come l'errore. Si noti in fine, che non c'è stato bisogno di riportare i passaggi intermedi per gli errori relativi e applicare le regole sulle cifre significative nella somma, perché, quando si calcolano gli errori, quelle regole non vanno applicate e si può impostare tutto il calcolo direttamente sulla calcolatrice. Stesso discorso quando abbiamo molte potenze di 10: conviene inserirle nella calcolatrice e lasciare che si occupi del calcolo, questo non è un corso di matematica!
 \end{testexample}

\begin{testexample}[\thetcbcounter \, Volume di un parallelepipedo]
Il volume di un parallelepipedo è dato dalla formula:
\[
V=a \cdot b \cdot c
\]
Immaginiamo di avere i seguenti valori dei lati: $a=\left(10,2 \pm 0,1\right)\si{cm}$, $b=\left(8,0 \pm 0,1\right)\si{cm}$, e $c=\left(12,6 \pm 0,1\right)\si{cm}$.
Calcoliamo l'errore assoluto sul volume. La regola del prodotto contiene tre fattori, pertanto avremo:
\[
\Delta V=\overline{V} \cdot \left( \frac{\Delta a}{\overline{a}}+\frac{\Delta b}{\overline{b}} +\frac{\Delta c}{\overline{c}}\right)
\]
Prima calcoliamo il volume: $\overline{V}=\left(\SI{10,2}{cm}\right)\cdot \left(\SI{8,0}{cm}\right)\cdot \left(\SI{12,6}{cm}\right)=\SI{1028,16}{cm^3}$.
L'errore diventa:
\[
\Delta V=\SI{1028,16}{cm^3} \cdot \left( \frac{0,1}{10,2}+\frac{0,1}{8,0} +\frac{0,1}{12,6}\right)=\SI{31,092}{cm^3}
\]
Questo errore ha troppe cifre significative. Per ridurle, possiamo usare la notazione scientifica oppure, un multiplo del centimetro, in modo da ottenere un errore minore di 9 che sappiamo approssimare. Pertanto scriveremo:
\[
\Delta V = \SI{0,031092}{dm^3}\approx \SI{0,03}{dm^3}
\]
Analogamente si ha:
\[
V=\SI{1,02816}{dm^3} 
\]
Dunque la misura finale scritta col corretto numero di cifre significative è:
\[
V=\left(1,03 \pm 0,03\right)\si{dm^3}
\]
Quando gli errori sono maggiori di 10, è concesso lasciare gli zeri ammettendo che, anche se normalmente sarebbero significativi (perché alla fine del numero) in questo caso non lo sono. Quindi, volendo lasciare tutto in centimetri, avremmo anche potuto scrivere:
\[
V=\left(1030 \pm 30\right)\,\si{\cubic\centi\meter}.
\]
\end{testexample}
\begin{testexample}[\thetcbcounter \, perimetro di un quadrato]
Consideriamo un quadrato di lato  $l=\left(10,2 \pm 0,1\right)\si{cm}$. La formula del perimetro è:
\[
\overline{P}=4\overline{l} = 4\cdot \left(\SI{10,2}{\centi\meter} \right) =\SI{40,8}{\centi\meter}.
\]
pertanto, usando la regola per la propagazione dell'errore in un prodotto, abbiamo:
\[
\Delta P= \overline{P}\cdot\left( \cancel{\frac{\Delta 4}{4}} +\frac{\Delta l}{\overline{l}}  \right) = \left(\SI{40,8}{\centi\meter}\right)\cdot \left(\frac{0,1}{10,2} \right) =\SI{0,4}{\centi\meter} .
\]
dove abbiamo cancellato il primo addendo perché 4 (cosi come $\pi$ e tutte le costanti nelle formule) non ha errore (però bisogna metterlo nel calcolo del perimetro). In definitiva abbiamo:
\[
P=\left(40,8 \pm 0,4\right) \si{\centi\meter}.
\]
\end{testexample}




Notiamo che se si ha una formula del tipo:
\[
c=\frac{a\cdot b^2}{d}
\]
ogni fattore al numeratore o al denominatore contribuisce tante volte all'errore relativo quanto è alta la sua potenza perchè la potenza non è altro che una moltiplicazione ripetuta. Dunque, la formula dell'errore diventa:
\[
\Delta c= \overline{c} \cdot \left(\frac{\Delta a}{\overline{a}}+\frac{\Delta b}{\overline{b}} +\frac{\Delta b}{\overline{b}}  +\frac{\Delta d}{\overline{d}} \right)= \left(\frac{\Delta a}{\overline{a}}+2\frac{\Delta b}{\overline{b}}  +\frac{\Delta d}{\overline{d}} \right)
\]
\begin{testexample}[\thetcbcounter \, Misura dell'accelerazione di gravità]
Nella teoria del moto di un oggetto in caduta, si ricava la seguente formula per il calcolo dell'accelerazione di gravità $g$:
\[
g=\frac{2\cdot h}{t^2}
\]
essendo $h$ l'altezza da cui il corpo cade, e $t$ il tempo impiegato a cadere. Vogliamo calcolare l'errore su $g$ supponendo le seguenti misure: $h=\left(6,00 \pm 0,01\right)\si{m}$, e $t=\left(1,11\pm 0,01\right)\si{s}$. Inserendo i valori otteniamo subito: $g = \SI{9,74}{m/s^2}$.
Per l'errore è facile convincersi che la formula da usare è:
\[
\Delta g= \overline{g}\left(\frac{\Delta h}{\overline{h}} +2\frac{\Delta t}{\overline{t}}\right)=\SI{0,19}{m/s^2}\approx \SI{0,2}{m/s^2}
\]
In definitiva:
\[
g=\left(9,7 \pm 0,2\right)\si{m/s^2}
\]
\end{testexample}
\begin{testexample}[\thetcbcounter \,  Volume del cilindro]
Consideriamo un cilindro di raggio $r=\left(1,205 \pm 0,005\right)\si{cm}$ e altezza $h=\left(4,000 \pm 0,005 \right)\si{cm}$. La formula per il calcolo del volume è:
\[
V=\pi\cdot  r^2 \cdot h
\]
pertanto il valore numerico risulta: $V =\SI{18,247}{cm^3}$, mentre per l'errore si ha:
\[
\Delta V=\overline{V}\cdot \left(2\frac{\Delta r}{\overline{r}} +\frac{\Delta h}{\overline{h}}\right)=\SI{0,174}{cm^3}\approx \SI{0,2}{cm^3}
\]
La misura comprensiva dell'errore è pertanto:
\[
V=\left(18,2 \pm 0,2\right)\si{cm^3}
\]
\end{testexample}


\subsection{Esercizi}
\begin{esercizio}
In laboratorio si vuole misurare il volume di un solido irregolare. Per fare questo, si misura inizialmente il volume di una certa quantità d'acqua presente in un becker, ottenendo il valore  $V_i=\left(60 \pm 1\right)\si{mL}$. Una volta inserito il corpo, il livello sale a $V_f=\left(105 \pm 1\right)\si{mL}$. Determina il volume dell'oggetto sommerso e scrivi il risultato in forma corretta.\\
 \hspace*{\fill}  $\left[V=\left(45 \pm 2\right)\si{mL}\right]$
\end{esercizio}

\begin{esercizio}
Si vuole misurare la lunghezza di una stanza utilizzando un metro con portata $\SI{2,000}{m}$ e sensibilità $\SI{0,1}{cm}$. Il valore misurato risulta essere $L=\SI{6,450}{m}$. Scrivi il risultato in forma corretta (hai già il valore della grandezza, ti manca l'errore). Rifletti su come potrebbe essere stata effettuata la misura per capire quanto vale l'errore.\\
 \hspace*{\fill}  $\left[L=\left(6,450 \pm 0,003\right)\si{m}\right]$
\end{esercizio}

\begin{esercizio}
Un pentagono regolare, ha lato $L=\left(2,5 \pm 0,1\right)\si{cm}$. Determina il suo perimetro e scrivi il risultato con l'errore.\\
 \hspace*{\fill}   $\left[2P=\left(12,5 \pm 0,5\right)\si{cm}\right]$
\end{esercizio}
\begin{esercizio}
Viene misurato 10 volte il tempo T di oscillazione di un pendolo, ottenendo i risultati in tabella:

\begin{center}
\begin{tabular}{|c|c|}
\hline 
T(s)& Ripetuto\\
\hline
1,25 & 2 volte\\
1,24 & 4 volte \\
1,23 & 1 volta \\
1,20 & 2 volte \\
1,21 & 1 volta \\
\hline
\end{tabular}
\end{center}
Determina il valore medio e l'errore. In fine, scrivi il risultato in forma corretta.\\
\hspace*{\fill}   $\left[T=\left(1,23 \pm 0,03\right)\si{s}\right]$
\end{esercizio}
\begin{esercizio}
Un modo per misurare l'accelerazione di caduta, è usare la seguente relazione:
\[
g=\frac{v^2}{2 h}
\]
essendo $h$ l'altezza da cui si fa cadere un oggetto, e $v$ la velocità di caduta (misurata con un cronometro fatto partire automaticamente quando l'oggetto cade, tramite una fotocellula). Un gruppo di studenti, ha misurato $h$ e $v$ ottenendo:
\[
h=\left(1226 \pm 1 \right) \, \si{\milli\meter} \text{;}\,\, v=\left(4,8531 \pm  0,0001\right) \, \si{\meter\per\second} 
\]
Determina il valore di $g$ col suo errore e scrivi il risultato in forma corretta. \\\hspace*{\fill} [$g=\left(9,605 \pm 0,008 \right) \, \si{\meter\per\square\second}$] 
\end{esercizio}

\begin{esercizio}
In geometria, il volume di una sfera di raggio $r$ si calcola con la formula:
\[
V=\frac{4\cdot\pi \cdot r^3}{3}
\] 
Supponiamo di aver misurato col calibro il diametro $d$ di una sfera avendo ottenuto $d=\left(9,555 \pm 0,005\right)\si{cm}$. Dimostra che la formula del volume, espressa rispetto al diametro è:
\[
V=\frac{\pi \cdot d^3}{6}
\]
Utilizzando questa formula:
\begin{enumerate}
\item[a)] Calcola il volume;
\item[b)] Scrivi la formula per il calcolo dell'errore;
\item[c)] Calcola l'errore.
\item[d)] Scrivi il risultato in forma corretta.
\end{enumerate}
 \hspace*{\fill}   $\left[\overline{V}=\SI{456.76}{cm^3}\text{,} \Delta V=\overline{V}\cdot\left(3\frac{\Delta d}{d}\right)\approx\SI{0.7}{cm^3}\text{,} V=\left(456,8 \pm 0,7 \right)\si{cm^3}\right] $
\end{esercizio}
\vspace{0.2cm}
\begin{esercizio}
Si sa che una grandezza $f$ dipende dalle grandezze $a$, $b$, $c$, $d$, $e$ attraverso la formula:
\[
f = 4 \cdot \frac{a\cdot b^2 \cdot c^3}{d^4 \cdot e^5}
\]
Determina la formula per il calcolo dell'errore assoluto su $f$.\\

\hspace*{\fill} $\left[\Delta f=\overline{f}\cdot\left(\frac{\Delta a}{\overline{a}}+ 2\cdot \frac{\Delta b}{\overline{b}} +3\cdot\frac{\Delta c}{\overline{c}} + 4\cdot \frac{\Delta d}{\overline{d}} + 5\cdot \frac{\Delta e}{\overline{e}}  \right)\right]$
\end{esercizio}

 \chapter{Grafici di misure}
In fisica spesso è possibile verificare una formula facendo variare una delle grandezze contenute e osservando su un grafico come varia un'altra grandezza. A questo scopo si predispone una tabella coi dati e poi si realizza un grafico avendo come obiettivo una buona leggibilità. Elencheremo di seguito una procedura passo-passo, per realizzare dei buoni grafici elencando tutte le caratteristiche indispensabili. Nelle prossieme sezioni, realizzeremo, prima un grafico a mano, poi con l'uso del foglio di calcolo.
\section{Un esempio di grafico realizzato a mano}

Supponiamo di aver misurato la capacità $C$ di un condensatore (unità di misura picofarad, pF)  in funzione della distanza tra le armature di un condensatore piano (distanza d espressa in mm). I dati sono in tabella \ref{tab:condensatore}: 

\begin{table}
\begin{center}
{
\def\arraystretch{1.5} %serve a inserire spazio bianco
\begin{tabular}{|c|c|}
\hline 
 $d\left( \si{\milli\meter}\right)$ & $C \left( \si{\pico\farad}\right)$   \\
\hline
2 & 420\\
3 &  282 \\
4 &  219 \\
5 & 182 \\
6 & 158 \\
7 &  139\\
8 &  127\\
9  & 115\\
10 & 108\\
\hline
\end{tabular}
}
\caption{Dati per il grafico capacità-distanza}
\label{tab:condensatore}
\end{center}
\end{table}
La prima cosa che notiamo, è che all'aumentare della distanza, la capacità diminuisce. Questo tipo di andamento, vedremo, è indizio (ma non è sufficiente) di una proporzionalità inversa. Ti invito a realizzare il grafico che stiamo costruendo, anche col foglio di calcolo, usando il metodo che vedremo più avanti. Vediamo quali scelte vanno fatte per il grafico.
\subsection{Disegno degli assi e titolo del grafico}
Possiamo disegnare subito gli assi lungo il bordo arancione e dare un titolo al grafico (figura \ref{fig:titoloassi})

   \begin{figure}[h!]
    \centering
    \includegraphics[width=\linewidth]{path_to_image/titoloassi.pdf} 
    \caption{Titolo e  etichette sugli assi}
    \label{fig:titoloassi}
\end{figure}  

\subsection{Scelta dei fattori di scala i valori maggiori} 
Per scegliere la scala, dobbiamo calcolare approssimativamente dove vogliamo che il grafico termini, e poiché vogliamo che venga grande, faremo in modo che i valori massimi delle due variabili capitino alla fine del foglio. Abbiamo un foglio in cui l'area arancione ha   dimensioni di $280 \times 200 $ millimetri e vogliamo che il massimo valore della distanza in tabella, ossia $\SI{10}{\milli\meter}$, capiti alla fine dell'asse X ad una distanza di circa $\SI{280}{\milli\meter}$. Dunque calcoliamo il fattore di scala sull'asse X:
\[
F_x=\frac{\SI{280}{\milli\meter}}{\SI{10}{\milli\meter}} = 28
\]
Con questo fattore potremo inserire ogni dato della prima colonna sul grafico moltiplicandolo per $F_x$. Per esempio, il valore $\SI{2}{\milli\meter}$ andrà inserito sul punto dell'asse X di ascissa $28\times\SI{2}{\milli\meter} = \SI{56}{\milli\meter}$. Analogamente, il fattore di scala sull'asse Y è:
\[
F_y = \frac{\SI{200}{\milli\meter}}{\SI{420}{\pico\farad}} = \SI{0,476}{\milli\meter\per\pico\farad}.
\] 

E' sempre meglio approssimare per difetto i fattori di scala, altrimenti i valori massimi della tabella, potrebbero uscire dai bordi del foglio (dalla parte colorata).
Effettuando i calcoli, otteniamo per i dati sull'asse X:
\begin{equation*}
\begin{aligned}
\SI{2}{\milli\meter} \times 28 &= \SI{56}{\milli\meter}  \\
\SI{3}{\milli\meter}  \times 28 &= \SI{84}{\milli\meter}  \\
\SI{4}{\milli\meter}  \times 28 &= \SI{112}{\milli\meter}   \\
\SI{5}{\milli\meter}  \times 28 &= \SI{140}{\milli\meter}  \\
\SI{6}{\milli\meter}  \times 28 &= \SI{168}{\milli\meter}   \\
\SI{7}{\milli\meter}  \times 28 &= \SI{196}{\milli\meter}  \\
\SI{8}{\milli\meter}  \times 28 &= \SI{224}{\milli\meter}  \\
\SI{9}{\milli\meter}  \times 28 &= \SI{252}{\milli\meter}  \\
\SI{10}{\milli\meter}  \times 28 &= \SI{280}{\milli\meter}  \\
\end{aligned}
\end{equation*}
Facciamo anche i calcoli per i dati da inserire sull'asse Y:

\[
\begin{array}{rcl}
\SI{420}{\pico\farad} \times \SI{0.476}{\milli\meter\per\pico\farad} &=& \SI{199.92}{\milli\meter} \\
\SI{282}{\pico\farad} \times \SI{0.476}{\milli\meter\per\pico\farad} &=& \SI{134.232}{\milli\meter} \\
\SI{219}{\pico\farad} \times \SI{0.476}{\milli\meter\per\pico\farad} &=& \SI{104.244}{\milli\meter} \\
\SI{182}{\pico\farad} \times \SI{0.476}{\milli\meter\per\pico\farad} &=& \SI{86.632}{\milli\meter} \\
\SI{158}{\pico\farad} \times \SI{0.476}{\milli\meter\per\pico\farad} &=& \SI{75.208}{\milli\meter} \\
\SI{139}{\pico\farad} \times \SI{0.476}{\milli\meter\per\pico\farad} &=& \SI{66.164}{\milli\meter} \\
\SI{127}{\pico\farad} \times \SI{0.476}{\milli\meter\per\pico\farad} &=& \SI{60.452}{\milli\meter} \\
\SI{115}{\pico\farad} \times \SI{0.476}{\milli\meter\per\pico\farad} &=& \SI{54.74}{\milli\meter} \\
\SI{108}{\pico\farad} \times \SI{0.476}{\milli\meter\per\pico\farad} &=& \SI{51.408}{\milli\meter} \\
\end{array}
\]
Ovviamente, questi valori devono essere arrotondati per difetto al millimetrio più vicino (il foglio permette al massimo di tracciare i millimetri). Queste due tabelle dei calcoli, non vanno riportate sul foglio di relazione ma solo sui propri appunti. Quello che ci interessa sono i numeri finali (i millimetri). Riportiamo in fine, i dati sul grafico (figura \ref{fig:graficomanuale}) dove abbiamo anche tracciato (ad occhio) una linea di tendenza. 
 
    \begin{figure}[h!]
    \centering
    \includegraphics[width=\linewidth]{path_to_image/graficomanuale.pdf} 
    \caption{Grafico su carta millimetrata.}
    \label{fig:graficomanuale}
\end{figure} 


 










\section{Uso di un foglio di calcolo}

\begin{definizione}
    Il foglio di calcolo è uno strumento informatico per l'elaborazione e l'analisi  dei dati.
\end{definizione}

Con il suo aiuto possiamo organizzare  i dati in tabelle ed eseguire un'ampia serie di operazioni matematiche per elaborare e analizzare dati scientifici. Esistono diversi programmi di calcolo tra i quali Microsoft Excel, Libreoffice calc, Google sheets, Numbers, OpenOffice calc. 

\begin{remark}
Tutti i fogli di calcolo presentano caratteristiche generali di funzionamento simili e le operazioni più comuni vengono eseguite con le stesse istruzioni nella maggior parte dei programmi. Perciò, una volta imparati i principi base di uno di essi, sarà facile cominciare ad usarne un altro. 
    \end{remark}

    Ogni file creato da un foglio di calcolo rappresenta una cartella di lavoro, che contiene a sua volta più fogli di lavoro. Ciascun foglio di lavoro è formato da una grande tabella di celle; la posizione delle celle nel foglio (indirizzo) è identificata da una lettera e da un numero (per esempio, la cella C6 si trova nella terza colonna, chiamata C, e nella sesta riga) che sono il suo riferimento di cella.
    Una cella può contenere parole, numeri, date o formule.
    \begin{figure}[h!]
        \centering
        \includegraphics[width=\linewidth]{path_to_image/calc.png} 
        \caption{Panoramica foglio di calcolo.}
        \label{fig:librecalc}
    \end{figure} 

\subsection{Inserire e modificare i dati}
\begin{figure}[h!]
    \centering
    \includegraphics[scale=0.3]{path_to_image/tasti.png} 
    \label{fig:tasti}
\end{figure} 

Per scrivere all'interno di una cella, è necessario selezionarla puntandola col mouse e facendo clic su di essa. La cella selezionata si attiva e il suo indirizzo compare nel riquadro in alto a sinistra.
Il testo digitato in una cella viene contemporaneamente riprodotto nella barra della formula. Per confermare l'immissione premiamo \texttt{Invio}, oppure un tasto freccia sulla tastiera che porta a un'altra cella.
Se l'immissione è riconosciuta come testo, viene allineata a sinistra della cella; se è riconosciuta come numero, viene allineata a destra.
Si può selezionare un insieme di celle facendo clic con il mouse sulla prima e, tenendo premuto il pulsante, trascinando il cursore fino all'ultima. Per selezionare gruppi di celle non adiacenti, si seleziona il primo gruppo e poi si procede con i successivi tenendo premuto il tasto CTRL (oppure il tasto Command).

\subsection{Formato di una cella}


Il formato di una cella determina come viene mostrato il suo contenuto.
 Per esempio, se vogliamo visualizzare numeri in notazione scientifica, scegliamo
  il menu \texttt{Formato}, poi \texttt{Celle} e infine, nella finestra di dialogo che comparirà, 
  scegliamo la scheda \texttt{Numeri} e la categoria \texttt{Scientifico} 
  (se digitiamo il numero 0,000005 e premiamo invio, 
  nella cella comparirà $\num{5,00e-6}$).
  Il numero può anche essere inserito direttamente in notazione scientifica,
   per esempio come 3E+6 per indicare $\num{3e+6}$.
  Utilizzando invece il menù \texttt{Formato/Celle/Numero} possiamo scegliere
   quante cifre decimali di ogni numero mostrare, per facilitare
    la lettura e il confronto. I dati sono comunque registrati e utilizzati 
    con tutte le cifre con cui sono stati inseriti o calcolati e 
    la modifica riguarda solo la loro presentazione grafica.
  \begin{figure}[h!]
    \centering
    \includegraphics[scale=0.4]{path_to_image/formato-cella.png} 
    \caption{Formato in notazione scientifica}
    \label{fig:formatocella}
\end{figure} 

Nella colonna A della figura \ref{fig:insdati} sono stati inseriti quattro numeri decimali, di cui due in notazione scientifica. I numeri sono stati copiati nella colonna B, formattata con due cifre decimali, 
e nella colonna C, formattata in notazione scientifica con una cifra decimale.
\begin{figure}[h!]
    \centering
    \includegraphics[scale=0.4]{path_to_image/inserimento-dati.png} 
    \caption{Inserimento dati}
    \label{fig:insdati}
\end{figure}
\subsection{Inserire e modificare le formule}

\begin{definizione}
Una formula è un'espressione che, digitata in una cella,
 effettua automaticamente dei calcoli o delle operazioni sui dati del 
 foglio di calclo. Ogni formula deve essere preceduta dal simbolo =.
\end{definizione}

















\begin{testexample}

La cella A2 contiene il valore 2,5 e la cella B2 contiene
 il valore 3,5. Se nella cella C2 digitiamo \texttt{=A2+B2} e premiamo \texttt{Invio}, 
 in essa comparirà il valore 6,
 ossia la somma dei contenuti delle celle A2 e B2.

 \begin{minipage}{\linewidth}
	\centering
    \includegraphics[scale=0.4]{path_to_image/inserimento-formula.png} 
	\captionof{figure}{Inserimento formule}
    \label{fig:esempioformule}\end{minipage}
\end{testexample}

\subsubsection{Riferimenti relativi}
Consideriamo la tabella a due colonne rappresentata nella figura \ref{fig:rifrel}. Vogliamo
eseguire il prodotto tra i dati $x$ della colonna A e i corrispondenti dati $y$ della 
colonna B, scrivendo il risultato nella colonna C. Dobbiamo allora eseguire le
seguenti operazioni:
\begin{itemize}
    \item Selezioniamo la cella C2.
    \item Digitiamo la formula = A2*B2.
    \begin{figure}[h!]
        \centering
        \includegraphics[scale=0.4]{path_to_image/riferimenti-relativi.png} 
        \caption{Riferimenti relativi.}
        \label{fig:rifrel}
    \end{figure}
    \item Premiamo \texttt{Invio} nella cella C2 compare il risultato del prodotto.
    \item Selezioniamo di nuovo la cella C2.
    \item Facciamo clic sul quadratino di trascinamento (nell'angolo in basso della cella) e,
    tenendo premuto, lo trasciniamo fino alla cella 6 (i dettagli potrebbero cambiare
    a seconda del foglio di calcolo). Otteniamo la situazione in figura \ref{fig:copyform}
    \begin{figure}[h!]
        \centering
        \includegraphics[scale=0.4]{path_to_image/copia-formula.png} 
        \caption{}
        \label{fig:copyform}
    \end{figure} 
\end{itemize}
In questo modo, appariranno i prodotti A3*B3 in C3, A4*B4 in C4 e così via. 
Osserviamo dunque che la formula è stata copiata \textit{adattando riga per riga i riferimenti} che, per questa
ragione, sono detti \textit{relativi}.
\subsubsection{Riferimenti assoluti}

Supponiamo ora di voler moltiplicare ciascun dato della colonna A 
per il coefficiente costante $k =  3$, memorizzato nella cella con riferimento E2,
 e scrivere i risultati nella colonna B.

Se seguissimo i passaggi descritti in precedenza, il foglio eseguirebbe i
 seguenti calcoli: A2*E2 in B2, A3*E3 in B3, ecc. Non è quello che vogliamo, 
 perché il secondo fattore deve rimanere sempre E2 in tutta la colonna B.
  Per ottenere questo risultato, è sufficiente ripetere le operazioni 
  descritte in precedenza con la differenza che nel passaggio 2 si deve digitare la
   formula =A2+\$E\$2.
Con l'aggiunta del simbolo \$ nella scrittura si specifica che, in tutti i prodotti,
 il riferimento al contenuto della cella E2 deve rimanere fisso, cioè \textit{assoluto}.

\subsubsection{Ricalcolo automatico}
Proviamo ora a cambiare il dato nella cella E2: inseriamo il valore
2 al posto di 3 e premiamo \texttt{Invio}. Tutti i valori contenuti nella cella B,
che si basano sul valore di E2, si modificano in modo automatico (\ref{fig:ricalcolo}).
\begin{figure}[h!]
    \centering
    \includegraphics[scale=0.4]{path_to_image/ricalcolo.png} 
    \caption{Ricalcolo istantaneo.}
    \label{fig:ricalcolo}
\end{figure} 
Il foglio ha effettuato il \textit{ricalcolo automatico}, utilizzando le formule
introdotte in precedenza nelle celle della colonna B, ma aggiornando il risultato
del prodotto.
\subsubsection{Funzioni integrate}
Un foglio di calcolo, contiene una \textit{Libreria di funzioni}, cioè Formule
predefinite che svolgono automaticamente le operazioni usate più di frequente.
Per esempio, se vogliamo sommare i numeri della colonna B della tabella in figura 
\ref{fig:somma}, possiamo usare la funzione \texttt{SOMMA}.La sintassi di una funzione
è definita dal \textit{nome} e dagli \textit{argomenti}.
\begin{testexample}

    Nella funzione \texttt{SOMMA(B1:B6)}, \texttt{SOMMA} è il nome 
    (ciò che definisce cosa fa la funzione) e \texttt{B1:B6} è \textit{l'argomento}
    (il gruppo di celle coi valori da sommare).
    
     \begin{minipage}{\linewidth}
        \centering
        \includegraphics[scale=0.4]{path_to_image/somma.png} 
        \captionof{figure}{Funzione somma.}
        \label{fig:somma}\end{minipage}
    \end{testexample}

\subsection{Grafici}
La rappresentazione grafica di una serie di dati è fondamentale per la loro interpretazione: ci permette di cogliere in modo visuale se c'è o meno una correlazione tra due grandezze e di che tipo sia.
Il foglio di calcolo permette di realizzare grafici di diversi tipi, come grafici cartesiani, grafici a barre, areogrammi e istogrammi.
Per l'analisi dei dati raccolti negli esperimenti useremo dei grafici cartesiani, che nel foglio di calcolo si chiamano Grafici a dispersione XY

\subsubsection{Grafici cartesiani}

Un grafico cartesiano rappresenta punti che corrispondono a coppie di valori e si usa per studiare l'eventuale relazione tra due grandezze.
Ecco i passaggi che dobbiamo seguire per costruire un grafico cartesiano a partire dai dati raccolti in una tabella.
\begin{enumerate}
    \item Inseriamo i dati in due colonne adiacenti, scrivendo nella prima
    riga il nome della grandezza e la sua unità di misura. 
    \item Se vogliamo inserire anche le incertezze, dobbiamo distinguere due casi: 
    \item \begin{itemize}
    \item se l'incertezza di una grandezza è la stessa per tutti i dati, 
    possiamo inserirla nella riga di intestazione;

\item se l'incertezza è diversa a seconda dei dati, aggiungiamo una colonna
che riporta l'incertezza di ogni dato.
    \end{itemize}

    \begin{testexample}

    
        
         \begin{minipage}{\linewidth}
            \centering
            \includegraphics[scale=0.2]{path_to_image/barre-libre.png} 
            \captionof{figure}{Inserimento dei dati per il grafico.}
            \label{fig:barrelibre}\end{minipage}
        \end{testexample}

\item Selezioniamo le celle che contengono i dati da rappresentare
(trascuriamo per ora l'eventuale colonna delle incertezze).
\item Seguiamo il percorso \texttt{Inserisci/Ogetto/Grafico} o selezioniamo
direttamente l'icona corrispondente a \texttt{Inserisci grafico}:
appare una finestra di dialogo in cui si può selezionare il tipo di grafico
(anche in questi casi la posizione dei comandi nei menù potrebbe cambiare in
base al programma usato.)
\item Per ottenere un grafico cartesiano, bisogna scegliere \texttt{Dispersione}.
\begin{figure}[h!]
    \centering
    \includegraphics[scale=0.3]{path_to_image/dispersione-libreoffice.png} 
    \caption{}
    \label{fig:disperslo}
\end{figure}
\item Scegliamo inoltre di rappresentare solo i punti, senza nessun  tipo di connessione
tra essi. 
\item Modifichiamo il \textit{Titolo del grafico} e aggiungiamo i \textit{Titoli degli assi},
inserendo le grandezze e le rispettive unità di misura.
\item Selezionando con il tasto destro gli assi cartesiani, possiamo inserire la
\textit{griglia secondaria}  ed eventualmente modificare l'intervallo da 
rappresentare su ogni asse.
\begin{figure}[h!]
    \centering
    \includegraphics[scale=0.3]{path_to_image/grafico-libreoffice.png} 
    \caption{Grafico a dispersione.}
    \label{fig:puntilibre}
\end{figure}
\end{enumerate}

\subsection{Elementi accessori per l'analisi dei dati}
Ai grafici è possibile aggiungere alcuni elementi che ci aiutano ad 
analizzare i dati, come le \textit{barre di errore} e le \textit{linee di tendenza}. 

\begin{definizione}
    Le barre di errore ci permettono di visualizzare in modo grafico le incertezze nelle
    misure raccolte.
\end{definizione}
Possiamo chiedere al foglio di calcolo di mostrare le barre di errore  sul grafico.

\begin{itemize}
    \item Selezioniamo i punti sul grafico con il tasto destro del mouse.
    \item Si apre un menù specifico per inserire le incertezze. Bisogna specificare
    se l'incertezza è la stessa per tutti i dati o se si vuole selezionare 
    una colonna con le diverse incertezze. 
\end{itemize}
In alternativa, è possibile inserire le barre di errore seguendo il percorso
\texttt{Struttura grafico/Aggiungi elemento  grafico/Barre d'errore.} In 
\texttt{Altre opzioni barre d'errore} è possibile stabilire nel dettaglio 
i valori da assegnare alle barre su entrambi gli assi.
\begin{remark}
    Se le barre sono troppo piccole per essere visualizzate, non le inserire
    nel grafico ma fa in modo che gli errori siano scritti altrove, per esempio nella
    tabella di raccolta e di elaborazione dei dati. Così, chi legge, può sempre
    trovare l'informazione sugli errori nella tua relazione.
\end{remark}

\begin{testexample}

    In questo grafico abbiamo impostato la stessa incertezza su tutte le misure 
    di tempo in ascissa e un'incertezza variabile per le misure in ordinata.
    
     \begin{minipage}{\linewidth}
        \centering
        \includegraphics[scale=0.3]{path_to_image/moto-carrello-libre.png} 
        \captionof{figure}{Esempio di grafico con barre di errore.}
        \label{fig:motocarrellolibre}\end{minipage}
    \end{testexample}

\subsubsection{Linea di tendenza}

\begin{definizione}
    La linea di tendenza è la curva (o la retta) che meglio approssima i dati
    rappresentati.
\end{definizione}

Possiamo chiedere al foglio elettronico di tracciare la linea di tendenza e 
scriverna l'equazione. 
\begin{enumerate}
    \item Quando il grafico è attivo, clicchiamo col tasto destro sui punti e 
    selezioniamo \texttt{Aggiungi linea di tendenza}. 
    \item Nella finestra di dialogo che si apre, scegliamo \texttt{Visualizza l'equazione sul grafico}. 
    \item Se riteniamo chela relazione sia di proporzionalità diretta (e non 
    semplicemente una relazione lineare), selezioniamo anche l'opzione 
    \texttt{Imposta intercetta 0,0}. 
\end{enumerate}

\begin{testexample}

    In questo grafico è stata selezionata una linea di tendenza senza fissare un valore
    dell'intercetta. L'equazione della retta è visualizzata sul grafico.
    
     \begin{minipage}{\linewidth}
        \centering
        \includegraphics[scale=0.2]{path_to_image/linea-tendenza-libre.png} 
        \captionof{figure}{Linea di tendenza lineare.}
        \label{fig:libretendenza}\end{minipage}
    \end{testexample}


\section{Uso dei fogli Google}
In questa sezione, impareremo come creare un grafico a dispersione utilizzando Google Sheets. Seguendo passo per passo, vedremo come inserire i dati, creare il grafico, personalizzarlo con titolo e etichette, e aggiungere una linea di tendenza. Infine, vedremo come mostrare due grafici contemporaneamente.



\subsection{Creazione del Documento}
\begin{enumerate}
    \item Apri Google Sheets e crea un nuovo foglio di calcolo.
    
\begin{figure}[h!]
    \centering
    \includegraphics[width=\linewidth]{path_to_image/creadoc.png} 
    \caption{Creazione del documento in Google Sheets}
\end{figure}    
    
    \item Rinomina il foglio in modo appropriato, ad esempio, \textit{Esperimento di Fisica} (figura \ref{fig:rinomina}).
    \begin{figure}[h!]
    \centering
    \includegraphics[width=\linewidth]{path_to_image/rinomina.png} 
    \caption{Rinomina il  documento in Google Sheets}
    \label{fig:rinomina}
\end{figure} 
\end{enumerate}


\subsection{Inserimento dei dati}
\begin{enumerate}
    \item Inserisci i dati nelle colonne A e B. Ad esempio, nella colonna A metti i valori della variabile indipendente (ad esempio, tempo in secondi), e nella colonna B i valori della variabile dipendente (ad esempio, distanza in metri) (figura \ref{fig:dati}).
    \item Digita i dati manualmente, facendo clic su una cella e digitando il valore. Premi \textbf{Invio} per passare alla cella sottostante.
    \item Esempio di dati:
    \begin{center}
    \begin{tabular}{|c|c|}
    \hline
    Tempo (s) & Distanza (m) \\
    \hline
    1 & 1,2 \\
    2 & 2,5 \\
    3 & 3,1 \\
    4 & 4,0 \\
    5 & 5,3 \\
    \hline
    \end{tabular}
    \end{center}
\end{enumerate}
    \begin{figure}[h!]
    \centering
    \includegraphics[width=\linewidth]{path_to_image/dati.png} 
    \caption{Inserimento dei dati.}
    \label{fig:dati}
\end{figure} 

Da notare che Google Sheets (``fogli di calcolo'', in italiano) utilizza le virgole per i numeri decimali (esattamente come facciamo noi). Altri programmi (e calcolatrici) invece, usano i puntini. Vi accorgete che il foglio di calcolo considera effettivamente scritte come 1,2 come numeri, perché li allinea a destra nella cella.


\subsection{Creazione del Grafico a Dispersione}
\begin{enumerate}
    \item Seleziona i dati inseriti nelle colonne A e B. Puoi farlo cliccando e trascinando il mouse dall'angolo superiore sinistro (A1) all'angolo inferiore destro (B5) dei tuoi dati (figura \ref{fig:selezione}).
     \begin{figure}[h!]
    \centering
    \includegraphics[width=\linewidth]{path_to_image/selezione.png} 
    \caption{Seleziome dei dati.}
    \label{fig:selezione}
\end{figure} 
    
    
    \item Vai su \textbf{Inserisci} nel menu e seleziona \textbf{Grafico} (figura \ref{fig:insgrafico}).
    
        \begin{figure}[h!]
    \centering
    \includegraphics[width=\linewidth]{path_to_image/inseriscigrafico.png} 
    \caption{Inserimento del grafico.}
    \label{fig:insgrafico}).
\end{figure}  
    
    
    
    \item Nella finestra delle opzioni del grafico, seleziona \textbf{Grafico a dispersione} dal menu a tendina \textbf{Tipo di grafico} (figura \ref{fig:grafico_dispersione}).
    
   \begin{figure}[h!]
    \centering
    \includegraphics[width=\linewidth]{path_to_image/dispersione.png} 
    \caption{Grafico a dispersione.}
            \label{fig:dispersione}
\end{figure} 
\end{enumerate}



\subsection{Personalizzazione del Grafico}
\begin{enumerate}
    \item Per personalizzare il grafico, devi aprire la finestra di modifica (a destra). Se non è già visibile, clicca i tre puntini sul grafico e scegli \textbf{modifica}. Se non vedi i tre puntini, clicca su un'area bianca sul grafico (non sul foglio di calcolo, proprio sul grafico):
    
    \begin{figure}[h!]
    \centering
    \includegraphics[width=\linewidth]{path_to_image/modifica.png} 
    \caption{Modifica del grafico.}
\end{figure} 
    
    \item Clicca sulla scheda \textbf{personalizza} (figura \ref{fig:personalizza})
    \begin{figure}[h!]
    \centering
    \includegraphics[width=\linewidth]{path_to_image/personalizza.png} 
    \caption{Scheda \textbf{Scheda personalizza.}}
    \label{fig:personalizza}
\end{figure} 
    
    
    \item Aggiungi il titolo del grafico: ad esempio, \textit{Relazione tra Tempo e Distanza}: espandendo il sottomenu \textbf{Titolo di grafico e assi} figura \ref{fig:titolo}:
    
 \begin{figure}[h!]
    \centering
    \includegraphics[width=\linewidth]{path_to_image/titolo.png} 
    \caption{Scheda \textbf{Modifica del titolo.}}
    \label{fig:titolo}
\end{figure} 
    
    \item Aggiungi le etichette agli assi. Seleziona dal menu a tendina \textbf{Titolo asse orizzontale: } (figura \ref{fig:sceltax})
        \begin{figure}[h!]
    \centering
    \includegraphics[width=\linewidth]{path_to_image/sceltax.png} 
    \caption{Scelta menu per il titolo dell'asse orizzontale.}
    \label{fig:sceltax}
\end{figure}  
        
        
        \item Imposta il titolo per l'asse X: \textit{Tempo (s)} (figura \ref{fig:titolox})
        
       \begin{figure}[h!]
    \centering
    \includegraphics[width=\linewidth]{path_to_image/titolox.png} 
    \caption{Scelta del titolo per l'asse x}
    \label{fig:titolox}
\end{figure} 



    \item Seleziona dal menu a tendina \textbf{Titolo asse verticale: } (figura \ref{fig:sceltay}).  
    \begin{figure}[h!]
    \centering
    \includegraphics[width=\linewidth]{path_to_image/sceltay.png} 
    \caption{Scelta menu per l'asse verticale.}
    \label{fig:sceltay}
\end{figure}  



        \item Imposta un titolo per l'asse  Y: \textit{Distanza (m)} (figura \ref{fig:titoloy})
        
    
\begin{figure}[h!]
    \centering
    \includegraphics[width=\linewidth]{path_to_image/titoloy.png} 
    \caption{Scelta del titolo per l'asse verticale.}
    \label{fig:titoloy}
\end{figure}
        

    \item Per aggiungere una linea di tendenza, espendi il sottomenu \textbf{Serie} (figura \ref{fig:serie}) e seleziona \textbf{Linea di tendenza}. Puoi scegliere il tipo di linea di tendenza (lineare, polinomiale etc.) in base ai tuoi dati. I nostri dati sembrano adattarsi ad una retta, per cui sceglieremo \textbf{Lineare}. In fine. scegliamo di mostrare l'equazione della linea di tendenza (sottomenu \textbf{Etichetta}, voce \textbf{Utilizza equazione}. Si tratta della formula che stiamo cercando, ossia il legame tra X ed Y, figura \ref{fig:tendenza}). Anticipiamo che nel caso specifico, l'equazione mostrata è la formula che lega lo spazio al tempo in un moto rettilineo uniforme. Più avanti la scriveremo nella forma:
    \[
    s=v\cdot t +s_0
    \]
    essendo $s$ la distanza percorsa ad un certo istante $t$, $v$ la velocità costante, ed $s_0$ la posizione iniziale del moto. Possiamo vedere che il corpo si è mosso ad una velocità di $\SI{0,97}{\meter\per\second}$ partendo da una posizione all'istante zero pari a $\SI{0,31}{\meter}$ (figura \ref{fig:polinomiale}). Perché è comoda questa equazione? L'equazione permette di calcolare la posizione ad un qualunque istante, anche se non presente in tabella. Ad esempio, per $t=\SI{2,9}{\second}$, abbiamo:
    
    \[
    s=\left(\SI{0,97}{\meter\per\second}\right)\times \left( \SI{2,9}{\second}\right) + \SI{0,31}{\meter}=\SI{3,12}{\meter}.
    \]
    
\begin{figure}[h!]
    \centering
    \includegraphics[width=\linewidth]{path_to_image/serie.png} 
    \caption{Scheda \textbf{serie} per la linea di tendenza.}
    \label{fig:serie}
\end{figure}    
  \begin{figure}[h!]
    \centering
    \includegraphics[width=\linewidth]{path_to_image/tendenza.png} 
    \caption{Scheda \textbf{serie} Inserimento linea di tendenza lineare (retta).}
    \label{fig:tendenza}
\end{figure}    
   In generale non è facile scegliere il tipo di linea di tendenza. In fisica, i dati possono essere legati da una relazione di proporzionalità diretta, quadratica, inversa, quadratica inversa o molte altre. Il tipo lineare è il più semplice. Quando i grafici hanno un andamento curvo, è possibile che i dati siano legati da una formula parabolica (ossia una formula del tipo $y=ax^2 +bx +c$.) In questo caso, la linea di tendenza sarà una parabola. Nei fogli google, basterà selezionare dal sottomenu la linea \textbf{Polinomiale} (figura \ref{fig:polinomiale}):
   \begin{figure}[h!]
    \centering
    \includegraphics[width=\linewidth]{path_to_image/polinomiale.png} 
    \caption{Scelta per l'inserimento della linea di tendenza polinomiale (parabola).}
    \label{fig:polinomiale}
\end{figure}     
 In generale, ti consiglio di provare varie linee di tendenza, in modo da trovare quella che meglio si adatta ai dati.    L'adattamento è tanto migliore quanto più il cosiddetto fatto $R^2$ è vicino ad uno. Per visualizzare questo fattore sul grafico, barrare la casella $\mathbf{Mostra \,\,R^2}$ come in figura \ref{fig:r2}:
 
   \begin{figure}[h!]
    \centering
    \includegraphics[width=\linewidth]{path_to_image/r2.png} 
    \caption{Visualizzazione del parametro $R^2$.}
    \label{fig:r2}
 \end{figure}
 
    
\end{enumerate}
Tutti gli elementi grafici e  testuali sono ulteriormente personalizzabili (ad esempio l'aspetto dei punti del grafico, il font per i titoli degli assi etc.). Ti consiglio di sperimentare con questi aspetti modificando a tuo piacimemnto questi elementi dalla scheda personalizzazione. Il loro uso è molto intuitivo.

\subsection{Visualizzare due Grafici Contemporaneamente}
\begin{enumerate}
    \item Inserisci una nuova serie di dati in una colonna diversa. Ad esempio, nella colonna C metti i valori della variabile dipendente di un secondo esperimento (ad esempio, distanza in metri).
    \item Esempio di dati:
    \begin{center}
    \begin{tabular}{|c|c|c|}
    \hline
    Tempo (s) & Distanza 1 (m) & Distanza 2 (m) \\
    \hline
    1 & 1,2 & 0,8 \\
    2 & 2,5 & 1,5 \\
    3 & 3,1 & 2,2 \\
    4 & 4,0 & 3,0 \\
    5 & 5,3 & 4,1 \\
    \hline
    \end{tabular}
    \end{center}
    \item Seleziona i dati inseriti nelle colonne A, B e C. Puoi farlo cliccando e trascinando il mouse dall'angolo superiore sinistro (A1) all'angolo inferiore destro (C5) dei tuoi dati oppure (vedi figura \ref{fig:intervallo2}) selezionando \textbf{aggiungi serie}, figura \ref{fig:intervallo2} e. successivamente, cliccando il tasto ``+''. Si aprirà un campo di testo in cui inserire l'intervallo di cella. Per farlo, si può agire graficamente, selezionando tutta la colonna col tasto destro del mouse e verrà automaticamente scritto l'intervallo (figura \ref{fig:intervallo2-dialogo}. La scrittura \textit{Foglio1!C1:C6} indica di selezionare l'intervallo del foglio attuale (Foglio1) dalla cella C1 alla cella C6). In fine clicca ``Ok'' e la serie verrà inserita. A quesato punto puoi personalizzare il grafico come vuoi, inserire linee di tendenza etc. Come regola generale, si consiglia di realizzare grafici in cui le due serie sullì'asse Y sono della stessa natura ma non è vietato inserire anche serie diverse purché, è bene ribadirlo, sull'asse X, ci sia una sola serie per entrambe le grandezze dipendenti (ossia le Y, nel nostro caso, il tempo). Come esercizio, puoi provare a disegnare la linea di tendenza della serie 2. Il grafico finale è in figura \ref{fig:finale}.
    
      \begin{figure}[h!]
    \centering
    \includegraphics[width=\linewidth]{path_to_image/intervallo2.png} 
    \caption{Selezione di una serie, tasto ``+''.}
    \label{fig:intervallo2}
 \end{figure} 
    
    
\begin{figure}[h!]
    \centering
    \includegraphics[width=\linewidth]{path_to_image/intervallo2-dialogo.png} 
    \caption{Finestra di dialogo per la scelta dell'intervallo della serie 2.}
    \label{fig:intervallo2-dialogo}
 \end{figure} 

\begin{figure}[h!]
    \centering
    \includegraphics[width=\linewidth]{path_to_image/finale.png} 
    \caption{Come si presenta il grafico costruito nel tutorial.}
    \label{fig:finale}
 \end{figure} 
\end{enumerate}





\subsection{Conclusione}
Seguendo questi passaggi, sarai in grado di creare e personalizzare grafici a dispersione su Google Sheets per analizzare e visualizzare i tuoi dati sperimentali. Questa competenza è utile non solo per le lezioni di fisica, ma anche per altre discipline scientifiche.

\chapter{Relazioni di Laboratorio}
In questo capitolo, illustreremo la metodica per realizzare una relazione di laboratorio di fisica. L'esempio svolto sarà abbastanza completo ma si tenga conto del fatto che, a seconda degli argomenti e del tempo a disposizione,  una relazione reale potrebbe essere diversa per varie ragioni (non sempre si realizzano grafici o si calcolano gli errori come abbiamo fatto). Nel caso trattato, abbiamo usato una tecnica (l'uso della linea di tendenza o di \textit{bestfit}) molto efficace quando si ha la possibilità di misurare la variazione di due grandezze correlate (si può usare a tal proposito il foglio di calcolo o un linguaggio di programmazione con python). Alcune misure però, specialmente le prime che svolgeremo, richiedono solo misure dirette e lo scopo della relazione sarà la misura di una o più grandezze e la loro  scrittura corretta (quella del tipo $x=\left( \overline{x} \pm \Delta x\right)$ per intenderci).
\section{Titolo e Scopo dell'Esperienza}
Il titolo deve essere chiaro e conciso. Lo scopo dell'esperienza deve spiegare brevemente l'obiettivo dell'esperimento.

\section{Materiali e Strumenti}
Elencare tutti i materiali e gli strumenti utilizzati, specificando la sensibilità e la portata degli strumenti di misura. Includere un'immagine dell'apparato sperimentale.

\section{Raccolta Dati}
Presentare i dati raccolti in forma tabulare, includendo gli errori nelle misure dirette.

\section{Cenni Teorici}
Esporre brevemente la teoria relativa all'esperimento, includendo le formule rilevanti.

\section{Elaborazione Dati}
Descrivere il processo di elaborazione dei dati raccolti.

\section{Calcolo degli Errori nelle Misure Indirette}
Spiegare come si propagano gli errori nelle misure indirette.

\section{Grafico Sperimentale}
Presentare i dati in forma grafica, includendo le barre di errore.

\section{Conclusioni}
Discutere i risultati ottenuti, confrontandoli con le previsioni teoriche.

\newpage

\section{Esempio: Studio del Moto}

\subsection{Titolo e Scopo dell'Esperienza}
\textbf{Titolo:} Studio del Moto Uniformemente Accelerato\\
\textbf{Scopo:} Verificare sperimentalmente le leggi del moto uniformemente accelerato.

\subsection{Materiali e Strumenti}
\begin{itemize}
    \item Piano inclinato (base \SI{2000}{\milli\metre}) e inclinato di \SI{30}{\degree}.
    \item Sfera metallica.
    \item Cronometro digitale (sensibilità: \SI{0,01}{\second}, portata: \SI{999,99}{\second}).
    \item Metro a nastro (sensibilità: \SI{1}{\milli\metre}, portata: \SI{3000}{\milli\metre}).
    \item Goniometro (sensibilità: \SI{1}{\degree}, portata: \SI{180}{\degree}).
\end{itemize}

\begin{figure}[!htbp] 
    \centering
    \begin{tikzpicture}[scale=0.7]
        % Piano inclinato
        \draw[thick] (0,0) -- (10,5);
        \draw[thick] (0,0) -- (10,0);
        
        % Sfera
        \fill[gray] (3,1.8) circle (0.3);
        
        % Cronometro
        \draw[rounded corners] (11,4) rectangle (13,6);
        \node at (12,5) {00:00};
        
        % Metro a nastro
        \draw[thick] (0,-0.5) -- (10,-0.5);
        \foreach \x in {0,1,...,10}
            \draw (\x,-0.7) -- (\x,-0.5);
        \node at (5,-1) {Metro a nastro};
        
        % Goniometro
        \draw (1.1,0) arc (0:30:1);
        \node at (1.8,0.3) {$\SI{30}{\celsius}$};
        
        % Etichette
        \node[below] at (5,0) {2.00 m};
        \node[above right] at (3,2.1) {Sfera};
        \node[above] at (12,6) {Cronometro};
    \end{tikzpicture}
    \caption{Apparato sperimentale per lo studio del moto uniformemente accelerato.}
    \label{fig:apparato}
\end{figure}

\subsection{Raccolta Dati}
Abbiamo misurato il tempo impiegato dalla sfera per percorrere diverse distanze lungo il piano inclinato, inclinato di \SI{30}{\degree} rispetto all'orizzontale.

\begin{table}[!htbp] 
    \centering
    \begin{tabular}{@{}ccc@{}}
        \toprule
        Distanza (\si{\metre}) & Tempo (\si{\second}) & Errore sul tempo (\si{\second}) \\
        \midrule
        0,200 & 0.29 & 0,01 \\
        0,400 & 0.41 & 0,01 \\
        0,600 & 0.50 & 0,01 \\
        0,800 & 0.58 & 0,01 \\
        1,000 & 0.65 & 0,01 \\
        \bottomrule
    \end{tabular}
    \caption{Dati sperimentali}
    \label{tab:datitab}
\end{table}

\subsection{Cenni Teorici}
Nel moto uniformemente accelerato, la posizione $s$ in funzione del tempo $t$ è data da:

\begin{equation}
    s = \frac{1}{2}at^2
\end{equation}

dove $a$ è l'accelerazione. Nel nostro caso, l'accelerazione lungo il piano inclinato è data da:

\begin{equation}
    a = g\sin\theta
\end{equation}

dove $g$ è l'accelerazione di gravità e $\theta$ è l'angolo di inclinazione del piano.

\subsection{Elaborazione Dati}
Per verificare la relazione $s = \frac{1}{2}at^2$, plotteremo $s$ in funzione di $t^2$.
 La pendenza della retta di best fit sarà $\frac{1}{2}a$. Prima di realizzare il grafico,
 calcoliamo anzitutto gli errori e poi metteremo i punti sul piano cartesiano. In fine calcoleremo
 l'accelerazione di gravità dalla pendenza. I dettagli per farlo,sono riassunti nei paragrafi
 sull'uso del foglio di calcolo (o con la procedura manuale) e l'uso di python con la libreria 
 \texttt{matplotlib}.

\subsection{Calcolo degli Errori nelle Misure Indirette}
L'errore su $t^2$ si propaga come:

\begin{equation}
    \Delta(t^2) =\overline{t^2} \left(2\cdot\frac{\Delta t}{\overline{t}}  \right)= 2\overline{t}\Delta t.
\end{equation}
Con questa formula, calcoliamo gli errori su $t^2$ e li mettiamo in tabella \ref{tab:tquadro}:

\begin{table}[h!]
\centering
\begin{tabular}{|c|c|}
\hline
$t^2$ (\si{\square\s}) & $\Delta \left( t^2 \right)$ (\si{\square\s}) \\
\hline
0.0841 & 0.006 \\
0.1681 & 0.008 \\
0.25   & 0.01   \\
0.3364 & 0.01 \\
0.4225 & 0.01  \\
\hline
\end{tabular}
\caption{Valori calcolati di $t^2$ e $\Delta t^2$}
\label{tab:tquadro}
\end{table}

\subsection{Grafico Sperimentale}

\begin{figure}[!htbp] 
    \centering
    \begin{tikzpicture}
        \begin{axis}[
            width=0.8\textwidth,
            height=0.6\textwidth,
            xlabel={$t^2$ (\si{\second\squared})},
            ylabel={$s$ (\si{\metre})},
            xmin=0, xmax=0.5,
            ymin=0, ymax=1.2,
            legend pos=north west,
            ]
            \addplot[only marks,error bars/.cd,x dir=both,x explicit,y dir=both,y explicit]
            coordinates {
                (0.0841, 0.20) +- (0.0058, 0.001)
                (0.1681, 0.40) +- (0.0082, 0.001)
                (0.2500, 0.60) +- (0.0100, 0.001)
                (0.3364, 0.80) +- (0.0116, 0.001)
                (0.4225, 1.00) +- (0.0130, 0.001)
            };
            \addplot[domain=0:0.5, samples=100, smooth, thick, red] {2.3604*x};
            \legend{Dati sperimentali, Fit lineare}
        \end{axis}
    \end{tikzpicture}
    \caption{Grafico di $s$ in funzione di $t^2$.}
    \label{fig:grafico}
\end{figure}
Il grafico con le barre di errore è stato ottenuto con un software perché non è possibile realizzarlo coi fogli google (non si possono inserire barre di errore personalizzate ad esempio). Ovviamente si può realizzare un ottimo grafico manuale con le tecniche esposte nei precedenti paragrafi.
\subsection{Conclusioni}
Dal fit lineare dei dati, abbiamo ottenuto una pendenza di \SI{2.3604}{\metre\per\second\squared}, che corrisponde a $\frac{1}{2}a$. La pendenza si misura sul grafico tracciando la retta a mano  oppure, se si è realizzato il grafico col foglio di calcolo, basta fargli scrivere l'equazione che sarà del tipo $y= bx +a$, dove $b$ è la pendenza. Quindi, l'accelerazione sperimentale è:

\begin{equation}
    a_{\text{exp}} = \SI{4.7208}{\metre\per\second\squared}
\end{equation}

Teoricamente, l'accelerazione dovrebbe essere:

\begin{equation}
    a_{\text{th}} = g\sin\theta = \SI{9.81}{\metre\per\second\squared} \cdot \sin(\SI{30}{\degree}) = \SI{4.905}{\metre\per\second\squared}
\end{equation}

La differenza relativa tra il valore sperimentale e quello teorico è:

\begin{equation}
    \frac{|a_{\text{exp}} - a_{\text{th}}|}{a_{\text{th}}} \cdot 100\% = \frac{|\SI{4.7208}{\metre\per\second\squared} - \SI{4.905}{\metre\per\second\squared}|}{\SI{4.905}{\metre\per\second\squared}} \cdot 100\% \approx 3.75\%
\end{equation}

Notiamo che non abbiamo potuto fare il confronto col metodo degli errori (sezione 2.4) perché non abbiamo calcolato l'errore sull'accelerazione e sul valore teorico atteso (sarebbe stato troppo complicato). Comunque, questa discrepanza del 3.75 \% tra il valore sperimentale e quello teorico è accettabile per un esperimento di laboratorio di fisica elementare. Le possibili fonti di errore includono:

\begin{itemize}
    \item Piccoli errori nella misurazione del tempo.
    \item Leggero attrito tra la sfera e il piano inclinato.
    \item Possibile imprecisione nella misurazione dell'angolo di inclinazione.
    \item Errori di parallasse nella lettura delle distanze.
\end{itemize}

Per migliorare ulteriormente l'esperimento, si potrebbe:
\begin{itemize}
    \item Utilizzare un sistema di rilevamento del tempo più preciso, come un timer fotoelettrico.
    \item Ridurre l'attrito utilizzando una superficie più liscia o una sfera con minore attrito.
    \item Ripetere le misurazioni più volte per ogni distanza per ridurre gli errori casuali.
\end{itemize}

In conclusione, l'esperimento ha confermato con buona approssimazione la relazione tra spazio e tempo nel moto uniformemente accelerato, dimostrando l'efficacia del metodo sperimentale nell'investigare le leggi della fisica.

%% ESEMPIO COMPLETO
\section{Caduta lungo un piano inclinato}
\subsection{Obiettivo}
L'obiettivo di questo esperimento è dimostrare la relazione di proporzionalità tra massa e accelerazione utilizzando un piano inclinato con altezza variabile.

\subsection{Materiali}
\begin{itemize}
    \item Piano inclinato di lunghezza $L = \SI{1,000}{\meter}$
    \item Pallina di massa $m = \SI{23,5}{\gram}$
    \item Spessori da inserire sotto uno dei due lati del piano
\end{itemize}

\subsection{Strumenti}
\begin{itemize}
    \item Cronometro con sensibilità \SI{0,01}{\second}
    \item Bilancia con sensibilità \SI{0,1}{\gram}
    \item Metro da falegname con sensibilità \SI{1}{\milli\meter}
\end{itemize}

\subsection{Procedura}
\begin{enumerate}
    \item Posizionare il piano inclinato e misurare la sua lunghezza $L$.
    \item Variare l'altezza $h$ del piano inserendo degli spessori.
    \item Per ogni configurazione, far rotolare la pallina e misurare il tempo di caduta tre volte.
    \item Calcolare il tempo medio di caduta.
    \item Calcolare l'accelerazione usando la formula $a = \frac{2L}{t^2}$.
    \item Calcolare la forza parallela $F_\parallel$ usando la formula $F_\parallel = m g \frac{h}{L}$.
    \item Ripetere i passaggi 2-6 per diverse altezze del piano.
\end{enumerate}

\subsection{Dati e Calcoli}

\begin{table}[h]
\centering
\caption{Misurazioni e Calcoli}
\begin{tabular}{ccccccc}
\toprule
$h$ (m) & $t_1$ (s) & $t_2$ (s) & $t_3$ (s) & $t_{medio}$ (s) & $a$ (m/s$^2$) & $F_\parallel$ (N) \\
\midrule
\multirow{2}{*}{0,100} & 1,41 & 1,43 & 1,42 & 1,42 & 0,99 & 0,0230 \\
& $\pm 0,001$ & $\pm 0,01$ & $\pm 0,01$ & $\pm 0,01$ & $\pm 0,02$ & $\pm 0,0002$ \\
\midrule
\multirow{2}{*}{0,150} & 1,16 & 1,15 & 1,17 & 1,16 & 1,49 & 0,0346 \\
& $\pm 0,001$ & $\pm 0,01$ & $\pm 0,01$ & $\pm 0,01$ & $\pm 0,03$ & $\pm 0,0003$ \\
\midrule
\multirow{2}{*}{0,200} & 1,00 & 1,01 & 0,99 & 1,00 & 2,00 & 0,0461 \\
& $\pm 0,001$ & $\pm 0,01$ & $\pm 0,01$ & $\pm 0,01$ & $\pm 0,04$ & $\pm 0,0005$ \\
\midrule
\multirow{2}{*}{0,250} & 0,89 & 0,90 & 0,88 & 0,89 & 2,52 & 0,0576 \\
& $\pm 0,001$ & $\pm 0,01$ & $\pm 0,01$ & $\pm 0,01$ & $\pm 0,06$ & $\pm 0,0006$ \\
\midrule
\multirow{2}{*}{0,300} & 0,82 & 0,81 & 0,83 & 0,82 & 2,98 & 0,0691 \\
& $\pm 0,001$ & $\pm 0,01$ & $\pm 0,01$ & $\pm 0,01$ & $\pm 0,07$ & $\pm 0,0007$ \\
\bottomrule
\end{tabular}
\end{table}

\subsubsection{Esempi di Calcolo}

Di seguito sono riportati alcuni esempi di calcolo per illustrare il processo di analisi dei dati:

\subsubsection{Calcolo del Tempo Medio e Semidispersione}

Prendiamo come esempio i dati per $h = 0,200$ m:

\begin{align*}
t_1 &= 1,00 \, \text{s} \\
t_2 &= 1,01 \, \text{s} \\
t_3 &= 0,99 \, \text{s}
\end{align*}

Il tempo medio è:
\[ t_{medio} = \frac{t_1 + t_2 + t_3}{3} = \frac{1,00 + 1,01 + 0,99}{3} = 1,00 \, \text{s} \]

La semidispersione è:
\[ \Delta t = \frac{t_{max} - t_{min}}{2} = \frac{1,01 - 0,99}{2} = 0,01 \, \text{s} \]

Quindi, il risultato arrotondato alla prima cifra significativa dell'errore è:

\[ t = (1,00 \pm 0,01) \, \text{s} \]

\subsubsection{Calcolo dell'Errore sulla Forza Parallela}

L'errore sulla forza parallela è dato da:

\[ \Delta F_\parallel = F_\parallel \left( \frac{\Delta h}{h} + \frac{\Delta L}{L} + \frac{\Delta m}{m} \right) \]

dove $F_\parallel = m g \frac{h}{L}$, $m = 23,5 \, \text{g} = 0,0235 \, \text{kg}$, $g = 9,81 \, \text{m/s}^2$, $L = 1,000 \, \text{m}$, $\Delta h = 0,001 \, \text{m}$, $\Delta L = 0,001 \, \text{m}$, e $\Delta m = 0,0001 \, \text{kg}$.

Prendiamo come esempio il caso con $h = 0,200 \, \text{m}$:

\begin{align*}
F_\parallel &= 0,0235 \, \text{kg} \times 9,81 \, \text{m/s}^2 \times \frac{0,200 \, \text{m}}{1,000 \, \text{m}} = 0,0461 \, \text{N} \\[10pt]
\Delta F_\parallel &= 0,0461 \, \text{N} \left( \frac{0,001 \, \text{m}}{0,200 \, \text{m}} + \frac{0,001 \, \text{m}}{1,000 \, \text{m}} + \frac{0,0001 \, \text{kg}}{0,0235 \, \text{kg}} \right) \\[10pt]
&= 0,0461 \, \text{N} \times (0,005 + 0,001 + 0,00426) \\[10pt]
&= 0,0461 \, \text{N} \times 0,01026 \\[10pt]
&= 0,000473 \, \text{N}
\end{align*}

Arrotondando alla prima cifra significativa dell'errore, otteniamo:

\[ F_\parallel = (0,0461 \pm 0,0005) \, \text{N} \]

\subsubsection{Calcolo dell'Errore sull'Accelerazione}

L'errore sull'accelerazione è dato da:

\[ \Delta a = \frac{2L}{t^2}\left( \frac{\Delta L}{L} + 2\frac{\Delta t}{t}\right) \]

dove $L = 1,000 \, \text{m}$, $\Delta L = 0,001 \, \text{m}$, $t = 1,00 \, \text{s}$, e $\Delta t = 0,01 \, \text{s}$.

Calcoliamo:

\begin{align*}
\Delta a &= \frac{2 \times 1,000 \, \text{m}}{(1,00 \, \text{s})^2}\left( \frac{0,001 \, \text{m}}{1,000 \, \text{m}} + 2\frac{0,01 \, \text{s}}{1,00 \, \text{s}}\right) \\[10pt]
&= 2,00 \, \text{m/s}^2 \times (0,001 + 0,02) \\[10pt]
&= 2,00 \, \text{m/s}^2 \times 0,021 \\[10pt]
&= 0,042 \, \text{m/s}^2
\end{align*}

Arrotondando alla prima cifra significativa dell'errore, otteniamo:

\[ a = (2,00 \pm 0,04) \, \text{m/s}^2 \]
\subsection{Grafico}

\begin{figure}[h]
\centering
\includegraphics[width=1.2\textwidth]{nome_del_file_grafico.png}
\caption{Relazione tra Forza Parallela e Accelerazione}
\label{fig:grafico}
\end{figure}

\subsection{Analisi del Fit Lineare}

Il grafico mostra una chiara relazione lineare tra la forza parallela ($F_\parallel$) e l'accelerazione ($a$). Abbiamo effettuato un fit lineare dei dati, ottenendo la seguente equazione:

\[ F_\parallel = (0,0230 \pm 0,0002) \, \text{kg} \cdot a + (0,0002 \pm 0,0003) \, \text{N} \]

dove $F_\parallel$ è espresso in N e $a$ in m/s$^2$.

\paragraph{Interpretazione dei Risultati:}

\begin{itemize}
    \item \textbf{Pendenza:} La pendenza della retta di best-fit rappresenta la massa dell'oggetto in movimento. Dal fit, otteniamo una massa di $(23,0 \pm 0,2) \, \text{g}$.
    
    \item \textbf{Confronto con la Massa Utilizzata:} La massa reale della pallina utilizzata nell'esperimento era di $23,5 \, \text{g}$. Osserviamo un'eccellente concordanza tra questo valore e quello ottenuto dal fit, con una differenza che rientra ampiamente nell'errore sperimentale.
    
    \item \textbf{Intercetta:} L'intercetta $(0,0002 \pm 0,0003) \, \text{N}$ è compatibile con zero entro l'errore sperimentale, come ci si aspetterebbe da un sistema ideale dove non ci sono forze aggiuntive quando l'accelerazione è nulla.
    
    \item \textbf{Coefficiente di Determinazione:} Il valore di $R^2 = 0,9996$ indica un'eccellente correlazione lineare tra la forza parallela e l'accelerazione, confermando la validità del modello lineare per descrivere la relazione tra queste due grandezze.
\end{itemize}

Questa analisi conferma la validità del nostro esperimento e l'ottima corrispondenza tra i risultati sperimentali e le previsioni teoriche della seconda legge di Newton.

\subsection{Conclusioni}

L'esperimento condotto ha permesso di studiare la relazione tra massa e accelerazione utilizzando un piano inclinato. Dall'analisi dei dati e del fit lineare, emergono le seguenti conclusioni:

\begin{enumerate}
    \item La relazione tra forza parallela e accelerazione mostra una chiara linearità, confermando quantitativamente la seconda legge di Newton ($F = ma$).
    
    \item La pendenza della retta di best-fit, che teoricamente rappresenta la massa dell'oggetto, risulta essere $(23,0 \pm 0,2) \, \text{g}$, in eccellente accordo con la massa reale della pallina $(23,5 \, \text{g})$.
    
    \item La minima discrepanza osservata può essere attribuita a fattori come errori di misura e incertezze strumentali, ma è notevolmente piccola considerando la natura dell'esperimento.
    
    \item L'intercetta prossima a zero conferma l'assenza di forze significative quando l'accelerazione è nulla, in accordo con le aspettative teoriche.
    
    \item L'elevato valore del coefficiente di determinazione (R² = 0,9996) sottolinea l'eccellente correlazione lineare tra le variabili misurate, validando ulteriormente il modello teorico.
\end{enumerate}

Nonostante l'ottima corrispondenza tra i risultati sperimentali e le previsioni teoriche, ci sono sempre margini di miglioramento per future iterazioni dell'esperimento:

\begin{itemize}
    \item Utilizzare strumenti di misura ancora più precisi, specialmente per la misurazione del tempo.
    \item Considerare e quantificare gli effetti dell'attrito nel modello teorico, anche se sembrano essere minimi in questo caso.
    \item Ripetere l'esperimento con diverse masse per verificare la coerenza dei risultati su un range più ampio.
    \item Effettuare una calibrazione ancora più accurata di tutti gli strumenti prima dell'esperimento.
\end{itemize}

In conclusione, l'esperimento ha confermato con notevole precisione la relazione lineare tra forza e accelerazione prevista dalla seconda legge di Newton. La qualità dei risultati ottenuti testimonia l'accuratezza delle misurazioni e l'efficacia del setup sperimentale utilizzato.


\chapter{Guida linguaggio python}



\section{Introduzione}
Python è un linguaggio di programmazione versatile e potente, ideale per iniziare a programmare e per l'analisi dati. In questa guida, esploreremo la sintassi base, l'ambiente di sviluppo Google Colab, e alcuni elementi chiave di Python, NumPy e Matplotlib. In queste pagine sono inseriti i listati degli script python che uderemo nell'analisi dei dati. Si tratta di comandi del linguaggio  evidenziati da un bordo grigio. I listati non si possono copiare da questi appunti, sono disponibili in una cartella sul drive. Alcuni caratteri, richiedono l'uso combinato di più tasti (ad esempio, il carattere ''\textasciicircum`` sulla mia tastiera, si ottiene premendo contemporaneamente shift (una freccia spessa verso l'alto che si trova sulla destra della tastiera) e il tasto ì accentato (in alto a destra)). Per questi simboli, occorre imparare le opportune combinazioni che cambiano da tastiera a tastiera. Siccome si tratta solo di caratteri estetici, potete sostituirli come vi pare negli script (questi lavorano solo con numeri e lettere, i simboli non sono importanti ma dovendo inserire le unità di misura bisogna un pò ingegnarsi).
 
\section{Sintassi Base e Ambiente di Sviluppo Google Colab}
Google Colab è un ambiente di sviluppo integrato (IDE) basato su Jupyter Notebook, che permette di eseguire codice Python direttamente nel browser.

\subsection{Accedere a Google Colab}
1. Apri il browser e vai su \url{https://colab.research.google.com/}.
2. Accedi con il tuo account Google.
3. Crea un nuovo notebook cliccando su \textit{New Notebook}.

\subsection{Scrivere ed Eseguire Codice}
Nel notebook, puoi scrivere ed eseguire codice Python in celle. Per eseguire una cella, premi \textit{Shift+Enter}.

Esempio:
\begin{lstlisting}[language=Python]
print("Hello, World!")
\end{lstlisting}

\section{Elementi del Linguaggio Python}

\subsection{Variabili e Oggetti}
Le variabili in Python non richiedono dichiarazioni esplicite del tipo. Esempio:
\begin{lstlisting}[language=Python]
x = 10
y = "Ciao"
z = 3.14
\end{lstlisting}

\subsection{Stringhe}
Le stringhe sono delimitate da virgolette singole o doppie. Esempio:
\begin{lstlisting}[language=Python]
stringa = "Questa è una stringa"
\end{lstlisting}

\subsubsection{F-String}
Le f-string sono una funzionalità di Python che permette di inserire variabili all'interno delle stringhe in modo semplice e leggibile. Si utilizzano anteponendo una `f` alla stringa e racchiudendo le variabili tra parentesi graffe `{}`.

Esempio:
\begin{lstlisting}[language=Python]
nome = "Alice"
eta = 25
stringa = f"Il nome è {nome} e l'età è {eta} anni."
print(stringa)
\end{lstlisting}

In questo esempio, le variabili `nome` e `eta` sono incluse direttamente all'interno della stringa senza la necessità di concatenare manualmente o utilizzare metodi di formattazione complessi.

Esempio con espressioni:
\begin{lstlisting}[language=Python]
a = 10
b = 5
stringa = f"La somma di {a} e {b} è {a + b}."
print(stringa)
\end{lstlisting}

In questo caso, anche le espressioni possono essere valutate direttamente all'interno delle parentesi graffe.

\subsection{Costrutto di Selezione}
Il costrutto di selezione in Python usa le parole chiave \textit{if}, \textit{elif} e \textit{else}. Esempio:
\begin{lstlisting}[language=Python]
x = 5
if x > 0:
    print("x è positivo")
elif x == 0:
    print("x è zero")
else:
    print("x è negativo")
\end{lstlisting}

\subsection{Controllo di Flusso}
I controlli di flusso includono strutture come cicli e dichiarazioni condizionali.

\subsubsection{Indentazione}
Python utilizza l'indentazione per definire blocchi di codice. Ogni blocco deve essere indentato con lo stesso numero di spazi o tabulazioni. Esempio:
\begin{lstlisting}[language=Python]
x = 10
if x > 0:
    print("x è positivo")
    if x > 5:
        print("x è maggiore di 5")
\end{lstlisting}

\subsubsection{Cicli \textit{for}}
Il ciclo \textit{for} in Python è estremamente versatile e può essere utilizzato per iterare su una varietà di sequenze e strutture dati.

\paragraph{Iterare su una Lista}
Il ciclo \textit{for} può iterare attraverso ogni elemento di una lista. Esempio:
\begin{lstlisting}[language=Python]
frutti = ['mela', 'banana', 'ciliegia']
for frutto in frutti:
    print(frutto)
\end{lstlisting}

\paragraph{Iterare su un Range di Numeri}
Il ciclo \textit{for} può iterare attraverso un range di numeri generato dalla funzione \textit{range}. Esempio:
\begin{lstlisting}[language=Python]
for i in range(5):
    print(i)
\end{lstlisting}

\paragraph{Iterare su un Insieme}
Gli insiemi (\textit{sets}) sono collezioni non ordinate di elementi unici. Esempio:
\begin{lstlisting}[language=Python]
insieme = {1, 2, 3, 4}
for numero in insieme:
    print(numero)
\end{lstlisting}

\paragraph{Iterare su un Dizionario}
I dizionari (\textit{dict}) sono collezioni di coppie chiave-valore. Esempio di iterazione su chiavi e valori:
\begin{lstlisting}[language=Python]
dizionario = {'nome': 'Alice', 'età': 25, 'città': 'Roma'}

# Iterare sulle chiavi
for chiave in dizionario:
    print(chiave, dizionario[chiave])

# Iterare su chiavi e valori
for chiave, valore in dizionario.items():
    print(chiave, valore)
\end{lstlisting}

\paragraph{Enumerare gli Elementi di una Sequenza}
La funzione \textit{enumerate} restituisce una coppia (indice, valore) per ogni elemento di una sequenza. Esempio:
\begin{lstlisting}[language=Python]
frutti = ['mela', 'banana', 'ciliegia']
for indice, frutto in enumerate(frutti):
    print(indice, frutto)
\end{lstlisting}

\paragraph{Uso di \textit{zip} per Iterare su Più Sequenze}
La funzione \textit{zip} può essere utilizzata per iterare su più sequenze contemporaneamente. Esempio:
\begin{lstlisting}[language=Python]
nomi = ['Alice', 'Bob', 'Charlie']
eta = [25, 30, 35]

for nome, eta_persona in zip(nomi, eta):
    print(nome, eta_persona)
\end{lstlisting}

\subsubsection{Cicli \textit{while}}
Il ciclo \textit{while} itera finché una condizione è vera. È utile quando non si conosce in anticipo il numero di iterazioni. Ecco alcuni esempi:

\paragraph{Esempio Base}
Un ciclo `while` di base per contare da 0 a 4:
\begin{lstlisting}[language=Python]
count = 0
while count < 5:
    print(count)
    count += 1
\end{lstlisting}

\paragraph{Esempio con Lista}
In questo esempio, gli elementi della lista vengono rimossi uno alla volta fino a che la lista è vuota:
\begin{lstlisting}[language=Python]
lista = [1, 2, 3, 4, 5]
while lista:
    elemento = lista.pop(0)
    print(f"Eliminato: {elemento}, Lista rimanente: {lista}")
\end{lstlisting}

\section{Elementi di NumPy}

\subsection{Array}
NumPy è una libreria fondamentale per il calcolo scientifico in Python. Gli array NumPy sono simili alle liste, ma permettono operazioni matematiche vettorizate. Esempio:
\begin{lstlisting}[language=Python]
import numpy as np
arr = np.array([1, 2, 3, 4])
print(arr)
\end{lstlisting}

\subsection{Generazione di Array}
NumPy fornisce diverse funzioni per creare array. Esempio:
\begin{lstlisting}[language=Python]
zeros = np.zeros(5)
ones = np.ones(5)
range_arr = np.arange(10)
\end{lstlisting}

\subsection{Indicizzazione degli Array}
Gli elementi di un array possono essere accessi e modificati utilizzando gli indici. Esempio:
\begin{lstlisting}[language=Python]
arr = np.array([1, 2, 3, 4, 5])
print(arr[0])  # Primo elemento
print(arr[-1])  # Ultimo elemento
arr[0] = 10  # Modifica il primo elemento
\end{lstlisting}

\subsection{Matematica con gli Array}
NumPy permette di eseguire operazioni matematiche sugli array in modo semplice. Esempio:
\begin{lstlisting}[language=Python]
arr = np.array([1, 2, 3, 4])
print(arr + 2)
print(arr * 2)
print(np.sqrt(arr))
\end{lstlisting}

\subsection{Principali Funzioni Matematiche di NumPy}
NumPy offre molte funzioni matematiche utili per operare sugli array. Ecco alcune delle principali:

\subsubsection{Funzioni Trigonometriche}
\begin{lstlisting}[language=Python]
import numpy as np
angles = np.array([0, np.pi/2, np.pi])
print(np.sin(angles))  # Calcola il seno
print(np.cos(angles))  # Calcola il coseno
print(np.tan(angles))  # Calcola la tangente
\end{lstlisting}

\subsubsection{Funzioni Esponenziali e Logaritmiche}
\begin{lstlisting}[language=Python]
values = np.array([1, 2, 3])
print(np.exp(values))  # Calcola l'esponenziale
print(np.log(values))  # Calcola il logaritmo naturale
print(np.log10(values))  # Calcola il logaritmo in base 10
\end{lstlisting}



\subsubsection{Altre Funzioni Utili}
\begin{lstlisting}[language=Python]
values = np.array([-1, 2, -3])
print(np.abs(values))  # Valore assoluto
print(np.sqrt(values + 4))  # Radice quadrata (aggiungiamo 4 per evitare valori negativi)
print(np.sum(values))  # Somma degli elementi
print(np.prod(values))  # Prodotto degli elementi
\end{lstlisting}

\subsection{Propagazione degli errori massimi}
In questa sezione, mostriamo tre esempi di codice python per il calcolo degli errori propagati in fisica. Come per gli esempi precedenti, la propagazione avverrà, appunto, con le due regole che abbiamo già visto. Se qualcuno di voi cercherà su internet questo argomento, troverà  anche un tipo di progazione che usa la cosiddetta \text{somma in quadratura}, una tecnica usata quando le grandezze iniziali sono affette da errore casuale (vedremo più avanti di cosa si tratta ma anticipiamo che in questo corso, la propagazione degli errori statistici non verrà adottata).


\begin{testexample}[ \thetcbcounter \, Errore sulla velocità media]
Un carrellino percorre una distanza s = (1.222 +- 0,001)m in un tempo t = (0,8730 +- 0,0001) s. Determina la migliore stima della velocità e il suo errore assoluto.

\begin{minted}{python}
s=1.222
Ds=0.001
t=0.8730
Dt=0.0001
v=s/t
Dv=v*(Ds/s +Dt/t)
f"v= {v} +-  {Dv} m/s"
\end{minted}
L'output del codice è il seguente:
\begin{verbatim}
v= 1.399770904925544 +-  0.0013058156821220556 m/s
\end{verbatim}
Il valore scritto in modo formale è: v = (1,400 +- 0,001)m/s
\end{testexample}

\begin{testexample}[\thetcbcounter \,Errore sulla densità]
Un blocchetto di ferro di massa $m = (450,0 +- 0,1) g$, viene inserito in un cilindro graduato con dell'acqua all'interno. L'acqua inizialmente raggiunge il livello $V1 = ( 730 +- 1) cm^3$ mentre, dopo l'inserimento del blocchetto, raggiunge il volume $V2= (810 +-1 ) cm^3$. Determina la densità del cubetto in $g/cm^3$.
\begin{minted}{python}
m=450
Dm=0.1
V1=730
DV1=1
V2=810
DV2=1
V=V2-V1
\end{minted}
La densità vale $d = m/V$, pertanto il suo errore è $Dd = d*(Dm/m + DV/v)$ . L'errore assoluto sul volume però, è uguale alla somma degli errori assoluti di V1 e V2, pertanto, lo indichiamo con $DV = DV1+DV2$. Procediamo:
\begin{minted}{python}
DV=DV1+DV2
d=m/V
Dd=d*(Dm/m+DV/V)
print(f"d = ({d} +- {d*(Dm/m+DV/V)}) g/cm^3")
\end{minted}
L'ouput è:
\begin{verbatim}
d = (5.625 +- 0.141875) g/cm^3
\end{verbatim}
In forma corretta abbiamo: $d = (5,6 +- 0,1) g/cm^3$.

\end{testexample}

\begin{testexample}[\thetcbcounter \,Errore nel moto uniformemente accelerato]

Un'automobile, compie un sorpasso passando dalla velocità iniziale $V1=(89 +-1 )km/h$ alla velocità di $V2=(122 +-1) km/h$, percorrendo una distanza $S = (200,2 +- 0,1 ) m$.  Determina l'accelerazione col suo errore assoluto e scrivi la misura in forma corretta usando la formula: $a = (V2^2 -V1^2)/(2*S)$.

In questo caso abbiamo molte regole da applicare. Anzitutto, il numeratore è una differenza, quindi il suo errore assoluto si scrive: $D(V2^2-V1^2)=D(V2^2) +D(V1^2)$. Ma noi sappiamo calcolare l'errore in una potenza: $D(V2^2) = 2*V2*DV2$ e, analogamente $D(V1^2)=2*V1*DV1$. Chiamiamo il numeratore $NUM$ e il denominatore $DEN$. In questo modo la nostra accelerazione, si scriverà $a = NUM/DEN$, essendo $NUM=V2^2 -V1^2$ e $DEN = 2*S$. Per quanto riguarda gli errori abbiamo:

$D(NUM) = 2*V2*DV2 +2*V1*DV1$. Per il denominatore abbiamo: $DEN = 2*S$ con un errore $D(DEN)= 2*DS$ e, in definitiva:

$Da = a*(D(NUM)/NUM +D(DEN)/DEN)$. Procediamo come prima, inserendo prima i dati e poi li assegnamo alle formule:

\begin{minted}{python}
V1 = 89/3.6
DV1 = 1/3.6
V2 = 122/3.6
DV2 = 1/3.6
S = 80.2
DS = 0.1
NUM=V2*V2-V1*V1
DEN=2*S
DNUM=2*V1*DV1+2*V2*DV2
DDEN=2*DS
a=NUM/DEN
Da=a*(DNUM/NUM+DDEN/DEN)
print(f"s = ({a} +-{Da}) m/s^2")
\end{minted}
Ecco l'output:
\begin{verbatim}
s = (3.3495543548536055 +-0.20717979592289554) m/s^2
\end{verbatim}
In forma corretta abbiamo $a = ( 3,5 +- 0,2) m/s^2$.



\end{testexample}


\section{Elementi di Matplotlib}

\subsection{Grafici a Dispersione}
Matplotlib è una libreria per la creazione di grafici in Python. Esempio di grafico a dispersione:
\begin{lstlisting}[language=Python]
import matplotlib.pyplot as plt
import numpy as np

x = np.random.rand(50)
y = np.random.rand(50)

plt.scatter(x, y)
plt.title("Grafico a Dispersione")
plt.xlabel("X")
plt.ylabel("Y")
plt.savefig('grafico_dispersione.png')
#se vuoi scaricare il file da google colab decommenta le righe:
#from google.colab import files
#files.download('grafico_dispersione.png')
plt.show()
\end{lstlisting}

Includi il grafico nel documento:
\begin{figure}[h!]
    \centering
    \includegraphics[width=0.8\textwidth]{grafico_dispersione.png}
    \caption{Grafico a Dispersione}
    \label{fig:grafico_dispersione}
\end{figure}

\subsection{Grafici con Barre di Errore}
I grafici con barre di errore mostrano l'incertezza nei dati. Esempio:
\begin{lstlisting}[language=Python]
import matplotlib.pyplot as plt
import numpy as np

x = np.linspace(0, 10, 10)
y = np.sin(x)
yerr = 0.2

plt.errorbar(x, y, yerr=yerr, fmt='-o')
plt.title("Grafico con Barre di Errore")
plt.xlabel("X")
plt.ylabel("Y")
plt.savefig('grafico_barre_errore.png')
#se vuoi scaricare il file da google colab decommenta le righe:
#from google.colab import files
#files.download('barre_errore.png')
plt.show()
\end{lstlisting}

Includi il grafico nel documento:
\begin{figure}[h!]
    \centering
    \includegraphics[width=0.8\textwidth]{grafico_barre_errore.png}
    \caption{Grafico con Barre di Errore}
    \label{fig:barre_errore}
\end{figure}

\subsection{Grafico Spazio-Tempo e Velocità-Tempo per Moto Uniforme a Tratti}
Consideriamo un moto uniforme a tratti, dove un oggetto si muove a tre diverse velocità in tre intervalli di tempo distinti. Creiamo due grafici: uno per il moto spaziale-temporale e uno per la velocità-tempo.

\subsubsection{Grafico Spazio-Tempo}
Nel grafico spazio-tempo, mostriamo il percorso dell'oggetto in funzione del tempo. Supponiamo che l'oggetto abbia velocità costanti di 2 m/s, 4 m/s e 6 m/s nei rispettivi intervalli di tempo.

Esempio di codice:
\begin{lstlisting}[language=Python]
import matplotlib.pyplot as plt
import numpy as np

# Dati
tempi = [0, 2, 5, 8, 10]
posizioni = [0, 4, 16, 28, 40]

# Creazione del grafico
plt.figure(figsize=(12, 6))
plt.plot(tempi, posizioni, marker='o')
plt.title("Grafico Spazio-Tempo per Moto Uniforme a Tratti")
plt.xlabel("Tempo (s)")
plt.ylabel("Posizione (m)")
plt.grid(True)
plt.savefig('grafico_spazio_temporale.png')
#se vuoi scaricare il file da google colab decommenta le righe:
#from google.colab import files
#files.download('grafico_spazio_temporale.png')
plt.show()
\end{lstlisting}

\subsubsection{Grafico Velocità-Tempo}
Il grafico velocità-tempo mostra la variazione della velocità in funzione del tempo. Ogni intervallo di tempo corrisponde a una velocità costante.

Esempio di codice:
\begin{lstlisting}[language=Python]
import matplotlib.pyplot as plt
import numpy as np

# Dati
tempi = [0, 2, 5, 8, 10]
velocita = [2, 2, 4, 4, 6]

# Creazione del grafico
plt.figure(figsize=(12, 6))
plt.step(tempi, velocita, where='post', marker='o')
plt.title("Grafico Velocità-Tempo per Moto Uniforme a Tratti")
plt.xlabel("Tempo (s)")
plt.ylabel("Velocità (m/s)")
plt.grid(True)
plt.savefig('grafico_velocita_temporale.png')
#se vuoi scaricare il file da google colab decommenta le righe:
#from google.colab import files
#files.download('grafico_velocita_temporale.png')
plt.show()
\end{lstlisting}

\subsubsection{Spiegazione dei Grafici}
\begin{itemize}
    \item \textbf{Grafico Spazio-Tempo:} Questo grafico mostra come la posizione dell'oggetto cambia nel tempo. È possibile osservare che il grafico è composto da segmenti lineari, ognuno con una pendenza diversa, che rappresenta le diverse velocità. L'area sotto la curva corrisponde alla distanza percorsa.
    \item \textbf{Grafico Velocità-Tempo:} Questo grafico mostra come la velocità cambia nel tempo. Utilizzando un grafico a passo, ogni intervallo di tempo mostra una velocità costante. Il grafico indica chiaramente le transizioni tra le diverse velocità.
\end{itemize}

Includi i grafici nel documento:
\begin{figure}[h!]
    \centering
    \includegraphics[width=0.8\textwidth]{grafico_spazio_temporale.png}
    \caption{Grafico Spazio-Tempo per Moto Uniforme a Tratti}
    \label{fig:spazio_temporale}
\end{figure}

\begin{figure}[h!]
    \centering
    \includegraphics[width=0.8\textwidth]{grafico_velocita_temporale.png}
    \caption{Grafico Velocità-Tempo per Moto Uniforme a Tratti}
    \label{fig:velocita_temporale}
\end{figure}

\subsection{Grafico Altezza-Diametro di un Cilindro con Volume Fisso}
Infine, consideriamo un grafico che mostra come l'altezza di un cilindro varia in funzione del diametro della base, mantenendo fisso il volume.

Esempio di codice:
\begin{lstlisting}[language=Python]
import matplotlib.pyplot as plt
import numpy as np

# Costanti
volume = 1000  # Volume costante del cilindro in m³

# Diametro della base
diametro = np.linspace(1, 10, 100)  # Diametro varia da 1 a 10 metri

# Calcolo dell'altezza
altezza = volume / (np.pi * (diametro / 2)**2)

# Creazione del grafico
plt.figure(figsize=(12, 6))
plt.plot(diametro, altezza, label='Altezza = Volume / (pigreco x (Diametro / 2)^2)', color='green')
plt.title("Grafico dell'Altezza in Funzione del Diametro di Base (Volume Fisso)")
plt.xlabel("Diametro della Base (m)")
plt.ylabel("Altezza (m)")
plt.grid(True)
plt.legend()
plt.savefig('grafico_altezza_diametro.png')
#se vuoi scaricare il file da google colab decommenta le righe:
#from google.colab import files
#files.download('grafico_altezza_diametro.png')
plt.show()
\end{lstlisting}

Includi il grafico nel documento:
\begin{figure}[h!]
    \centering
    \includegraphics[width=0.8\textwidth]{grafico_altezza_diametro.png}
    \caption{Grafico dell'Altezza in Funzione del Diametro di Base (Volume Fisso)}
    \label{fig:altezza_diametro}
\end{figure}
\chapter{Statistica}
\section{Introduzione}
La distribuzione gaussiana, nota anche come distribuzione normale, è una delle più importanti distribuzioni di probabilità in statistica e nelle scienze naturali. Molte grandezze fisiche e biologiche, quando vengono misurate, tendono a seguire una distribuzione normale, soprattutto quando sono influenzate da un gran numero di piccoli effetti casuali indipendenti. Un esempio classico è l'altezza delle persone in una popolazione.

\section{Distribuzione di Probabilità}
In questa e nelle prossime sezioni, parleremo di probabilità. Quando misuriamo una grandezza casuale, vediamo che alcuni valori si ripetono più frequentemente di altri, ossia abbiamo una \textit{distribuzione} di valori. La statistica ci permette di estrarre da questa distribuzione, informazioni sulla grandezza. Si suppone che ogni grandezza abbia un valore ''vero`` e che solo effetti casuali discostino il risultato da quest'ultimo. La distribuzione dei valori viene misurata non tanto dalla semidispersione massima (una stima troppo grande) ma dalla deviazione standard (il nostro famoso errore assoluto nel caso di misure ripetute). Ripetendo le misure molte volte, costruendo un  particolare grafico chiamato \textbf{istogramma delle frequenze}, otteniamo che questo istogramma diventa sempre  più liscio e  si avvicina ad una curva a forma di campana. In fisica, questa curva si presenta quando ripetiamo molte volte una misura casuale. In quel caso, la grandezza non ha sempre lo stesso valore ma ha, appunto, una \textit{distribuzione}. Se facciamo oscillare un pendolo, e misuriamo la durata $t$ di 20 oscillazioni con un cronometro manuale, non otterremo quasi mai due volte lo stesso valore (a causa di errori umani) ma una serie di valori. La statistica ci consente di prevedere quali saranno i valori più probabili e come questi si distribuiscono, (quanti ad esempio sono molto più grandi o più piccoli della media). Le prossime sezioni ci insegneranno come trarre informazioni utili dalla distribuzione di queste misure. Impareremo che il risultato di una misura è dato dalla media e dall'errore della media, come segue:
\[
x=\left(\overline{x} \pm \sigma_{\overline{x}}\right) \, \text{u.m.}
\]
essendo $\overline{x}$ la media dei valori e $\sigma_{\overline{x}}$ la cosiddetta deviazione standard della media.

\section{Cosa sono gli Istogrammi a Bins}
Un istogramma è uno strumento grafico utilizzato per rappresentare la distribuzione di un insieme di dati. Esso divide i dati in intervalli, chiamati \textit{bins}, e conta il numero di eventi (o frequenze) che ricadono in ciascun intervallo.

L'area di ciascun rettangolo nell'istogramma rappresenta la frequenza degli eventi nell'intervallo considerato rispetto al totale degli eventi. Questo significa che l'altezza del rettangolo (quando i bins hanno larghezze uguali) è proporzionale al numero di eventi in quel bin. Se i bins hanno larghezze diverse, l'altezza del rettangolo è proporzionale alla densità di frequenza, in modo che l'area rimanga rappresentativa della frequenza relativa.

\section{Esempio di Istogramma}

Di seguito è riportato un semplice esempio di istogramma generato con Python e Matplotlib.

\begin{figure}[h!]
    \centering
    \includegraphics[width=0.8\textwidth]{histogram_example.png}
    \caption{Esempio di istogramma generato con Python e Matplotlib}
    \label{fig:histogram_example}
\end{figure}

\subsection{Codice Python per Generare l'Istogramma}

\begin{lstlisting}[language=Python, caption=Codice Python per generare un istogramma, label=code:histogram]
import matplotlib.pyplot as plt
import numpy as np

# Generazione di dati casuali
data = np.random.randn(1000)

# Creazione dell'istogramma
plt.hist(data, bins=30, edgecolor='black')

# Aggiunta di titolo e etichette
plt.title('Esempio di Istogramma')
plt.xlabel('Valore')
plt.ylabel('Frequenza')

# Salvataggio dell'istogramma
plt.savefig('histogram_example.png')
#se vuoi scaricare il file da google colab decommenta le righe:
#from google.colab import files
#files.download('histogram_example.png')
plt.show()
\end{lstlisting}

Nel codice sopra, abbiamo generato 1000 dati casuali con una distribuzione normale usando \texttt{numpy}. Questi dati sono stati suddivisi in 30 \textit{bins} per creare l'istogramma. Il parametro \texttt{edgecolor='black'} è utilizzato per disegnare i bordi neri intorno ai rettangoli dell'istogramma, rendendoli più chiari.






\subsection{La Curva di Gauss}

La distribuzione gaussiana è caratterizzata dalla seguente funzione densità di probabilità (pdf):

\begin{equation}
f(x|\mu,\sigma) = \frac{1}{\sigma \sqrt{2\pi}} e^{-\frac{(x-\mu)^2}{2\sigma^2}}
\end{equation}

dove:
\begin{itemize}
    \item $\mu$ è la media della distribuzione, che indica il valore centrale attorno al quale i dati sono distribuiti.
    \item $\sigma$ è la deviazione standard, che misura la dispersione dei dati rispetto alla media.
\end{itemize}

La curva di Gauss ha una forma a campana, simmetrica rispetto alla media $\mu$. La maggior parte dei dati (circa il 68\%) si trova entro un intervallo di una deviazione standard dalla media ($\mu \pm \sigma$), mentre il 95\% dei dati si trova entro due deviazioni standard ($\mu \pm 2\sigma$). Questo comportamento rende la distribuzione gaussiana particolarmente utile per descrivere fenomeni naturali dove le variazioni sono dovute a molti fattori piccoli e indipendenti. In figura \ref{fig:curva_gaussiana} vediamo il tipico aspetto di una gaussiana.

\begin{figure}[h!]
    \centering
     \includegraphics[scale=0.7]{curva_gaussiana.png} 
    \caption{Curva Gaussiana con Evidenziati \(\overline{\mu}\), \(\overline{\mu} - \sigma\), e \(\overline{\mu} + \sigma\)}
    \label{fig:curva_gaussiana}
\end{figure}


\section{Esempio di istogramma sperimentale}

\subsection{Introduzione}
In questo esempio, verranno presentati i dati di un esperimento di misura fisica, la fisica sottostante e il processo per creare un istogramma che rappresenti questi dati utilizzando il linguaggio di programmazione Python con la libreria Matplotlib.

\subsection{Descrizione dell'esperimento}
L'esperimento consiste nel misurare la lunghezza di un campione di barre metalliche. Le misure sono state effettuate utilizzando un calibro digitale con una precisione di 0.1 mm. Di seguito sono riportati i dati sperimentali ottenuti (in cm):

\[
\begin{array}{cccccc}
45.1 & 47.2 & 49.3 & 50.5 & 52.6 & 54.7 \\
48.3 & 46.9 & 51.2 & 53.8 & 50.0 & 49.9 \\
48.7 & 51.5 & 52.1 & 47.6 & 46.3 & 50.9 \\
51.8 & 48.0 & 49.5 & 50.3 & 47.0 & 46.5 \\
52.4 & 48.8 & 49.2 & 51.3 & 47.8 & 50.7 \\
\end{array}
\]
Nel contesto della statistica, useremo solo programmi software e ci disinteresseremo dei problemi relativi alle cifre significative durante i calcoli, mentre queste saranno importanti nel momento in cui scriveremo il risultato delle misure.
\subsection{Fisica dell'esperimento}
La misura della lunghezza delle barre metalliche è un esperimento comune in fisica per studiare le proprietà dei materiali. La lunghezza può variare a causa di fattori come:
\begin{itemize}
    \item Differenze nel processo di produzione.
    \item Espansione termica a diverse temperature.
    \item Errori sistematici e casuali durante la misurazione.
\end{itemize}
L'istogramma delle misure permette di visualizzare la distribuzione delle lunghezze e di identificare eventuali deviazioni significative dalla media.

\subsection{Creazione dell'istogramma con Python}
Per creare l'istogramma, utilizziamo il seguente codice Python:

\begin{lstlisting}[language=Python, caption=Codice Python per creare l'istogramma]
import matplotlib.pyplot as plt

# Dati sperimentali
dati_sperimentali = [45.1, 47.2, 49.3, 50.5, 52.6, 54.7, 48.3, 46.9, 51.2, 53.8, 
                     50.0, 49.9, 48.7, 51.5, 52.1, 47.6, 46.3, 50.9, 51.8, 48.0, 
                     49.5, 50.3, 47.0, 46.5, 52.4, 48.8, 49.2, 51.3, 47.8, 50.7]

# Crea l'istogramma
plt.figure(figsize=(10, 6))
plt.hist(dati_sperimentali, bins=8, edgecolor='black', alpha=0.7)

# Aggiungi titolo e etichette
plt.title('Istogramma dei dati sperimentali')
plt.xlabel('Misura (cm)')
plt.ylabel('Frequenza')

# Mostra l'istogramma
plt.savefig('istogramma2.png')  # Salva l'immagine come istogramma2.png
#se vuoi scaricare il file da google colab decommenta le righe:
#from google.colab import files
#files.download('istogramma2.png')
plt.show()
\end{lstlisting}

\subsubsection{Spiegazione del codice}
Il codice Python utilizzato per creare l'istogramma è spiegato di seguito:

\paragraph{Importazione delle librerie}
\begin{lstlisting}[language=Python, caption=Importazione delle librerie]
import matplotlib.pyplot as plt
\end{lstlisting}
Questo codice importa la libreria Matplotlib, necessaria per la creazione di grafici.

\paragraph{Definizione dei dati}
\begin{lstlisting}[language=Python, caption=Definizione dei dati]
dati_sperimentali = [45.1, 47.2, 49.3, 50.5, 52.6, 54.7, 48.3, 46.9, 51.2, 53.8, 
                     50.0, 49.9, 48.7, 51.5, 52.1, 47.6, 46.3, 50.9, 51.8, 48.0, 
                     49.5, 50.3, 47.0, 46.5, 52.4, 48.8, 49.2, 51.3, 47.8, 50.7]
\end{lstlisting}
Definisce un array contenente i dati delle misurazioni fisiche.

\paragraph{Creazione dell'istogramma}
\begin{lstlisting}[language=Python, caption=Creazione dell'istogramma]
plt.figure(figsize=(10, 6))
plt.hist(dati_sperimentali, bins=8, edgecolor='black', alpha=0.7)
\end{lstlisting}
Imposta la dimensione della figura e crea un istogramma con 8 bin, bordi neri per i bin e trasparenza del 70\%.

\paragraph{Aggiunta di titolo e etichette}
\begin{lstlisting}[language=Python, caption=Aggiunta di titolo e etichette]
plt.title('Istogramma dei dati sperimentali')
plt.xlabel('Misura (cm)')
plt.ylabel('Frequenza')
\end{lstlisting}
Aggiunge il titolo e le etichette agli assi del grafico.

\paragraph{Visualizzazione e salvataggio dell'istogramma}
\begin{lstlisting}[language=Python, caption=Visualizzazione e salvataggio dell'istogramma]
plt.savefig('istogramma2.png')  # Salva l'immagine come istogramma2.png
plt.show()
\end{lstlisting}
Visualizza l'istogramma generato e lo salva come `istogramma2.png`.

\subsection{Risultati}
L'istogramma risultante mostra la distribuzione delle lunghezze delle barre metalliche. L'analisi visiva dell'istogramma permette di identificare la variabilità delle misurazioni e la presenza di eventuali outlier.

\begin{figure}[!htbp] 
    \centering
    \includegraphics[width=0.8\textwidth]{istogramma2.png}
    \caption{Istogramma dei dati sperimentali}
    \label{fig:istogramma2}
\end{figure}



\section{Approssimazione della Gaussiana}
In un esperimento, se ripetiamo molte volte una misura e calcoliamo la media e la deviazione standard, al crescere del numero di misure, l'istogramma dei dati approssima sempre meglio la distribuzione gaussiana. Questo è dovuto al teorema centrale del limite, il quale afferma tra l'altro che le formule per media e deviazione standard della media, se applicate a campioni di dati molto grandi, ci restituiscono proprio i valori teorici di queste grandezze che restano comunque un concetto teorico. Quando una persona fà un esperimento di misura, a quella persona e quell'apparato, corrisponde una media teorica, ossia i valori delle grandezze si sparpaglieranno in un certo modo perché quella persona e quell'apparato hanno una sorta di sensibilità: se qualcun altro fà la misura, questa si \textit{sparpaglierà} diversamente. Ad esempio, se Marco lascia cadere un pallina mille volte (povero Marco... ) e misura la media e la deviazione standard dei tempi di caduta, magari otterrà una deviazione standard di 0,8 s e una media di 0,2 s. Se facciamo cadere la pallina usando una fotocellula per misurare il tempo, le misure si sparpaglieranno di meno e avremo una deviazione standard ad esempio di 0,01 s con una media di 0,16 s. Come si vede, la grandezza da misurare (il tempo di caduta) è la stessa ma uno dei sue sistemi è più preciso (quello con la fotocellula). Se andassimo a costruire gli istogrammi sperimentali, quello della misura del ragazzo, avrebbe una forma a campana più larga perché la deviazione standard misura tra l'altro quanto è largo l'istogramma. 

\section{Stima di Media e Deviazione Standard}
Data una serie di $n$ misurazioni sperimentali $\{x_1, x_2, \ldots, x_n\}$, possiamo stimare i parametri della distribuzione gaussiana, cioè la media $\mu$ e la deviazione standard $\sigma$, utilizzando le seguenti formule:

\subsection{Calcolo della Media}
La media campionaria $\hat{\mu}$ è data dalla somma di tutte le osservazioni divisa per il numero totale di osservazioni:

\begin{equation}
\hat{\mu} = \frac{1}{n} \sum_{i=1}^{n} x_i
\end{equation}

\subsection{Calcolo della Deviazione Standard}
La deviazione standard campionaria $\hat{\sigma}$ è data dalla radice quadrata della somma dei quadrati delle differenze tra ciascuna osservazione e la media campionaria, divisa per il numero di osservazioni meno uno:

\begin{equation}
\hat{\sigma} = \sqrt{\frac{1}{n-1} \sum_{i=1}^{n} (x_i - \hat{\mu})^2}
\end{equation}

\subsection{Calcolo della Deviazione Standard della Media}
La deviazione standard della media, nota anche come errore standard della media, è calcolata come:

\begin{equation}
\sigma_{\overline{x}} = \frac{\hat{\sigma}}{\sqrt{n}}
\end{equation}

Dove $\hat{\sigma}$ è la deviazione standard campionaria e $n$ è il numero di osservazioni. La deviazione standard della media rappresenta quanto la media campionaria è attesa essere distante dalla media vera della popolazione. Per una stima più precisa, questa deviazione standard della media viene approssimata a una cifra significativa.

\section{Esempio Pratico in Python}
Per illustrare questi concetti, consideriamo un esempio pratico in Python. Abbiamo misurato le altezze di 100 persone (in cm) e calcoleremo la media, la deviazione standard e la deviazione standard della media di queste misure. Successivamente, creeremo un istogramma delle altezze e lo sovrapporremo con la curva gaussiana corrispondente.

\begin{lstlisting}[language=Python, caption={Script Python per calcolare e visualizzare le altezze}]
import numpy as np
import matplotlib.pyplot as plt
from scipy.stats import norm

# Altezze di 100 persone (in cm)
heights = [170, 165, 180, 175, 160, 155, 178, 172, 168, 169, 
           174, 167, 166, 171, 173, 177, 182, 181, 176, 179, 
           164, 163, 162, 161, 159, 158, 157, 156, 154, 153, 
           152, 151, 150, 149, 148, 147, 146, 145, 144, 143,
           150, 155, 160, 165, 170, 175, 180, 185, 190, 195,
           172, 177, 182, 187, 192, 197, 162, 167, 172, 177,
           180, 175, 170, 165, 160, 155, 150, 145, 140, 135,
           142, 147, 152, 157, 162, 167, 172, 177, 182, 187,
           165, 170, 175, 180, 185, 190, 195, 200, 205, 210]

# Stima di media, deviazione standard e deviazione standard della media
mu = np.mean(heights)
sigma = np.std(heights, ddof=1)
sigma_x_mean = sigma / np.sqrt(len(heights))

# Arrotonda la deviazione standard della media e la media alle unità
sigma_x_mean_rounded = round(sigma_x_mean)
mu_rounded = round(mu)

print(f"Media delle altezze: {mu_rounded} cm")
print(f"Deviazione standard delle altezze: {sigma:.2f} cm")
print(f"Deviazione standard della media: {sigma_x_mean_rounded} cm")

# Creazione dell'istogramma
count, bins, ignored = plt.hist(heights, bins=6, density=True, alpha=0.6, color='g', edgecolor='black')

# Sovrapposizione della curva gaussiana
xmin, xmax = plt.xlim()
x = np.linspace(xmin, xmax, 100)
p = norm.pdf(x, mu, sigma)
plt.plot(x, p, 'k', linewidth=2)
title = "Istogramma delle Altezze e Curva Gaussiana"
plt.title(title)

# Visualizzazione di media e deviazione standard
plt.axvline(mu, color='r', linestyle='dashed', linewidth=1)
plt.text(mu + mu/10, max(p), f'Media: {mu_rounded} cm', color='r')
plt.axvline(mu + sigma, color='b', linestyle='dashed', linewidth=1)
plt.axvline(mu - sigma, color='b', linestyle='dashed', linewidth=1)
plt.text(mu + sigma + mu/10, max(p)/2, f'Sigma: {sigma:.2f} cm', color='b')
plt.text(mu - sigma - mu/10, max(p)/2, f'Sigma: {sigma:.2f} cm', color='b')

# Impostazione dei valori sui bins come etichette sull'asse x
bin_labels = [f"{int(bins[i])}" for i in range(len(bins))]
plt.xticks(bins, labels=bin_labels, rotation=45)

# Migliora la disposizione dei sottotitoli e delle etichette
plt.tight_layout()

plt.savefig('istogramma.png')
#se vuoi scaricare il file da google colab decommenta le righe:
#from google.colab import files
#files.download('istogramma.png')
plt.show()

# Risultato finale, arrotondato a due cifre significative
print(f"Altezza = ({mu_rounded} +- {sigma_x_mean_rounded}) cm")
\end{lstlisting}

\subsection{Significato delle Variabili, Moduli, Funzioni e Parametri}
Spieghiamo brevemente il significato delle variabili, dei moduli usati, delle funzioni e dei parametri:

\begin{itemize}
    \item \texttt{numpy} (\texttt{np}): Una libreria per il calcolo numerico in Python. Utilizzata per calcolare la media (\texttt{np.mean}) e la deviazione standard (\texttt{np.std}) delle altezze.
    \item \texttt{matplotlib.pyplot} (\texttt{plt}): Una libreria per la creazione di grafici. Utilizzata per creare l'istogramma (\texttt{plt.hist}) e sovrapporre la curva gaussiana (\texttt{plt.plot}).
    \item \texttt{scipy.stats.norm}: Fornisce la funzione di densità di probabilità per una distribuzione normale. Utilizzata per calcolare la curva gaussiana da sovrapporre all'istogramma.
    \item \texttt{plt.hist()}: Funzione per creare un istogramma. Il parametro \texttt{bins} definisce il numero di intervalli.
    \item \texttt{plt.plot()}: Funzione per tracciare una linea su un grafico. Utilizzata per disegnare la curva gaussiana.
    \item \texttt{np.linspace()}: Funzione per generare una sequenza di numeri spaziati uniformemente. Utilizzata per generare i valori x della curva gaussiana.
\end{itemize}

Ecco l'output:
\begin{verbatim}
Media delle altezze: 167.83333333333334 cm
Deviazione standard delle altezze: 15.80 cm
Deviazione standard della media: 1.67 cm
Altezza = (168 +- 2) cm
\end{verbatim}
Nel grafico \ref{fig:istogramma} vediamo sovrapposto l'istogramma costruito e la curva che lo approssima. L'istogramma ha in ascisse (asse X) gli intervalli di altezza e in ordinate (asse Y) un valore tale che l'area del bin sia uguale alla frazione di persone che hanno un'altezza compresa tra i suoi estremi. Ad esempio, se guardiamo l'intervallo tra 160 e 172, l'altezza è 0,025 perché $\left(172-160 \right)\times 0,025 =0,3$ ossia, il 30\% delle persone aveva un'altezza compresa tra 160 e 172 centimetri. Ancora un commento su media e deviazione standard. Notare che la curva è centrata attorno alla media (il valore $\SI{167,3}{\centi\meter}$) evidenziata dalla linea rossa tratteggiata. Notare anche le due linee blu. La distanza tra la linea verde e quella blu è proprio uguale alla deviazione standard (pari a $\SI{15,80}{\centi\meter}$ e indicata sul grafico come \textit{Sigma}), infatti $\SI{167,83}{\centi\meter} + \SI{15,80}{\centi\meter} = \SI{183,63}{\centi\meter}$ che è proprio dove si trova la linea verde. Notiamo infine che il 68\% delle misure capitano tra i valori $\SI{167,83}{\centi\meter} -\SI{15,80}{\centi\meter}$ e $\SI{167,83}{\centi\meter} +\SI{15,80}{\centi\meter}$, ossia tra $\SI{152,03}{\centi\meter}$ e $\SI{183,63}{\centi\meter}$. Questo è sempre vero. Quando abbiamo una grandezza che si distribuisce come la curva a campana (la gaussiana scoperta dallo scienziato Gauss) l'area sotto la curva, quella compresa tra le due linee blu, è sempre 0,68 ossia il 68\% delle misure che la approssimano, dovrebbero capitare tra $\overline{x} -\sigma$ e $\overline{x} + \sigma$.

\begin{figure}[h!]
    \centering
    \includegraphics[width=0.8\textwidth]{istogramma.png}
    \caption{Istogramma delle Altezze con Sovrapposta la Curva Gaussiana}
    \label{fig:istogramma}
\end{figure}


\section{Uso di Matplotlib per la Visualizzazione dei Dati}

Matplotlib è una libreria potente per la creazione di grafici in Python. Di seguito è riportato un esempio di come realizzare un grafico delle misure sperimentali con barre di errore.

\subsection{Esempio di Grafico}

Consideriamo l'esempio dell'accelerazione di gravità con i seguenti dati:

\[
\begin{array}{c|c}
\text{Tempo (s)} & \text{Distanza (m)} \\
\hline
1.0 & 4.900 \\
2.0 & 19.600 \\
3.0 & 44.100 \\
4.0 & 78.400 \\
5.0 & 122.500 \\
\end{array}
\]

Per mostrare le barre di errore, inseriamo nel codice errori volutamente molto grandi. Si noti inoltre che abbiamo modificato la grandezza dei punti col parametro \textbf{markersize}, la larghezza dei segmenti per le barre con il parametro \textbf{capsize}, e la dimensione del grafico con l'istruzione \textbf{plt.figure(figsize=(12, 9))}.

\begin{lstlisting}[caption={Grafico delle misure sperimentali con barre di errore}]
import numpy as np
import numpy as np
import matplotlib.pyplot as plt

# Dati sperimentali
tempo = np.array([1.0, 2.0, 3.0, 4.0, 5.0])
distanza = np.array([4.900, 19.600, 44.100, 78.400, 122.500])
error_distanza = np.array([1, 2, 3, 1, 4])  # Errori aumentati
error_t2 = np.array([1, 1, 1, 1, 1])

# Configurazione per usare LaTeX in Matplotlib


# Creazione del grafico con dimensioni maggiori e punti più piccoli
plt.figure(figsize=(12, 9))  # Dimensioni della figura
plt.errorbar(tempo**2, distanza, yerr=error_distanza, xerr=error_t2, fmt='o', label='Dati sperimentali', capsize=5, elinewidth=2, markersize=6)
plt.title(r"Misurazione dell'Accelerazione di gravità", fontsize=16)
plt.xlabel(r'Tempo al quadrato (s^2)', fontsize=14)
plt.ylabel(r'Distanza (m)', fontsize=14)
plt.legend()
plt.grid(True)
plt.savefig('grafico_misure.png')  # Salvataggio dell'immagine
#se vuoi scaricare il file da google colab decommenta le righe:
#from google.colab import files
#files.download('grafico_misure.png')
plt.show()



\end{lstlisting}

Nella figura \ref{fig:grafico_misure} è mostrato il grafico risultante.

\begin{figure}[h!]
    \centering
    \includegraphics[width=\textwidth]{grafico_misure.png}
    \caption{Grafico delle misure sperimentali con barre di errore.}
    \label{fig:grafico_misure}
\end{figure}


\section{Regressione Lineare}
In fisica, il moto uniforme è un tipo di moto in cui un oggetto si sposta con velocità costante. Questo significa che la distanza percorsa dall'oggetto è proporzionale al tempo impiegato. La relazione tra spazio \( y \) e tempo \( x \) in un moto uniforme può essere espressa tramite la seguente equazione lineare:

\[
y = v \cdot x + s_0
\]

dove:
\begin{itemize}
    \item \( v \) è la velocità costante dell'oggetto (pendenza della retta).
    \item \( s_0 \) è la posizione iniziale dell'oggetto (intercetta della retta).
\end{itemize}

L'obiettivo è trovare i parametri \( v \) e \( s_0 \) che meglio rappresentano i dati sperimentali. Utilizzando una regressione lineare, possiamo ottenere questi parametri adattando una retta ai dati.

\subsection{Tabella dei Dati}

Per l'analisi, consideriamo i seguenti dati sperimentali raccolti per tempo e spazio:

\begin{table}[h!]
    \centering
    \begin{tabular}{|c|c|}
    \hline
    \textbf{Tempo (\si{\second})} & \textbf{Spazio (\si{\meter})} \\
    \hline
    1.0 & 2.0 \\
    2.0 & 4.0 \\
    3.0 & 6.0 \\
    4.0 & 8.0 \\
    5.0 & 10.0 \\
    \hline
    \end{tabular}
    \caption{Dati sperimentali di tempo e spazio.}
    \label{tab:dati}
\end{table}

\subsection{Regressione Lineare}

La regressione lineare cerca di adattare una retta ai dati sperimentali, trovando i parametri \( a \) e \( b \) che minimizzano la somma dei quadrati delle differenze tra i valori osservati e quelli previsti dalla retta. In questo caso, la retta di regressione è data da:

\[
y = a \cdot x + b
\]

dove:
\begin{itemize}
    \item \( a \) è la pendenza della retta, che rappresenta la velocità \( v \).
    \item \( b \) è l'intercetta, che rappresenta la posizione iniziale \( s_0 \).
\end{itemize}

\subsection{Uso di Python per la Regressione Lineare}
Per calcolare i parametri della retta di regressione e il loro errore, utilizziamo il modulo \texttt{scipy.optimize.curve\_fit} di Python, che permette di adattare una funzione ai dati sperimentali. La funzione \texttt{curve\_fit} ritorna i parametri ottimizzati e le loro deviazioni standard. Utilizziamo anche \texttt{matplotlib} per visualizzare i dati e la retta di regressione.


Ecco il codice Python utilizzato:

\begin{lstlisting}[caption={Semplice regressione lineare}]
import numpy as np
from scipy.optimize import curve_fit
import matplotlib.pyplot as plt
# Dati sperimentali
tempo = np.array([1.0, 2.0, 3.0, 4.0, 5.0])
spazio = np.array([1.9, 4.1, 6.0, 7.7, 12.0])

# Definizione della funzione di modello lineare
def linear_model(x, a, b):
    return a * x + b

# Fitting dei dati
params, params_covariance = curve_fit(linear_model, tempo, spazio)

# Estrazione dei parametri
pendenza, intercetta = params
pendenza_error, intercetta_error = np.sqrt(np.diag(params_covariance))

print(f"Pendenza: {pendenza:.3f} +- {pendenza_error:.3f}")
print(f"Intercetta: {intercetta:.3f} +- {intercetta_error:.3f}")

# Creazione del grafico
plt.figure(figsize=(8, 6))
plt.scatter(tempo, spazio, label='Dati sperimentali')
plt.plot(tempo, linear_model(tempo, *params), 'r-', label='Retta di regressione')
plt.xlabel('Tempo (s)')
plt.ylabel('Spazio (m)')
plt.title('Fitting Lineare')
plt.legend()
plt.grid(True)
plt.savefig('regressione_lineare.png')  # Salvataggio dell'immagine
#se vuoi scaricare il file da google colab decommenta le righe:
#from google.colab import files
#files.download('regressione_lineare.png')
plt.show()
\end{lstlisting}

\subsection{Output del Codice Python}

L'output del codice Python è il seguente:

\begin{mdframed}[backgroundcolor=lightgray, linecolor=black, linewidth=1pt]
\textbf{Pendenza}: \(2.020 \pm 0.083 \, \si{\meter\per\second}\) \\
\textbf{Intercetta}: \(0.200 \pm 0.276 \, \si{\meter}\)
\end{mdframed}

\textbf{Pendenza}: La pendenza della retta di regressione è \(2.020 \pm 0.083 \, \si{\meter\per\second}\). Questo valore rappresenta la velocità media del moto. L'errore associato indica l'incertezza nella determinazione della pendenza, che riflette la variabilità dei dati sperimentali rispetto alla retta di regressione.

\textbf{Intercetta}: L'intercetta della retta di regressione è \(0.200 \pm 0.276 \, \si{\meter}\). Questo valore rappresenta la posizione iniziale, cioè il valore di \(y\) quando \(x\) è zero. L'errore associato all'intercetta indica l'incertezza nella misura del punto in cui la retta di regressione interseca l'asse delle ordinate. Questo valore può dare indicazioni sul punto di partenza del movimento descritto dai dati.



Di seguito , in figura \ref{fig:regressione_lineare} è mostrato il grafico della regressione lineare che visualizza i dati sperimentali insieme alla retta di regressione ottenuta.

\begin{figure}[h!]
    \centering
    \includegraphics[width=0.8\textwidth]{regressione_lineare.png}
    \caption{Grafico della regressione lineare: i dati sperimentali sono mostrati come punti, e la retta di regressione è mostrata in rosso.}
    \label{fig:regressione_lineare}
\end{figure}


\section{Esempio complesso}
Supponiamo di avere i dati sperimentali nella tabella \ref{tab:regrcompl} che rappresenta i tempi di caduta di un oggetto da diverse altezze:
\begin{table}[h!]
\centering
\begin{tabular}{|c|c|}
\hline
Tempo (s) & Distanza (m) \\
\hline
1.0 & 4.9 \\
2.0 & 19.6 \\
3.0 & 44.1 \\
4.0 & 78.4 \\
5.0 & 122.5 \\
\hline
\end{tabular}
\caption{Dati di un moto di caduta.}
\label{tab:regrcompl}
\end{table}



\subsection{Analisi dei Dati}
Prima di applicare il best fit lineare, calcoliamo il tempo al quadrato \( t^2 \). Infatti la formula del moto di caduta:
\[
s=\frac{1}{2}g\cdot t^2
\]
ci dice che se pongo $t^2 =t2$, la relazione tra $s$ e  $t2$ è lineare perché la formula si scrive:
\[
s=\frac{1}{2} g \cdot t2
\]
mentre quella tra $t$ ed $s$ è quadratica. A questo punto, se poniamo  $b=\frac{g}{2}$, la relazione si può scrivere:
\[
s=b\cdot t2
\]
essendo quindi una relazione senza parametro intercetta. Una volta trovato $b$, determiniamo l'accelerazione di gravità da una formula inversa.
\[
g = 2\cdot b
\]
e il suo errore propagato vale:
\[\sigma_g = 2\cdot \sigma_b
\]
essendo $\sigma_b$ l'errore sul parametro $b$ (nel codice, gli errori sulle $y$, sul parametro $a$ e sul parametro $b$ si ottengono dal vettorem
\begin{center}
\begin{tabular}{|c|c|}
\hline
Tempo (s) & Tempo al quadrato (s\(^2\)) \\
\hline
1.0 & 1.0 \\
2.0 & 4.0 \\
3.0 & 9.0 \\
4.0 & 16.0 \\
5.0 & 25.0 \\
\hline
\end{tabular}
\end{center}

Il codice Python per eseguire la regressione lineare e calcolare l'accelerazione di gravità è:

\begin{lstlisting}[caption={Calcolo della Regressione Lineare e Accelerazione di Gravità}]
import numpy as np
from scipy.optimize import curve_fit
import matplotlib.pyplot as plt

# Dati sperimentali
tempo = np.array([1.0, 2.0, 3.0, 4.0, 5.0])
distanza = np.array([4.9, 19.6, 44.1, 78.4, 122.5])
error_distanza = np.array([0.1, 0.1, 0.1, 0.1, 0.1])

# Calcolo di t^2
tempo_squared = tempo**2

# Definizione della funzione di modello lineare senza intercetta
def linear_model(t_squared, g_over_2):
    return g_over_2 * t_squared

# Regressione lineare forzata attraverso l'origine
params, covariance = curve_fit(linear_model, tempo_squared, distanza, sigma=error_distanza, absolute_sigma=True)
g_over_2 = params[0]

# Calcolo dell'accelerazione di gravità
g = 2 * g_over_2

# Calcolo dell'errore associato
std_err = np.sqrt(np.diag(covariance))
g_error = 2 * std_err[0]


print(f"Accelerazione di gravità (g): {g:.2f} +- {g_error:.2f} m/s^2")

# Creazione del grafico
plt.figure(figsize=(8, 6))
plt.errorbar(tempo_squared, distanza, yerr=error_distanza, fmt='o', label='Dati sperimentali', capsize=5)
plt.plot(tempo_squared, linear_model(tempo_squared, *params), 'r-', label='Retta di regressione')
plt.title('Misurazione dell\'Accelerazione di Gravità')
plt.xlabel('Tempo al quadrato (s^2)')
plt.ylabel('Distanza (m)')
plt.legend()
plt.grid(True)
plt.savefig('regressione2.png')
#se vuoi scaricare il file da google colab decommenta le righe:
#from google.colab import files
#files.download('regressione2.png')
plt.show()


\end{lstlisting}


Il grafico è mostrato in figura.\ref{fig:g}. L'output dello script è:
\begin{verbatim}
Accelerazione di gravità (g): 9.80 +- 0.01 m/s^2
\end{verbatim}

Segue spiegazione del codice.




\begin{figure}[!htbp] 
    \centering
\includegraphics[scale=0.6]{g.png} 
    \caption{Grafico della regressione lineare per il calcolo dell'accelerazione dalla pendenza.}
    \label{fig:g}
\end{figure}

\section{Dettagli sul codice}

\subsection{Dati Sperimentali}
I dati di tempo e distanza sono memorizzati nei seguenti array:

\begin{itemize}
    \item \texttt{tempo}: Array contenente i valori di tempo (in secondi).
    \item \texttt{distanza}: Array contenente i valori di distanza (in metri) misurati.
    \item \texttt{error\_distanza}: Array contenente gli errori associati alle misure di distanza (in metri).
\end{itemize}

\subsection{Calcolo di \texorpdfstring{$t^2$}{t²}}
Per adattare il modello \( d = \frac{1}{2} g t^2 \), è necessario calcolare il tempo al quadrato. Questo viene fatto con:

\begin{lstlisting}
# Calcolo del tempo al quadrato
tempo_squared = tempo**2
\end{lstlisting}

Il vettore \texttt{tempo\_squared} contiene i valori del tempo al quadrato, che sono usati per la regressione lineare.

\subsection{Definizione della Funzione di Modello}
Definiamo una funzione lineare che rappresenta la relazione tra distanza e tempo al quadrato:

\begin{lstlisting}
# Definizione della funzione lineare
def linear_model(t_squared, g_over_2):
    return g_over_2 * t_squared
\end{lstlisting}

Questa funzione modella la relazione \( d = \frac{1}{2} g t^2 \) e assume che l'intercetta sia zero. \texttt{g\_over\_2} rappresenta la metà dell'accelerazione di gravità.

\subsection{Regressione Lineare}
Utilizziamo la funzione \texttt{curve\_fit} per eseguire la regressione lineare. I parametri e la covarianza sono ottenuti da:

\begin{lstlisting}
# Regressione lineare forzata attraverso l'origine
params, covariance = curve_fit(linear_model, tempo_squared, distanza, sigma=error_distanza, absolute_sigma=True)
g_over_2 = params[0]
\end{lstlisting}

\begin{itemize}
    \item \texttt{curve\_fit}: Funzione che calcola i parametri migliori per adattare il modello ai dati.
    \item \texttt{linear\_model}: La funzione di modello lineare definita in precedenza.
    \item \texttt{tempo\_squared}: Array contenente i valori di tempo al quadrato.
    \item \texttt{distanza}: Array contenente i valori di distanza misurati.
    \item \texttt{sigma=error\_distanza}: Specifica gli errori associati alle misure di distanza.
    \item \texttt{absolute\_sigma=True}: Indica che gli errori forniti sono errori assoluti.
    \item \texttt{params}: Array contenente i parametri stimati (in questo caso, \texttt{g\_over\_2}).
    \item \texttt{covariance}: Matrice di covarianza dei parametri stimati, utilizzata per calcolare l'incertezza associata.
\end{itemize}

\subsection{Calcolo dell'Accelerazione di Gravità}
L'accelerazione di gravità \( g \) è calcolata come:

\begin{lstlisting}
# Calcolo di g
g = 2 * g_over_2
\end{lstlisting}

Poiché il nostro modello è \( d = \frac{1}{2} g t^2 \), la pendenza \( g\_over\_2 \) è la metà dell'accelerazione di gravità.

\subsection{Calcolo dell'Errore Associato}
L'errore associato a \( g \) è calcolato utilizzando la covarianza dei parametri:

\begin{lstlisting}
# Calcolo dell'errore associato
std_err = np.sqrt(np.diag(covariance))
g_error = 2 * std_err[0]

\end{lstlisting}

\begin{itemize}
    \item \texttt{np.sqrt(np.diag(covariance))}: Calcola l'errore standard dei parametri stimati.
    \item \texttt{std\_err}: Array contenente gli errori standard.
    \item \texttt{g\_error}: L'errore associato all'accelerazione di gravità \( g \), calcolato come il doppio dell'errore della pendenza \( g/2 \).
\end{itemize}

\subsection{Creazione del Grafico}
Il grafico visualizza i dati sperimentali e la retta di regressione. Le barre d'errore sono mostrate per indicare l'incertezza nella misura della distanza:

\begin{lstlisting}
# Creazione del grafico
plt.figure(figsize=(8, 6))
plt.errorbar(tempo_squared, distanza, yerr=error_distanza, fmt='o', label='Dati sperimentali', capsize=5)
plt.plot(tempo_squared, linear_model(tempo_squared, *params), 'r-', label='Retta di regressione')
plt.title('Misurazione dell\'Accelerazione di Gravità')
plt.xlabel('Tempo al quadrato (s^2)')
plt.ylabel('Distanza (m)')
plt.legend()
plt.grid(True)
plt.show()
\end{lstlisting}

\begin{itemize}
    \item \texttt{plt.figure(figsize=(8, 6))}: Crea una nuova figura di dimensioni 8x6 pollici.
    \item \texttt{plt.errorbar}: Crea un grafico a dispersione con barre d'errore per ogni punto dati.
    \item \texttt{tempo\_squared}: Array con i valori di tempo al quadrato.
    \item \texttt{distanza}: Array con i valori di distanza misurati.
    \item \texttt{yerr=error\_distanza}: Specifica gli errori associati alle misure di distanza.
    \item \texttt{fmt='o'}: Specifica che i punti dati sono rappresentati come cerchi.
    \item \texttt{label='Dati sperimentali'}: Etichetta per i dati sperimentali.
    \item \texttt{capsize=5}: Lunghezza delle barre orizzontali alle estremità delle barre d'errore.
    \item \texttt{plt.plot}: Disegna la retta di regressione.
    \item \texttt{linear\_model(tempo\_squared, *params)}: Valori previsti dal modello lineare.
    \item \texttt{'r-'}: Specifica che la retta di regressione è rossa e solida.
    \item \texttt{label='Retta di regressione'}: Etichetta per la retta di regressione.
    \item \texttt{plt.title}: Aggiunge il titolo al grafico.
    \item \texttt{plt.xlabel}: Aggiunge l'etichetta all'asse delle ascisse.
    \item \texttt{plt.ylabel}: Aggiunge l'etichetta all'asse delle ordinate.
    \item \texttt{plt.legend()}: Mostra la legenda nel grafico.
    \item \texttt{plt.grid(True)}: Aggiunge una griglia al grafico.
    \item \texttt{plt.show()}: Visualizza il grafico.
\end{itemize}


\section{Conclusioni}

In questo capitolo, abbiamo esplorato tre argomenti fondamentali nell'ambito dell'analisi dei dati e della statistica: le distribuzioni di probabilità e gli istogrammi, la regressione lineare e l'uso di script Python per analizzare questi argomenti. 

\subsection{Distribuzioni di Probabilità e Istogrammi}

Abbiamo iniziato con un'analisi delle distribuzioni di probabilità, che ci permettono di descrivere come i valori di una variabile casuale sono distribuiti. Gli istogrammi sono stati introdotti come uno strumento visivo per rappresentare queste distribuzioni, consentendo di vedere facilmente la frequenza di occorrenza dei valori dei dati all'interno di specifici intervalli. Attraverso esempi pratici, abbiamo visto come costruire istogrammi utilizzando Python e Matplotlib, e come interpretare l'area di ciascun bin in termini di frequenza relativa.

\subsection{Regressione Lineare}

Successivamente, abbiamo affrontato la regressione lineare, una tecnica statistica utilizzata per modellare la relazione tra una variabile dipendente e una o più variabili indipendenti. Abbiamo discusso il concetto di minimizzazione dell'errore quadratico medio per trovare la linea di regressione che meglio approssima i dati osservati. 

\subsection{Script Python per l'Analisi}

Infine, abbiamo integrato questi concetti attraverso l'uso di script Python. Utilizzando librerie come NumPy, SciPy, Matplotlib, abbiamo visto come implementare praticamente le tecniche di analisi dei dati. Gli esempi di codice forniti hanno mostrato come generare istogrammi, calcolare la linea di regressione, e visualizzare i risultati in modo chiaro e intuitivo.

\subsection{Sintesi e Prospettive Future}

In sintesi, questo capitolo ha fornito una panoramica delle tecniche di base per l'analisi statistica dei dati, combinando teoria e pratica. La comprensione delle distribuzioni di probabilità e l'uso degli istogrammi sono fondamentali per qualsiasi analisi dei dati esplorativa. La regressione lineare rappresenta uno strumento potente per identificare e quantificare le relazioni tra variabili. L'implementazione di queste tecniche attraverso script Python non solo rafforza la comprensione teorica, ma fornisce anche competenze pratiche essenziali per l'analisi dei dati nel mondo reale.



\end{document}





% Local Variables:
% TeX-engine: xetex
% End:
